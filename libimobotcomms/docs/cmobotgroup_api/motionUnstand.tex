\noindent
\vspace{5pt}
\rule{4.5in}{0.015in}\\
\noindent
{\LARGE \texttt{CMobotGroup::motionUnstand()}\index{CMobotGroup::motionUnstand()}}\\
{\LARGE \texttt{CMobotGroup::motionUnstandNB()}\index{CMobotGroup::motionUnstandNB()}}\\
%\phantomsection
\addcontentsline{toc}{section}{motionUnstand()}
\addcontentsline{toc}{section}{motionUnstandNB()}

\noindent
{\bf Synopsis}
\vspace{-8pt}
\begin{verbatim}
#include <mobot.h>
int CMobotGroup::motionUnstand();
int CMobotGroup::motionUnstandNB();
\end{verbatim}

\noindent
{\bf Purpose}\\
Move robots currently standing on a faceplate back down into a prone position.\\

\noindent
{\bf Return Value}\\
The function returns 0 on success and non-zero otherwise.\\

\noindent
{\bf Parameters}\\
None.\\

\noindent
{\bf Description}\\
\vspace{-12pt}
\begin{quote}
{\bf CMobot::motionUnstand()}\\
This function causes the robot to move down from the camera platform.

{\bf CMobot::motionUnstandNB()}\\
This function causes the robot to move down from the camera platform.

This function has both a blocking and non-blocking version.
The blocking version, \texttt{motionUnstand()}, will block until the
robot motion has completed. The non-blocking version, \texttt{motionUnstandNB()},
will return immediately, and the motion will be performed asynchronously.\\
\end{quote}

\noindent
{\bf See Also}\\

%\CPlot::\DataThreeD(), \CPlot::\DataFile(), \CPlot::\Plotting(), \plotxy().\\
