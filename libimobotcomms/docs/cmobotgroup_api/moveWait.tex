\noindent
\vspace{5pt}
\rule{4.5in}{0.015in}\\
\noindent
{\LARGE \texttt{CMobotGroup::moveWait()}\index{CMobotGroup::moveWait()}}\\
%\phantomsection
\addcontentsline{toc}{section}{moveWait()}

\noindent
{\bf Synopsis}
\vspace{-8pt}
\begin{verbatim}
#include <mobot.h>
int CMobotGroup::moveWait();
\end{verbatim}

\noindent
{\bf Purpose}\\
Wait for all joints of all mobots in the group to stop moving.\\

\noindent
{\bf Return Value}\\
The function returns 0 on success and non-zero otherwise.\\

\noindent
{\bf Description}\\
This function is used to wait for all joint motions to finish. Functions such as
\texttt{moveJointToNB()} and \texttt{moveJointNB()} do not wait for a joint to finish
moving before continuing to allow multiple joints to move at the same time. The
\texttt{moveWait()} or \texttt{moveJointWait()} functions are used to wait for
mobotic motions to complete.

Please note that if this function is called after a motor has been commanded to
turn indefinitely, this function may never return and your program may hang.\\

\noindent
{\bf Example}\\
See the sample program in Section \ref{sec:democode} on page \pageref{sec:democode}.
\noindent

\noindent
{\bf See Also}\\
\texttt{moveWait(), moveJointWait()}

%\CPlot::\DataThreeD(), \CPlot::\DataFile(), \CPlot::\Plotting(), \plotxy().\\
