\noindent
\vspace{5pt}
\rule{4.5in}{0.015in}\\
\noindent
{\LARGE \texttt{CMobotGroup::motionArch()}\index{CMobotGroup::motionArch()}}\\
{\LARGE \texttt{CMobotGroup::motionArchNB()}\index{CMobotGroup::motionArchNB()}}\\
%\phantomsection
\addcontentsline{toc}{section}{motionArch()}
\addcontentsline{toc}{section}{motionArchNB()}

\noindent
{\bf Synopsis}
\vspace{-8pt}
\begin{verbatim}
#include <mobot.h>
int CMobotGroup::motionArch(double angle);
int CMobotGroup::motionArchNB(double angle);
\end{verbatim}

\noindent
{\bf Purpose}\\
Arch the mobots in the group for more ground clearance.\\

\noindent
{\bf Return Value}\\
The function returns 0 on success and non-zero otherwise.\\

\noindent
{\bf Parameters}\\
\vspace{-0.1in}
\begin{description}
\item               
\begin{tabular}{p{10 mm}p{145 mm}}
\texttt{angle} & The angle which to arch. This number can range from 0 degrees, which is
no arch, to 180 degrees, which is a fully curled up position.\\
\end{tabular}
\end{description}

\noindent
{\bf Description}\\
\vspace{-12pt}
\begin{quote}
{\bf CMobot::motionArch()}\\
This function causes the mobots to Arch up for better ground clearance while 
rolling.

{\bf CMobot::motionArch()}\\
This function causes the mobots to Arch up for better ground clearance while 
rolling.

This function has both a blocking and non-blocking version.
The blocking version, \texttt{motionArch()}, will block until the
mobot motion has completed. The non-blocking version, \texttt{motionArchNB()},
will return immediately, and the motion will be performed asynchronously.\\
\end{quote}

\noindent
{\bf See Also}\\

%\CPlot::\DataThreeD(), \CPlot::\DataFile(), \CPlot::\Plotting(), \plotxy().\\
