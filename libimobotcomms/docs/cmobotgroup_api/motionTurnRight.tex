\noindent
\vspace{5pt}
\rule{4.5in}{0.015in}\\
\noindent
{\LARGE \texttt{CMobotGroup::motionTurnRight()}\index{CMobotGroup::motionTurnRight()}}\\
{\LARGE \texttt{CMobotGroup::motionTurnRightNB()}\index{CMobotGroup::motionTurnRightNB()}}\\
%\phantomsection
\addcontentsline{toc}{section}{motionTurnRight()}
\addcontentsline{toc}{section}{motionTurnRightNB()}

\noindent
{\bf Synopsis}
\vspace{-8pt}
\begin{verbatim}
#include <mobot.h>
int CMobotGroup::motionTurnRight(double angle);
int CMobotGroup::motionTurnRightNB(double angle);
\end{verbatim}

\noindent
{\bf Purpose}\\
Rotate the mobots using the faceplates as wheels.\\

\noindent
{\bf Return Value}\\
The function returns 0 on success and non-zero otherwise.\\

\noindent
{\bf Parameters}\\
\vspace{-0.1in}
\begin{description}
\item               
\begin{tabular}{p{10 mm}p{145 mm}}
\texttt{angle} & The angle in degrees to turn the wheels. The wheels will turn in opposite directions by the amount specifid by this argument in order to turn the mobot to the right. \\
\end{tabular}
\end{description}

\noindent
{\bf Description}\\
This function causes the mobots to rotate the faceplates in opposite directions
to cause the mobot to rotate clockwise.

This function has both a blocking and non-blocking version.
The blocking version, \texttt{motionTurnRight()}, will block until the
mobot motion has completed. The non-blocking version, \texttt{motionTurnRightNB()},
will return immediately, and the motion will be performed asynchronously.\\

\noindent
{\bf See Also}\\
\texttt{motionTurnLeft()}

%\CPlot::\DataThreeD(), \CPlot::\DataFile(), \CPlot::\Plotting(), \plotxy().\\
