\noindent
\vspace{5pt}
\rule{4.5in}{0.015in}\\
\noindent
{\LARGE \texttt{CMobotGroup::setTwoWheelRobotSpeed()}\index{CMobotGroup::setTwoWheelRobotSpeed()}}\\
%\phantomsection
\addcontentsline{toc}{section}{setTwoWheelRobotSpeed()}

\noindent
{\bf Synopsis}
\vspace{-8pt}
\begin{verbatim}
#include <mobot.h>
int CMobotGroup::setTwoWheelRobotSpeed(double speed, double radius);
\end{verbatim}

\noindent
{\bf Purpose}\\
Roll the mobots in the group at a certain speed in a straight line.\\

\noindent
{\bf Return Value}\\
The function returns 0 on success and non-zero otherwise.\\

\noindent
{\bf Parameters}
\vspace{-0.1in}
\begin{description}
\item               
\begin{tabular}{p{10 mm}p{145 mm}}
\texttt{speed} & The speed at which to roll the mobot. The units used will be the units
specified in the \texttt{unit} parameter. \\
\texttt{radius} & The radius of the wheels attached to the mobot. The units of the parameter
should match the units provided in the \texttt{unit} parameter. \\

\begin{tabular}{ll}
speed & radius \\
\hline \\
cm/s & cm \\
m/s & m \\
inch/s & inch \\
foot/s & foot \\
\hline
\end{tabular}
\end{tabular}
\end{description}

\noindent
{\bf Description}\\
This function is used to make a two wheeled mobot roll at a certain speed. The desired 
speed and radius of the wheels is provided and the function will rotate the wheels at the
appropriate rate in order to achieve the desired speed.
\noindent\\
{\bf Example}\\
\noindent

\noindent
{\bf See Also}\\

%\CPlot::\DataThreeD(), \CPlot::\DataFile(), \CPlot::\Plotting(), \plotxy().\\
