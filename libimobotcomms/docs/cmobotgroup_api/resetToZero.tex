\noindent
\vspace{5pt}
\rule{4.5in}{0.015in}\\
\noindent
{\LARGE \texttt{CMobotGroup::resetToZero()}\index{CMobotGroup::resetToZero()}}\\
{\LARGE \texttt{CMobotGroup::resetToZeroNB()}\index{CMobotGroup::resetToZeroNB()}}\\
%\phantomsection
\addcontentsline{toc}{section}{resetToZero()}
\addcontentsline{toc}{section}{resetToZeroNB()}

\noindent
{\bf Synopsis}
\vspace{-8pt}
\begin{verbatim}
#include <mobot.h>
int CMobotGroup::resetToZero();
int CMobotGroup::resetToZeroNB();
\end{verbatim}

\noindent
{\bf Purpose}\\
Move all of the joints of mobots in the group to their zero position.\\

\noindent
{\bf Return Value}\\
The function returns 0 on success and non-zero otherwise.\\

\noindent
{\bf Parameters}\\
None.\\

\noindent
{\bf Description}\\
\vspace{-12pt}
\begin{quote}
{\bf CMobot::resetToZero()}\\
This function moves all of the joints of mobots in the group to their zero position.
Please note that this function is non-blocking and will return immediately. Use
this function in conjunction with the \texttt{moveWait()} function to block
until the movement completes.

This function differs from the \texttt{moveToZero()} function by limiting the total
rotation back to the zero position to 180 degrees max, whereas the \texttt{moveToZero()}
function will completely ``rewind'' back to the zero position, possible turning
the faceplate joints more than 180 degrees.

{\bf CMobot::resetToZeroNB()}\\
This function moves all of the joints of mobots in the group to their zero position.
Please note that this function is non-blocking and will return immediately. Use
this function in conjunction with the \texttt{moveWait()} function to block
until the movement completes.

The function \texttt{resetToZero()} is a blocking function, which means that 
the function will not return until the commanded motion is 
completed. The function \texttt{resetToZeroNB()} is the non-blocking version of
the \texttt{resetToZero()} function, which means that the function will return
immediately and the physical mobot motion will occur asynchronously. For
more details on blocking and non-blocking functions, please refer to 
Section \ref{sec:blocking} on page \pageref{sec:blocking}.\\
\end{quote}

\noindent
{\bf Example}\\
Please see the demo at Section \ref{sec:democode} on page \pageref{sec:democode}.\\
\noindent

\noindent
{\bf See Also}\\

%\CPlot::\DataThreeD(), \CPlot::\DataFile(), \CPlot::\Plotting(), \plotxy().\\
