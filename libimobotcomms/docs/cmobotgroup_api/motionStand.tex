\noindent
\vspace{5pt}
\rule{4.5in}{0.015in}\\
\noindent
{\LARGE \texttt{CMobotGroup::motionStand()}\index{CMobotGroup::motionStand()}}\\
{\LARGE \texttt{CMobotGroup::motionStandNB()}\index{CMobotGroup::motionStandNB()}}\\
%\phantomsection
\addcontentsline{toc}{section}{motionStand()}
\addcontentsline{toc}{section}{motionStandNB()}

\noindent
{\bf Synopsis}
\vspace{-8pt}
\begin{verbatim}
#include <mobot.h>
int CMobotGroup::motionStand();
int CMobotGroup::motionStandNB();
\end{verbatim}

\noindent
{\bf Purpose}\\
Stand mobots up on a faceplate.\\

\noindent
{\bf Return Value}\\
The function returns 0 on success and non-zero otherwise.\\

\noindent
{\bf Parameters}\\
None.\\

\noindent
{\bf Description}\\
\vspace{-12pt}
\begin{quote}
{\bf CMobot::motionStand()}\\
This function causes the mobots to stand up into the camera platform.

{\bf CMobot::motionStandNB()}\\
This function causes the mobots to stand up into the camera platform.

This function has both a blocking and non-blocking version.
The blocking version, \texttt{motionStand()}, will block until the
mobot motion has completed. The non-blocking version, \texttt{motionStandNB()},
will return immediately, and the motion will be performed asynchronously.\\
\end{quote}

\noindent
{\bf See Also}\\

%\CPlot::\DataThreeD(), \CPlot::\DataFile(), \CPlot::\Plotting(), \plotxy().\\
