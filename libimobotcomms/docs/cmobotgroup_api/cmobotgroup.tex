%\lhead{libimobotcomms API Documentation}
The \texttt{CMobotGroup} class is used to control multiple modules simultaneously. The
member functions of the \texttt{CMobotGroup} class closely mimic those of the \texttt{CMobot}
group. The main difference is that the member functions of the \texttt{CMobot} class
affect a single robot, whereas the member functions of the \texttt{CMobotGroup} class
move and affect a group of many robots. 

\begin{table}[!h]
%\capstart
\begin{center}
\caption{CMobotGroup Member Functions.}
\begin{tabular}{p{38 mm}p{107 mm}}
%\begin{tabular}{ll}
\hline
Function & Description \\
\hline
%\texttt{pose()} \dotfill & Pose multiple joints of the robot. \\
\texttt{CMobotGroup()} & The CMobotGroup constructor function. This function
is called automatically and should not be called explicitly. \\
\texttt{\textasciitilde CMobotGroup()} & The CMobotGroup destructor function. This function
is called automatically and should not be called explicitly. \\
& \\
\texttt{addRobot()} & Add a robot to be a member of the robot group. \\
\texttt{move()} & Move all four joints of the robots by specified angles. \\
\texttt{moveNB()} & Identical to \texttt{move()} but non-blocking. \\
\texttt{moveContinuousNB()} & Move joints continuously. Joints will move untill stopped.\\
\texttt{moveContinuousTime()} & Move joints continuously for a certain amount of time.\\
\texttt{moveJointContinuousNB()} & Move a single joint on all robots continuously. \\
\texttt{moveJointContinuousTime()} & Move a single joint on all robots continuously for a specific amount of time. \\
\texttt{moveTo()} & Move all four joints of the robots to specified absolute angles. \\
\texttt{moveToNB()} & Identical to \texttt{moveTo()} but non-blocking. \\
\texttt{moveJoint()} & Move a motor from its current position by an angle. \\
\texttt{moveJointNB()} & Identical to \texttt{moveJoint()} but non-blocking. \\
\texttt{moveJointTo()} & Set the desired motor position for a joint. \\
\texttt{moveJointToNB()} & Identical to \texttt{moveJointTo()} but non-blocking. \\
\texttt{moveJointWait()} & Wait until the specified motor has stopped moving. \\
\texttt{moveWait()} & Wait until all motors have stopped moving. \\
\texttt{moveToZero()} & Instructs all motors to go to their zero positions. \\
\texttt{moveToZeroNB()} & Identical to \texttt{moveToZero()} but non-blocking. \\
%\texttt{setJointDirection()} & Set the motor direction of a motor. Set
%to "0" for automatic direction, "1" for forward, and "2" for reverse. \\
\texttt{setJointSpeed()} & Set a motor's speed setting in radians per second. \\
\texttt{setJointSpeeds()} & Set all motor speeds in radians per second. \\
\texttt{setJointSpeedRatio()} & Set a joints speed setting to a fraction of its maximum speed, a value between 0 and 1. \\
\texttt{setJointSpeedRatios()} & Set all joint speed settings to a fraction of its
maximum speed, expressed as a value from 0 to 1. \\
\texttt{setTwoWheelRobotSpeed()} & Move the robot at a constant forward velocity. \\
\texttt{stop()} & Stop all currently executing motions of the robot. \\
\hline
\end{tabular}
\end{center}
\label{mobilec_api_cbinary}
\end{table}

\begin{table}[!h]
%\capstart
\begin{center}
\caption{CMobot Member Functions for Compound Motions.}
\begin{tabular}{p{38 mm}p{107 mm}}
Compound Motions & These are convenience functions of commonly used compound motions. \\
\hline
\texttt{motionInchwormLeft()} \dotfill & Inchworm motion towards the left. \\
\texttt{motionInchwormLeftNB()} \dotfill & Identical to \texttt{motionInchwormLeft} but non-blocking. \\
\texttt{motionInchwormRight()} \dotfill & Inchworm motion towards the right. \\
\texttt{motionInchwormRightNB()} \dotfill & Identical to \texttt{motionInchwormRight} but non-blocking. \\
\texttt{motionRollBackward()} \dotfill & Roll on the faceplates toward the backward direction. \\
\texttt{motionRollBackwardNB()} \dotfill & Identical to \texttt{motionRollBackward()} but non-blocking. \\
\texttt{motionRollForward()} \dotfill & Roll on the faceplates forwards. \\
\texttt{motionRollForwardNB()} \dotfill & Identical to \texttt{motionRollForward()} but non-blocking. \\
\texttt{motionStand()} \dotfill & Stand the robot up on its end. \\
\texttt{motionStandNB()} \dotfill & Identical to \texttt{motionStandNB()} but non-blocking. \\
\texttt{motionTurnLeft()} \dotfill & Rotate the robot counterclockwise. \\
\texttt{motionTurnLeftNB()} \dotfill & Identical to \texttt{motionTurnLeft()} but non-blocking. \\
\texttt{motionTurnRight()} \dotfill & Rotate the robot clockwise. \\
\texttt{motionTurnRightNB()} \dotfill & Identical to \texttt{motionTurnRight()} but non-blocking. \\
\hline
\end{tabular}
\end{center}
\label{mobilec_api_compound}
\end{table}

\clearpage
\newpage
\noindent
\vspace{5pt}
\rule{4.5in}{0.015in}\\
\noindent
{\LARGE \texttt{CMobotGroup::addMobot()}\index{CMobotGroup::addMobot()}}\\
%\phantomsection
\addcontentsline{toc}{section}{addMobot()}

\noindent
{\bf Synopsis}
\vspace{-8pt}
\begin{verbatim}
#include <mobot.h>
int CMobotGroup::addMobot(CMobot &mobot);
\end{verbatim}

\noindent
{\bf Purpose}\\
Add a mobot to a mobot group.\\

\noindent
{\bf Return Value}\\
The function returns 0 on success and non-zero otherwise.\\

\noindent
{\bf Parameters}\\
A mobot handle attached to the mobot to add to the group.\\

\noindent
{\bf Description}\\
This function is used to add a mobot to a mobot group. \\
\noindent
{\bf Example}\\
\noindent

\noindent
{\bf See Also}\\

%\CPlot::\DataThreeD(), \CPlot::\DataFile(), \CPlot::\Plotting(), \plotxy().\\

\noindent
\vspace{5pt}
\rule{4.5in}{0.015in}\\
\noindent
{\LARGE \texttt{CMobot::motionArch()}\index{CMobot::motionArch()}}\\
{\LARGE \texttt{CMobot::motionArchNB()}\index{CMobot::motionArchNB()}}\\
%\phantomsection
\addcontentsline{toc}{section}{motionArch()}
\addcontentsline{toc}{section}{motionArchNB()}

\noindent
{\bf Synopsis}
\vspace{-8pt}
\begin{verbatim}
#include <mobot.h>
int CMobot::motionArch(double angle);
int CMobot::motionArchNB(double angle);
\end{verbatim}

\noindent
{\bf Purpose}\\
Arch the mobot for more ground clearance.\\

\noindent
{\bf Return Value}\\
The function returns 0 on success and non-zero otherwise.\\

\noindent
{\bf Parameters}\\
\vspace{-0.1in}
\begin{description}
\item               
\begin{tabular}{p{10 mm}p{145 mm}}
\texttt{angle} & The angle in degrees to arch. This number can range from 0 degrees, which is
no arch, to 180 degrees, which is a fully curled up position.\\
\end{tabular}
\end{description}

\noindent
{\bf Description}\\
\vspace{-12pt}
\begin{quote}
{\bf CMobot::motionArch()}\\
This function causes the mobot to Arch up for better ground clearance while 
rolling.

{\bf CMobot::motionArchNB()}\\
This function causes the mobot to Arch up for better ground clearance while 
rolling.

The non-blocking function, \texttt{motionArchNB()},
will return immediately, and the motion will be performed asynchronously.\\
\end{quote}

\noindent
{\bf See Also}\\

%\CPlot::\DataThreeD(), \CPlot::\DataFile(), \CPlot::\Plotting(), \plotxy().\\

\noindent
\vspace{5pt}
\rule{4.5in}{0.015in}\\
\noindent
{\LARGE \texttt{CMobotGroup::motionInchwormLeft()}\index{CMobotGroup::motionInchwormLeft()}}\\
{\LARGE \texttt{CMobotGroup::motionInchwormLeftNB()}\index{CMobotGroup::motionInchwormLeftNB()}}\\
%\phantomsection
\addcontentsline{toc}{section}{motionInchwormLeft()}
\addcontentsline{toc}{section}{motionInchwormLeftNB()}

\noindent
{\bf Synopsis}
\vspace{-8pt}
\begin{verbatim}
#include <mobot.h>
int CMobotGroup::motionInchwormLeft(int num);
int CMobotGroup::motionInchwormLeftNB(int num);
\end{verbatim}

\noindent
{\bf Purpose}\\
Make all mobots in the group perform the inch-worm gait to the left.\\

\noindent
{\bf Return Value}\\
The function returns 0 on success and non-zero otherwise.\\

\noindent
{\bf Parameters}\\
\vspace{-0.1in}
\begin{description}
\item               
\begin{tabular}{p{15 mm}p{145 mm}}
\texttt{num} & The number of times to perform the inchworm gait.\\
\end{tabular}
\end{description}

\noindent
{\bf Description}\\
\vspace{-12pt}
\begin{quote}
{\bf CMobot::motionInchwormLeft()}\\
This function causes the mobots to perform a single cycle of the inchworm gait
to the left. 

{\bf CMobot::motionInchwormLeftNB()}\\
This function causes the mobots to perform a single cycle of the inchworm gait
to the left. 

The function \texttt{motionInchwormLeft()} is blocking, and the function
will hang until the motion has finished. The alternative function, \texttt{motionInchwormLeftNB()} 
will return immediately, and the motion will execute asynchronously. \\
\end{quote}

\noindent
{\bf See Also}\\
\texttt{motionInchwormRight()}

%\CPlot::\DataThreeD(), \CPlot::\DataFile(), \CPlot::\Plotting(), \plotxy().\\

\noindent
\vspace{5pt}
\rule{4.5in}{0.015in}\\
\noindent
{\LARGE \texttt{CMobot::motionInchwormRight()}\index{CMobot::motionInchwormRight()}}\\
{\LARGE \texttt{CMobot::motionInchwormRightNB()}\index{CMobot::motionInchwormRightNB()}}\\
%\phantomsection
\addcontentsline{toc}{section}{motionInchwormRight()}
\addcontentsline{toc}{section}{motionInchwormRightNB()}

\noindent
{\bf Synopsis}
\vspace{-8pt}
\begin{verbatim}
#include <mobot.h>
int CMobot::motionInchwormRight(int num);
int CMobot::motionInchwormRightNB(int num);
\end{verbatim}

\noindent
{\bf Purpose}\\
Perform the inch-worm gait to the right.\\

\noindent
{\bf Return Value}\\
The function returns 0 on success and non-zero otherwise.\\

\noindent
{\bf Parameters}\\
\vspace{-0.1in}
\begin{description}
\item               
\begin{tabular}{p{15 mm}p{145 mm}}
\texttt{num} & The number of times to perform the inchworm gait.\\
\end{tabular}
\end{description}

\noindent
{\bf Description}\\
\vspace{-12pt}
\begin{quote}
{\bf CMobot::motionInchwormRight()}\\
This function causes the robot to perform a single cycle of the inchworm gait
to the right. 

{\bf CMobot::motionInchwormRightNB()}\\
This function causes the robot to perform a single cycle of the inchworm gait
to the right. 

This function has both a blocking and non-blocking version.
The blocking version, \texttt{motionInchwormRight()}, will block until the
robot motion has completed. The non-blocking version, \texttt{motionInchwormRightNB()},
will return immediately, and the motion will be performed asynchronously.\\
\end{quote}

\noindent
{\bf See Also}\\
\texttt{motionInchwormLeft()}

%\CPlot::\DataThreeD(), \CPlot::\DataFile(), \CPlot::\Plotting(), \plotxy().\\

\noindent
\vspace{5pt}
\rule{4.5in}{0.015in}\\
\noindent
{\LARGE \texttt{CMobotGroup::motionRollBackward()}\index{CMobotGroup::motionRollBackward()}}\\
{\LARGE \texttt{CMobotGroup::motionRollBackwardNB()}\index{CMobotGroup::motionRollBackwardNB()}}\\
%\phantomsection
\addcontentsline{toc}{section}{motionRollBackward()}
\addcontentsline{toc}{section}{motionRollBackwardNB()}

\noindent
{\bf Synopsis}
\vspace{-8pt}
\begin{verbatim}
#include <mobot.h>
int CMobotGroup::motionRollBackward(double angle);
int CMobotGroup::motionRollBackwardNB(double angle);
\end{verbatim}

\noindent
{\bf Purpose}\\
Use the faceplates as wheels to roll all the robots in a group backward.\\

\noindent
{\bf Return Value}\\
The function returns 0 on success and non-zero otherwise.\\

\noindent
{\bf Parameters}\\
\vspace{-0.1in}
\begin{description}
\item               
\begin{tabular}{p{15 mm}p{145 mm}}
\texttt{angle} & The angle to turn the wheels, specified in degrees.\\
\end{tabular}
\end{description}

\noindent
{\bf Description}\\
\vspace{-12pt}
\begin{quote}
{\bf CMobot::motionRollBackward()}\\
This function causes each of the faceplates to rotate 90 degrees to roll the
robots backward.

{\bf CMobot::motionRollBackwardNB()}\\
This function causes each of the faceplates to rotate 90 degrees to roll the
robots backward.

This function has both a blocking and non-blocking version.
The blocking version, \texttt{motionRollBackward()}, will block until the
robot motion has completed. The non-blocking version, \texttt{motionRollBackwardNB()},
will return immediately, and the motion will be performed asynchronously.\\
\end{quote}

\noindent
{\bf See Also}\\
\texttt{motionRollForward()}

%\CPlot::\DataThreeD(), \CPlot::\DataFile(), \CPlot::\Plotting(), \plotxy().\\

\noindent
\vspace{5pt}
\rule{4.5in}{0.015in}\\
\noindent
{\LARGE \texttt{CMobot::motionRollForward()}\index{CMobot::motionRollForward()}}\\
{\LARGE \texttt{CMobot::motionRollForwardNB()}\index{CMobot::motionRollForwardNB()}}\\
%\phantomsection
\addcontentsline{toc}{section}{motionRollForward()}
\addcontentsline{toc}{section}{motionRollForwardNB()}

\noindent
{\bf Synopsis}
\vspace{-8pt}
\begin{verbatim}
#include <mobot.h>
int CMobot::motionRollForward(double angle);
int CMobot::motionRollForwardNB(double angle);
\end{verbatim}

\noindent
{\bf Purpose}\\
Use the faceplates as wheels to roll forward.\\

\noindent
{\bf Return Value}\\
The function returns 0 on success and non-zero otherwise.\\

\noindent
{\bf Parameters}\\
\vspace{-0.1in}
\begin{description}
\item               
\begin{tabular}{p{15 mm}p{145 mm}}
\texttt{angle} & The angle to turn the wheels, specified in degrees.\\
\end{tabular}
\end{description}

\noindent
{\bf Description}\\
\vspace{-12pt}
\begin{quote}
{\bf CMobot::motionRollForward()}\\
This function causes each of the faceplates to rotate to roll the
mobot forward. The amount to roll the wheels is specified by the argument,
\texttt{angle}.

{\bf CMobot::motionRollForwardNB()}\\
This function causes each of the faceplates to rotate to roll the
mobot forward. The amount to roll the wheels is specified by the argument,
\texttt{angle}.

This function has both a blocking and non-blocking version.
The blocking version, \texttt{motionRollForward()}, will block until the
mobot motion has completed. The non-blocking version, \texttt{motionRollForwardNB()},
will return immediately, and the motion will be performed asynchronously.\\
\end{quote}

\noindent
{\bf See Also}\\
\texttt{motionRollBackward()}

%\CPlot::\DataThreeD(), \CPlot::\DataFile(), \CPlot::\Plotting(), \plotxy().\\

\noindent
\vspace{5pt}
\rule{4.5in}{0.015in}\\
\noindent
{\LARGE \texttt{CMobot::motionSkinny()}\index{CMobot::motionSkinny()}}\\
{\LARGE \texttt{CMobot::motionSkinnyNB()}\index{CMobot::motionSkinnyNB()}}\\
%\phantomsection
\addcontentsline{toc}{section}{motionSkinny()}
\addcontentsline{toc}{section}{motionSkinnyNB()}

\noindent
{\bf Synopsis}
\vspace{-8pt}
\begin{verbatim}
#include <mobot.h>
int CMobot::motionSkinny(double angle);
int CMobot::motionSkinnyNB(double angle);
\end{verbatim}

\noindent
{\bf Purpose}\\
Move the robot into a skinny profile.\\

\noindent
{\bf Return Value}\\
The function returns 0 on success and non-zero otherwise.\\

\noindent
{\bf Parameters}\\
None.\\

\noindent
{\bf Description}\\
This function causes the robot to motionSkinny up into the camera platform.

This function has both a blocking and non-blocking version.
The blocking version, \texttt{motionSkinny()}, will block until the
robot motion has completed. The non-blocking version, \texttt{motionSkinnyNB()},
will return immediately, and the motion will be performed asynchronously.\\

\noindent
{\bf See Also}\\

%\CPlot::\DataThreeD(), \CPlot::\DataFile(), \CPlot::\Plotting(), \plotxy().\\

\noindent
\vspace{5pt}
\rule{4.5in}{0.015in}\\
\noindent
{\LARGE \texttt{CMobot::motionStand()}\index{CMobot::motionStand()}}\\
{\LARGE \texttt{CMobot::motionStandNB()}\index{CMobot::motionStandNB()}}\\
%\phantomsection
\addcontentsline{toc}{section}{motionStand()}
\addcontentsline{toc}{section}{motionStandNB()}

\noindent
{\bf Synopsis}
\vspace{-8pt}
\begin{verbatim}
#include <mobot.h>
int CMobot::motionStand();
int CMobot::motionStandNB();
\end{verbatim}

\noindent
{\bf Purpose}\\
Stand the mobot up on a faceplate.\\

\noindent
{\bf Return Value}\\
The function returns 0 on success and non-zero otherwise.\\

\noindent
{\bf Parameters}\\
None.\\

\noindent
{\bf Description}\\
\vspace{-12pt}
\begin{quote}
{\bf CMobot::motionStand()}\\
This function causes the mobot to motionStand up into the camera platform.

{\bf CMobot::motionStandNB()}\\
This function causes the mobot to motionStand up into the camera platform.

This function has both a blocking and non-blocking version.
The blocking version, \texttt{motionStand()}, will block until the
mobot motion has completed. The non-blocking version, \texttt{motionStandNB()},
will return immediately, and the motion will be performed asynchronously.\\
\end{quote}

\noindent
{\bf See Also}\\

%\CPlot::\DataThreeD(), \CPlot::\DataFile(), \CPlot::\Plotting(), \plotxy().\\

\noindent
\vspace{5pt}
\rule{4.5in}{0.015in}\\
\noindent
{\LARGE \texttt{CMobot::motionStand()}\index{CMobot::motionStand()}}\\
{\LARGE \texttt{CMobot::motionStandNB()}\index{CMobot::motionStandNB()}}\\
%\phantomsection
\addcontentsline{toc}{section}{motionStand()}
\addcontentsline{toc}{section}{motionStandNB()}

\noindent
{\bf Synopsis}
\vspace{-8pt}
\begin{verbatim}
#include <mobot.h>
int CMobot::motionStand();
int CMobot::motionStandNB();
\end{verbatim}

\noindent
{\bf Purpose}\\
Stand the robot up on a faceplate.\\

\noindent
{\bf Return Value}\\
The function returns 0 on success and non-zero otherwise.\\

\noindent
{\bf Parameters}\\
None.\\

\noindent
{\bf Description}\\
This function causes the robot to motionStand up into the camera platform.

This function has both a blocking and non-blocking version.
The blocking version, \texttt{motionStand()}, will block until the
robot motion has completed. The non-blocking version, \texttt{motionStandNB()},
will return immediately, and the motion will be performed asynchronously.\\

\noindent
{\bf See Also}\\

%\CPlot::\DataThreeD(), \CPlot::\DataFile(), \CPlot::\Plotting(), \plotxy().\\

\noindent
\vspace{5pt}
\rule{4.5in}{0.015in}\\
\noindent
{\LARGE \texttt{CMobotGroup::motionTurnLeft()}\index{CMobotGroup::motionTurnLeft()}}\\
{\LARGE \texttt{CMobotGroup::motionTurnLeftNB()}\index{CMobotGroup::motionTurnLeftNB()}}\\
%\phantomsection
\addcontentsline{toc}{section}{motionTurnLeft()}
\addcontentsline{toc}{section}{motionTurnLeftNB()}

\noindent
{\bf Synopsis}
\begin{verbatim}
#include <mobot.h>
int CMobotGroup::motionTurnLeft();
int CMobotGroup::motionTurnLeftNB();
\end{verbatim}

\noindent
{\bf Purpose}\\
Rotate the robots using the faceplates as wheels.\\

\noindent
{\bf Return Value}\\
The function returns 0 on success and non-zero otherwise.\\

\noindent
{\bf Parameters}\\
None.\\

\noindent
{\bf Description}\\
This function causes the robots to rotate the faceplates in opposite directions
to cause the robot to rotate counter-clockwise.

This function has both a blocking and non-blocking version.
The blocking version, \texttt{motionTurnLeft()}, will block until the
robot motion has completed. The non-blocking version, \texttt{motionTurnLeftNB()},
will return immediately, and the motion will be performed asynchronously.\\


\noindent
{\bf See Also}\\
\texttt{motionTurnRight()}

%\CPlot::\DataThreeD(), \CPlot::\DataFile(), \CPlot::\Plotting(), \plotxy().\\

\noindent
\vspace{5pt}
\rule{4.5in}{0.015in}\\
\noindent
{\LARGE \texttt{CMobot::motionTurnRight()}\index{motionTurnRight()}}\\
%\phantomsection
\addcontentsline{toc}{section}{motionTurnRight()}

\noindent
{\bf Synopsis}\\
\begin{verbatim}
#include <mobot.h>
int CMobot::motionTurnRight();
\end{verbatim}

\noindent
{\bf Purpose}\\
Rotate the MoBot using the faceplates as wheels.\\

\noindent
{\bf Return Value}\\
The function returns 0 on success and non-zero otherwise.\\

\noindent
{\bf Parameters}\\
None.\\

\noindent
{\bf Description}\\
This function causes the MoBot to rotate the faceplates in opposite directions
to cause the robot to rotate clockwise.\\

\noindent
{\bf See Also}\\
\texttt{motionTurnLeft()}

%\CPlot::\DataThreeD(), \CPlot::\DataFile(), \CPlot::\Plotting(), \plotxy().\\

\noindent
\vspace{5pt}
\rule{4.5in}{0.015in}\\
\noindent
{\LARGE \texttt{CMobotGroup::motionUnstand()}\index{CMobotGroup::motionUnstand()}}\\
{\LARGE \texttt{CMobotGroup::motionUnstandNB()}\index{CMobotGroup::motionUnstandNB()}}\\
%\phantomsection
\addcontentsline{toc}{section}{motionUnstand()}
\addcontentsline{toc}{section}{motionUnstandNB()}

\noindent
{\bf Synopsis}
\vspace{-8pt}
\begin{verbatim}
#include <mobot.h>
int CMobotGroup::motionUnstand();
int CMobotGroup::motionUnstandNB();
\end{verbatim}

\noindent
{\bf Purpose}\\
Move robots currently standing on a faceplate back down into a prone position.\\

\noindent
{\bf Return Value}\\
The function returns 0 on success and non-zero otherwise.\\

\noindent
{\bf Parameters}\\
None.\\

\noindent
{\bf Description}\\
This function causes the robot to move down from the camera platform.

This function has both a blocking and non-blocking version.
The blocking version, \texttt{motionUnstand()}, will block until the
robot motion has completed. The non-blocking version, \texttt{motionUnstandNB()},
will return immediately, and the motion will be performed asynchronously.\\

\noindent
{\bf See Also}\\

%\CPlot::\DataThreeD(), \CPlot::\DataFile(), \CPlot::\Plotting(), \plotxy().\\

\noindent
\vspace{5pt}
\rule{4.5in}{0.015in}\\
\noindent
{\LARGE \texttt{CMobot::move()}\index{move()}}\\
{\LARGE \texttt{CMobot::moveNB()}\index{moveNB()}}\\
%\phantomsection
\addcontentsline{toc}{section}{move()}
\addcontentsline{toc}{section}{moveNB()}

\noindent
{\bf Synopsis}\\
\begin{verbatim}
#include <mobot.h>
int CMobot::move(double angle1, double angle2, double angle3, double angle4);
int CMobot::moveNB(double angle1, double angle2, double angle3, double angle4);
\end{verbatim}

\noindent
{\bf Purpose}\\
Move all of the joints of a robot by specified angles.\\

\noindent
{\bf Return Value}\\
The function returns 0 on success and non-zero otherwise.\\

\noindent
{\bf Parameters}\\
\vspace{-0.1in}
\begin{description}
\item               
\begin{tabular}{p{15 mm}p{105 mm}}
\texttt{angle1} & The amount to move joint 1, expressed in radians. \\
\texttt{angle2} & The amount to move joint 2, expressed in radians. \\
\texttt{angle3} & The amount to move joint 3, expressed in radians. \\
\texttt{angle4} & The amount to move joint 4, expressed in radians. \\
\end{tabular}
\end{description}
\noindent
{\bf Description}\\
This function moves all of the joints of a robot by the specified number of degrees
from their current positions. 

The function \texttt{move()} is a blocking function,
which means that the function will not return until the commanded motion is 
completed. The function \texttt{moveNB()} is the non-blocking version of
the \texttt{move()} function, which means that the function will return
immediately and the physical robot motion will occur asynchronously. For 
more information on blocking and non-blocking functions, please refer to 
Section \ref{sec:blocking} on page \pageref{sec:blocking}.\\

\noindent
{\bf Example}\\
Please see the demo at Section \ref{sec:democode} on page \pageref{sec:democode}.\\
\noindent

\noindent
{\bf See Also}\\

%\CPlot::\DataThreeD(), \CPlot::\DataFile(), \CPlot::\Plotting(), \plotxy().\\

\noindent
\vspace{5pt}
\rule{4.5in}{0.015in}\\
\noindent
{\LARGE \texttt{CMobotGroup::moveContinuousNB()}\index{CMobotGroup::moveContinuousNB()}}\\
%\phantomsection
\addcontentsline{toc}{section}{moveContinuousNB()}

\noindent
{\bf Synopsis}
\vspace{-8pt}
\begin{verbatim}
#include <mobot.h>
int CMobotGroup::moveContinuousNB(
  robotJointState_t dir1, 
  robotJointState_t dir2, 
  robotJointState_t dir3, 
  robotJointState_t dir4);
\end{verbatim}

\noindent
{\bf Purpose}\\
Move the joints of grouped robots continuously in the specified directions.\\

\noindent
{\bf Return Value}\\
The function returns 0 on success and non-zero otherwise.\\

\noindent
{\bf Parameters}\\
Each integer parameter specifies the direction the joint should move. The types
are enumerated in \texttt{mobot.h} and have the following values:
\begin{itemize}
\item \texttt{ROBOT\_NEUTRAL} : The joint should not move.
\item \texttt{ROBOT\_FORWARD} : The joint will begin moving in the positive direction.
\item \texttt{ROBOT\_BACKWARD}: The joint will begin moving in the negative direction.
\item \texttt{ROBOT\_HOLD}: The joint will hold its current position.
\end{itemize}
More documentation about these types may be found at Section
\ref{sec:robotJointState_t} on page
\pageref{sec:robotJointState_t}.

\noindent
{\bf Description}\\
This function causes joints of robots to begin moving at the previously set
speed. The joints will continue moving until the joint hits a joint limit, or
the joint is stopped by setting the speed to zero. This function is a non-blocking
function.\\

\noindent
{\bf Example}\\
\noindent

\noindent
{\bf See Also}\\

%\CPlot::\DataThreeD(), \CPlot::\DataFile(), \CPlot::\Plotting(), \plotxy().\\

\noindent
\vspace{5pt}
\rule{4.5in}{0.015in}\\
\noindent
{\LARGE \texttt{CMobot::moveContinuousTime()}\index{CMobot::moveContinuousTime()}}\\
%\phantomsection
\addcontentsline{toc}{section}{moveContinuousTime()}

\noindent
{\bf Synopsis}
\vspace{-8pt}
\begin{verbatim}
#include <mobot.h>
int CMobot::moveContinuousTime( robotJointState_t dir1, 
                                robotJointState_t dir2, 
                                robotJointState_t dir3, 
                                robotJointState_t dir4, 
                                double seconds);
\end{verbatim}

\noindent
{\bf Purpose}\\
Move the joints of a robot continuously in the specified directions for some amount of time.\\

\noindent
{\bf Return Value}\\
The function returns 0 on success and non-zero otherwise.\\

\noindent
{\bf Parameters}\\
Each direction parameter specifies the direction the joint should move. The types
are enumerated in \texttt{mobot.h} and have the following values:
\\
\noindent
\begin{tabular}{p{1.75in}p{4.5in}} \hline 
Value & Description \\
\hline \\
\texttt{MOBOT\_NEUTRAL}& This value indicates that the joint is not moving and is not actuated. The joint is freely backdrivable. \\
\texttt{MOBOT\_FORWARD}& This value indicates that the joint is currently moving forward. \\
\texttt{MOBOT\_BACKWARD}& This value indicates that the joint is currently moving backward. \\
\texttt{MOBOT\_HOLD}& This value indicates that the joint is currently not moving and is holding its current position. The joint is not currently backdrivable. \\
\hline
\end{tabular}\\


The \texttt{seconds} parameter is the time to perform the movement, in seconds.
\\

\noindent
{\bf Description}\\
This function causes joints of a robot to begin moving. The joints will continue moving
until the joint hits a joint limit, or the time specified in the \texttt{seconds} parameter
is reached. This function will block until the motion is completed.\\

\noindent
{\bf Example}\\
\noindent

\noindent
{\bf See Also}\\

%\CPlot::\DataThreeD(), \CPlot::\DataFile(), \CPlot::\Plotting(), \plotxy().\\

\noindent
\vspace{5pt}
\rule{4.5in}{0.015in}\\
\noindent
{\LARGE \texttt{CiMobotComms::moveTo()}\index{moveTo()}}\\
%\phantomsection
\addcontentsline{toc}{section}{moveTo()}

\noindent
{\bf Synopsis}\\
\begin{verbatim}
#include <imobot.h>
int CiMobotComms::moveTo(double angle1, double angle2, double angle3, double angle4);
\end{verbatim}

\noindent
{\bf Purpose}\\
Move all of the joints of an iMobot to the specified positions.\\

\noindent
{\bf Return Value}\\
The function returns 0 on success and non-zero otherwise.\\

\noindent
{\bf Parameters}\\
None.\\

\noindent
{\bf Description}\\
This function moves all of the joints of an iMobot to the specified absolute positions. \\

\noindent
{\bf Example}\\
Please see the demo at Section \ref{sec:democode} on page \pageref{sec:democode}.\\
\noindent

\noindent
{\bf See Also}\\

%\CPlot::\DataThreeD(), \CPlot::\DataFile(), \CPlot::\Plotting(), \plotxy().\\

\noindent
\vspace{5pt}
\rule{4.5in}{0.015in}\\
\noindent
{\LARGE \texttt{CMobot::moveJoint()}\index{CMobot::moveJoint()}}\\
{\LARGE \texttt{CMobot::moveJointNB()}\index{CMobot::moveJointNB()}}\\
%\phantomsection
\addcontentsline{toc}{section}{moveJoint()}
\addcontentsline{toc}{section}{moveJointNB()}

\noindent
{\bf Synopsis}
\vspace{-8pt}
\begin{verbatim}
#include <mobot.h>
int CMobot::moveJoint(robotJointId_t id, double angle);
int CMobot::moveJointNB(robotJointId_t id, double angle);
\end{verbatim}

\noindent
{\bf Purpose}\\
Move a joint on the robot by a specified angle with respect to the current position.\\

\noindent
{\bf Return Value}\\
The function returns 0 on success and non-zero otherwise.\\

\noindent
{\bf Parameters}\\
\vspace{-0.1in}
\begin{description}
\item               
\begin{tabular}{p{10 mm}p{145 mm}}
\texttt{id} & The joint number to move. \\
\texttt{angle} & The angle in degrees to move the motor relative to its current position.  \\
\end{tabular}
\end{description}

\noindent
{\bf Description}\\
\vspace{-12pt}
\begin{quote}
{\bf CMobot::moveJoint()}\\
This function commands the motor to move by an angle relative to the
joint's current position at the joints current speed setting.
The current motor speed may be set with the
\texttt{setJointSpeed()} member function. Please note that if the motor speed
is set to zero, the motor will not move after calling the
\texttt{moveJoint()} function. 

{\bf CMobot::moveJointNB()}\\
This function commands the motor to move by an angle relative to the
joint's current position at the joints current speed setting.
The current motor speed may be set with the
\texttt{setJointSpeed()} member function. Please note that if the motor speed
is set to zero, the motor will not move after calling the
\texttt{moveJoint()} function. 

 The function \texttt{moveJointNB()} is the non-blocking version of
the \texttt{moveJoint()} function, which means that the function will return
immediately and the physical robot motion will occur asynchronously. For
more details on blocking and non-blocking functions, please refer to 
Section \ref{sec:blocking} on page \pageref{sec:blocking}.\\
\end{quote}

\noindent
{\bf Example}\\
Please see the example in Section \ref{sec:democode} on page \pageref{sec:democode}.\\
\noindent

\noindent
{\bf See Also}\\
\texttt{connectWithAddress()}

%\CPlot::\DataThreeD(), \CPlot::\DataFile(), \CPlot::\Plotting(), \plotxy().\\

\noindent
\vspace{5pt}
\rule{4.5in}{0.015in}\\
\noindent
{\LARGE \texttt{CMobot::moveJointTo()}\index{moveJointTo()}}\\
%\phantomsection
\addcontentsline{toc}{section}{moveJointTo()}

\noindent
{\bf Synopsis}\\
\begin{verbatim}
#include <mobot.h>
int CMobot::moveJointTo(int id, double position);
\end{verbatim}

\noindent
{\bf Purpose}\\
Connect to a remote MoBot via Bluetooth.\\

\noindent
{\bf Return Value}\\
The function returns 0 on success and non-zero otherwise.\\

\noindent
{\bf Parameters}\\
\vspace{-0.1in}
\begin{description}
\item               
\begin{tabular}{p{10 mm}p{145 mm}}
\texttt{id} & The joint number to wait for. \\
\texttt{position} & The absolute angle to move the motor to.  \\
\end{tabular}
\end{description}

\noindent
{\bf Description}\\
This function commands the motor to move to a position specified in degrees at
the current motor's speed. The current motor speed may be set with the
\texttt{setJointSpeed()} member function. Please note that if the motor speed
is set to zero, the motor will not move after calling the
\texttt{moveJointTo()} function. \\

\noindent
{\bf Example}\\
Please see the example in Section \ref{sec:democode} on page \pageref{sec:democode}.\\
\noindent

\noindent
{\bf See Also}\\
\texttt{connectWithAddress()}

%\CPlot::\DataThreeD(), \CPlot::\DataFile(), \CPlot::\Plotting(), \plotxy().\\

\noindent
\vspace{5pt}
\rule{4.5in}{0.015in}\\
\noindent
{\LARGE \texttt{CMobotGroup::moveJointWait()}\index{CMobotGroup::moveJointWait()}}\\
%\phantomsection
\addcontentsline{toc}{section}{moveJointWait()}

\noindent
{\bf Synopsis}
\vspace{-8pt}
\begin{verbatim}
#include <mobot.h>
int CMobotGroup::moveJointWait(robotJointId_t id);
\end{verbatim}

\noindent
{\bf Purpose}\\
Wait for a joint to stop moving on all robots in a group.\\

\noindent
{\bf Return Value}\\
The function returns 0 on success and non-zero otherwise.\\

\noindent
{\bf Parameters}
\vspace{-0.1in}
\begin{description}
\item               
\begin{tabular}{p{10 mm}p{145 mm}}
\texttt{id} & The joint number to wait for. \\
\end{tabular}
\end{description}

\noindent
{\bf Description}\\
This function is used to wait for a joint motion to finish. Functions such as
\texttt{moveJointToNB()} and \texttt{moveJointNB()} do not wait for a joint to finish
moving before continuing to allow multiple joints to move at the same time. The
\texttt{moveWait()} or \texttt{moveJointWait()} functions are used to wait for
robotic motions to complete.

Please note that if this function is called after a motor has been commanded to
turn indefinitely, this function may never return and your program may hang.\\

\noindent
{\bf Example}\\
Please see the example in Section \ref{sec:democode} on page \pageref{sec:democode}.\\
\noindent

\noindent
{\bf See Also}\\
\texttt{moveWait()}

%\CPlot::\DataThreeD(), \CPlot::\DataFile(), \CPlot::\Plotting(), \plotxy().\\

\noindent
\vspace{5pt}
\rule{6.5in}{0.015in}
\noindent
{\LARGE \texttt{CiMobot::moveWait()}\index{moveWait()}}\\
\phantomsection
\addcontentsline{toc}{section}{moveWait()}

\noindent
{\bf Synopsis}\\
\begin{verbatim}
#include <imobot.h>
int CiMobot::moveWait();
\end{verbatim}

\noindent
{\bf Purpose}\\
Wait for all joints to stop moving.\\

\noindent
{\bf Return Value}\\
The function returns 0 on success and non-zero otherwise.\\

\noindent
{\bf Description}\\
This function is used to wait for all joint motions to finish. Functions such as
\texttt{poseJoint()} and \texttt{moveJoint()} do not wait for a joint to finish
moving before continuing to allow multiple joints to move at the same time. The
\texttt{moveWait()} or \texttt{waitMotor()} functions are used to wait for
robotic motions to complete.

Please note that if this function is called after a motor has been commanded to
turn indefinitely, this function may never return and your program may hang.\\

\noindent
{\bf Example}\\
See the sample program in Section \ref{subsec:simple.cpp} on page \pageref{subsec:simple.cpp}.
\noindent

\noindent
{\bf See Also}\\
\texttt{moveWait(), waitMotor()}

%\CPlot::\DataThreeD(), \CPlot::\DataFile(), \CPlot::\Plotting(), \plotxy().\\

\noindent
\vspace{5pt}
\rule{4.5in}{0.015in}\\
\noindent
{\LARGE \texttt{CMobot::moveToZero()}\index{CMobot::moveToZero()}}\\
{\LARGE \texttt{CMobot::moveToZeroNB()}\index{CMobot::moveToZeroNB()}}\\
%\phantomsection
\addcontentsline{toc}{section}{moveToZero()}
\addcontentsline{toc}{section}{moveToZeroNB()}

\noindent
{\bf Synopsis}
\vspace{-8pt}
\begin{verbatim}
#include <mobot.h>
int CMobot::moveToZero();
int CMobot::moveToZeroNB();
\end{verbatim}

\noindent
{\bf Purpose}\\
Move all of the joints of a robot to their zero position.\\

\noindent
{\bf Return Value}\\
The function returns 0 on success and non-zero otherwise.\\

\noindent
{\bf Parameters}\\
None.\\

\noindent
{\bf Description}\\
\vspace{-12pt}
\begin{quote}
{\bf CMobot::moveToZero()}\\
This function moves all of the joints of a robot to their zero position.

{\bf CMobot::moveToZeroNB()}\\
This function moves all of the joints of a robot to their zero position.

The function \texttt{moveToZeroNB()} is the non-blocking version of
the \texttt{moveToZero()} function, which means that the function will return
immediately and the physical robot motion will occur asynchronously. For
more details on blocking and non-blocking functions, please refer to 
Section \ref{sec:blocking} on page \pageref{sec:blocking}.\\
\end{quote}

\noindent
{\bf Example}\\
Please see the demo at Section \ref{sec:democode} on page \pageref{sec:democode}.\\
\noindent

\noindent
{\bf See Also}\\

%\CPlot::\DataThreeD(), \CPlot::\DataFile(), \CPlot::\Plotting(), \plotxy().\\

%\noindent
\vspace{5pt}
\rule{4.5in}{0.015in}\\
\noindent
{\LARGE \texttt{CMobot::setJointDirection()}\index{CMobot::setJointDirection()}}\\
%\phantomsection
\addcontentsline{toc}{section}{setJointDirection()}

\noindent
{\bf Synopsis}
\vspace{-8pt}
\begin{verbatim}
#include <mobot.h>
enum robot_motor_direction_e
{
  IMOBOT_JOINT_DIR_AUTO,
  IMOBOT_JOINT_DIR_FORWARD,
  IMOBOT_JOINT_DIR_BACKWARD
};
int CMobot::setJointDirection(int id, int direction);
\end{verbatim}

\noindent
{\bf Purpose}\\
Set's a motor's direction. In conjunction with \texttt{setJointSpeed()}, this
function may be used to cause a motor to turn indefinitely.\\

\noindent
{\bf Return Value}\\
The function returns 0 on success and non-zero otherwise.\\

\noindent
{\bf Parameters}
\vspace{-0.1in}
\begin{description}
\item               
\begin{tabular}{p{20 mm}p{145 mm}}
\texttt{id} & The joint number to move. \\
\texttt{direction} & A value indicating the desired direction.
\end{tabular}
\end{description}

\noindent
{\bf Description}\\
This function is used to set a motor's turn direction. Possible values for the
direction are:
\begin{itemize}
\item 0: Automatic direction. This is the default setting. 
\item 1: Forward. If this value is used with a non-zero speed set using the
\texttt{setJointSpeed()} function, the motor will turn forward indefinitely.
\item 2: Backward. Similar to "1", except the motor will spin backward.
\end{itemize}

\noindent
{\bf Example}\\
\noindent

\noindent
{\bf See Also}\\

%\CPlot::\DataThreeD(), \CPlot::\DataFile(), \CPlot::\Plotting(), \plotxy().\\

\noindent
\vspace{5pt}
\rule{4.5in}{0.015in}\\
\noindent
{\LARGE \texttt{CMobot::setJointSpeed()}\index{setJointSpeed()}}\\
%\phantomsection
\addcontentsline{toc}{section}{setJointSpeed()}

\noindent
{\bf Synopsis}\\
\begin{verbatim}
#include <mobot.h>
int CMobot::setJointSpeed(int id, double speed);
\end{verbatim}

\noindent
{\bf Purpose}\\
Get the speed of a joint on the iMobot.\\

\noindent
{\bf Return Value}\\
The function returns 0 on success and non-zero otherwise.\\

\noindent
{\bf Parameters}
\vspace{-0.1in}
\begin{description}
\item               
\begin{tabular}{p{10 mm}p{145 mm}}
\texttt{id} & The joint number to pose. \\
\texttt{speed} & An variable of type \texttt{double} indicating the requested speed.
\end{tabular}
\end{description}

\noindent
{\bf Description}\\
This function is used to set the speed of a joint of an iMobot. Valid speed
values range from 0 to 1.
\noindent\\
{\bf Example}\\
\noindent

\noindent
{\bf See Also}\\

%\CPlot::\DataThreeD(), \CPlot::\DataFile(), \CPlot::\Plotting(), \plotxy().\\

\noindent
\vspace{5pt}
\rule{4.5in}{0.015in}\\
\noindent
{\LARGE \texttt{CMobotGroup::setJointSpeedRatio()}\index{CMobotGroup::setJointSpeedRatio()}}\\
%\phantomsection
\addcontentsline{toc}{section}{setJointSpeedRatio()}

\noindent
{\bf Synopsis}
\vspace{-8pt}
\begin{verbatim}
#include <mobot.h>
int CMobotGroup::setJointSpeedRatio(robotJointId_t id, double ratio);
\end{verbatim}

\noindent
{\bf Purpose}\\
Set the speed ratio settings of a joint on all robots in the group.\\

\noindent
{\bf Return Value}\\
The function returns 0 on success and non-zero otherwise.\\

\noindent
{\bf Parameters}
\vspace{-0.1in}
\begin{description}
\item               
\begin{tabular}{p{10 mm}p{145 mm}}
\texttt{id} & Set the speed ratio setting of this joint. This is an 
enumerated type discussed in Section \ref{sec:robotJointId_t} on page
\pageref{sec:robotJointId_t}.\\
\texttt{ratio} & A variable of type double with a value from 0 to 1. 
\end{tabular}
\end{description}

\noindent
{\bf Description}\\
This function is used to set the speed ratio setting of a joint for all robots in the group. The speed
ratio setting of a joint is the percentage of the maximum joint speed, and the
value ranges from 0 to 1. In other words, if the ratio is set to 0.5, the joint 
will turn at 50\% of its maximum angular velocity while moving continuously
or moving to a new goal position.\\

\noindent
{\bf Example}\\
\noindent

\noindent
{\bf See Also}\\
\texttt{setJointSpeeds(), setJointSpeedRatio()}

%\CPlot::\DataThreeD(), \CPlot::\DataFile(), \CPlot::\Plotting(), \plotxy().\\

\noindent
\vspace{5pt}
\rule{4.5in}{0.015in}\\
\noindent
{\LARGE \texttt{CMobot::setJointSpeedRatios()}\index{CMobot::setJointSpeedRatios()}}\\
%\phantomsection
\addcontentsline{toc}{section}{setJointSpeedRatios()}

\noindent
{\bf Synopsis}
\vspace{-8pt}
\begin{verbatim}
#include <mobot.h>
int CMobot::setJointSpeedRatios(double ratio1, double ratio2, double ratio3, double ratio4);
\end{verbatim}

\noindent
{\bf Purpose}\\
Set the speed ratio settings of all joints on the robot.\\

\noindent
{\bf Return Value}\\
The function returns 0 on success and non-zero otherwise.\\

\noindent
{\bf Parameters}
\vspace{-0.1in}
\begin{description}
\item               
\begin{tabular}{p{10 mm}p{145 mm}}
\texttt{ratio1} & The speed ratio setting for the first joint. \\ 
\texttt{ratio2} & The speed ratio setting for the second joint. \\
\texttt{ratio3} & The speed ratio setting for the third joint. \\
\texttt{ratio4} & The speed ratio setting for the fourth joint. \\
\end{tabular}
\end{description}

\noindent
{\bf Description}\\
This function is used to simultaneously set the angular speed ratio settings of
all four joints of a robot. The speed ratio is a percentage of the maximum
speed of a joint, expressed in a value from 0 to 1.\\

\noindent
{\bf Example}\\
\noindent

\noindent
{\bf See Also}\\
\texttt{getJointSpeeds(), setJointSpeed(), getJointSpeed()}

%\CPlot::\DataThreeD(), \CPlot::\DataFile(), \CPlot::\Plotting(), \plotxy().\\

\noindent
\vspace{5pt}
\rule{4.5in}{0.015in}\\
\noindent
{\LARGE \texttt{CMobot::setJointSpeeds()}\index{CMobot::setJointSpeeds()}}\\
%\phantomsection
\addcontentsline{toc}{section}{setJointSpeeds()}

\noindent
{\bf Synopsis}
\begin{verbatim}
#include <mobot.h>
int CMobot::setJointSpeeds(double speeds[4]);
\end{verbatim}

\noindent
{\bf Purpose}\\
Set the speed settings of all joints on the robot.\\

\noindent
{\bf Return Value}\\
The function returns 0 on success and non-zero otherwise.\\

\noindent
{\bf Parameters}
\vspace{-0.1in}
\begin{description}
\item               
\begin{tabular}{p{10 mm}p{145 mm}}
\texttt{speeds} & An array of type double. Each element of the array
represents the speed, expressed in radians per second, to set a joint. \\
\end{tabular}
\end{description}

\noindent
{\bf Description}\\
This function is used to simultaneously set the angular speed settings of
all four joints of a robot. \\

\noindent
{\bf Example}\\
\noindent

\noindent
{\bf See Also}\\
\texttt{getJointSpeeds(), setJointSpeed(), getJointSpeed()}

%\CPlot::\DataThreeD(), \CPlot::\DataFile(), \CPlot::\Plotting(), \plotxy().\\

\noindent
\vspace{5pt}
\rule{4.5in}{0.015in}\\
\noindent
{\LARGE \texttt{CMobotGroup::setTwoWheelRobotSpeed()}\index{CMobotGroup::setTwoWheelRobotSpeed()}}\\
%\phantomsection
\addcontentsline{toc}{section}{setTwoWheelRobotSpeed()}

\noindent
{\bf Synopsis}
\vspace{-8pt}
\begin{verbatim}
#include <mobot.h>
int CMobotGroup::setTwoWheelRobotSpeed(double speed, double radius);
\end{verbatim}

\noindent
{\bf Purpose}\\
Roll the mobots in the group at a certain speed in a straight line.\\

\noindent
{\bf Return Value}\\
The function returns 0 on success and non-zero otherwise.\\

\noindent
{\bf Parameters}
\vspace{-0.1in}
\begin{description}
\item               
\begin{tabular}{p{10 mm}p{145 mm}}
\texttt{speed} & The speed at which to roll the mobot. The units used will be the units
specified in the \texttt{unit} parameter. \\
\texttt{radius} & The radius of the wheels attached to the mobot. The units of the parameter
should match the units provided in the \texttt{unit} parameter. \\

\begin{tabular}{ll}
speed & radius \\
\hline \\
cm/s & cm \\
m/s & m \\
inch/s & inch \\
foot/s & foot \\
\hline
\end{tabular}
\end{tabular}
\end{description}

\noindent
{\bf Description}\\
This function is used to make a two wheeled mobot roll at a certain speed. The desired 
speed and radius of the wheels is provided and the function will rotate the wheels at the
appropriate rate in order to achieve the desired speed.
\noindent\\
{\bf Example}\\
\noindent

\noindent
{\bf See Also}\\

%\CPlot::\DataThreeD(), \CPlot::\DataFile(), \CPlot::\Plotting(), \plotxy().\\

\noindent
\vspace{5pt}
\rule{4.5in}{0.015in}\\
\noindent
{\LARGE \texttt{CMobot::stopAllJoints()}\index{CMobot::stopAllJoints()}}\\
{\LARGE \texttt{CMobot::stopOneJoint()}\index{CMobot::stopOneJoint()}}\\
{\LARGE \texttt{CMobot::stopTwoJoints()}\index{CMobot::stopTwoJoints()}}\\
{\LARGE \texttt{CMobot::stopThreeJoints()}\index{CMobot::stopThreeJoints()}}\\
%\phantomsection
\addcontentsline{toc}{section}{stopAllJoints()}
\addcontentsline{toc}{section}{stopOneJoint()}
\addcontentsline{toc}{section}{stopTwoJoints()}
\addcontentsline{toc}{section}{stopThreeJoints()}

\noindent
{\bf Synopsis}
\vspace{-8pt}
\begin{verbatim}
#include <mobot.h>
int CMobot::stopAllJoints();
int CMobot::stopOneJoint(mobotJointId_t id);
int CMobot::stopTwoJoints(mobotJointId_t id1, mobotJointId_t id2);
int CMobot::stopThreeJoints(
    mobotJointId_t id1,
    mobotJointId_t id2,
    mobotJointId_t id3);
\end{verbatim}

\noindent
{\bf Purpose}\\
These functions stop joints on a mobot.\\

\noindent
{\bf Return Value}\\
The function returns 0 on success and non-zero otherwise.\\

\noindent
{\bf Description}\\
These functions are used to stop joints on a mobot. A stopped joint
will immediately cease any and all actuation and go limp.

\noindent
{\bf Example}\\
\noindent

\noindent
{\bf See Also}\\
\texttt{setJointSpeed(), setJointSpeeds()}

%\CPlot::\DataThreeD(), \CPlot::\DataFile(), \CPlot::\Plotting(), \plotxy().\\

