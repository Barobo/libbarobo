\noindent
\vspace{5pt}
\rule{4.5in}{0.015in}\\
\noindent
{\LARGE \texttt{CMobotGroup::setJointSpeedRatio()}\index{CMobotGroup::setJointSpeedRatio()}}\\
%\phantomsection
\addcontentsline{toc}{section}{setJointSpeedRatio()}

\noindent
{\bf Synopsis}
\vspace{-8pt}
\begin{verbatim}
#include <mobot.h>
int CMobotGroup::setJointSpeedRatio(mobotJointId_t id, double ratio);
\end{verbatim}

\noindent
{\bf Purpose}\\
Set the speed ratio settings of a joint on all mobots in the group.\\

\noindent
{\bf Return Value}\\
The function returns 0 on success and non-zero otherwise.\\

\noindent
{\bf Parameters}
\vspace{-0.1in}
\begin{description}
\item               
\begin{tabular}{p{10 mm}p{145 mm}}
\texttt{id} & Set the speed ratio setting of this joint. This is an 
enumerated type discussed in Section \ref{sec:mobotJointId_t} on page
\pageref{sec:mobotJointId_t}.\\
\texttt{ratio} & A variable of type double with a value from 0 to 1. 
\end{tabular}
\end{description}

\noindent
{\bf Description}\\
This function is used to set the speed ratio setting of a joint for all mobots in the group. The speed
ratio setting of a joint is the percentage of the maximum joint speed, and the
value ranges from 0 to 1. In other words, if the ratio is set to 0.5, the joint 
will turn at 50\% of its maximum angular velocity while moving continuously
or moving to a new goal position.\\

\noindent
{\bf Example}\\
\noindent

\noindent
{\bf See Also}\\
\texttt{setJointSpeeds(), setJointSpeedRatio()}

%\CPlot::\DataThreeD(), \CPlot::\DataFile(), \CPlot::\Plotting(), \plotxy().\\
