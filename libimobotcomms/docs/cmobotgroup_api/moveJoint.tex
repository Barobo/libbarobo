\noindent
\vspace{5pt}
\rule{4.5in}{0.015in}\\
\noindent
{\LARGE \texttt{CMobotGroup::moveJoint()}\index{CMobotGroup::moveJoint()}}\\
{\LARGE \texttt{CMobotGroup::moveJointNB()}\index{CMobotGroup::moveJointNB()}}\\
%\phantomsection
\addcontentsline{toc}{section}{moveJoint()}
\addcontentsline{toc}{section}{moveJointNB()}

\noindent
{\bf Synopsis}
\vspace{-8pt}
\begin{verbatim}
#include <mobot.h>
int CMobotGroup::moveJoint(robotJointId_t id, double angle);
int CMobotGroup::moveJointNB(robotJointId_t id, double angle);
\end{verbatim}

\noindent
{\bf Purpose}\\
Move a joint on the robots in the group by a specified angle with respect
to the current position.\\

\noindent
{\bf Return Value}\\
The function returns 0 on success and non-zero otherwise.\\

\noindent
{\bf Parameters}\\
\vspace{-0.1in}
\begin{description}
\item               
\begin{tabular}{p{10 mm}p{145 mm}}
\texttt{id} & The joint number to wait for. \\
\texttt{angle} & The angle in degrees to move the motor, relative to the current position.  \\
\end{tabular}
\end{description}

\noindent
{\bf Description}\\
\vspace{-12pt}
\begin{quote}
{\bf CMobot::moveJoint()}\\
This function commands the motor to move by an angle relative to the
joint's current position at the joints current speed setting.
The current motor speed may be set with the
\texttt{setJointSpeed()} member function. Please note that if the motor speed
is set to zero, the motor will not move after calling the
\texttt{moveJoint()} function. 

{\bf CMobot::moveJointNB()}\\
This function commands the motor to move by an angle relative to the
joint's current position at the joints current speed setting.
The current motor speed may be set with the
\texttt{setJointSpeed()} member function. Please note that if the motor speed
is set to zero, the motor will not move after calling the
\texttt{moveJoint()} function. 

The function \texttt{moveJoint()} is a blocking function, which means that 
the function will not return until the commanded motion is 
completed. The function \texttt{moveJointNB()} is the non-blocking version of
the \texttt{moveJoint()} function, which means that the function will return
immediately and the physical robot motion will occur asynchronously. For
more details on blocking and non-blocking functions, please refer to 
Section \ref{sec:blocking} on page \pageref{sec:blocking}.\\
\end{quote}

\noindent
{\bf Example}\\
Please see the example in Section \ref{sec:democode} on page \pageref{sec:democode}.\\
\noindent

\noindent
{\bf See Also}\\
\texttt{connectWithAddress()}

%\CPlot::\DataThreeD(), \CPlot::\DataFile(), \CPlot::\Plotting(), \plotxy().\\
