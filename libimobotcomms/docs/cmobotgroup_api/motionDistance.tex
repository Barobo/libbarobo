\noindent
\vspace{5pt}
\rule{4.5in}{0.015in}\\
\noindent
{\LARGE \texttt{CMobotGroup::motionDistance()}\index{CMobotGroup::motionDistance()}}\\
{\LARGE \texttt{CMobotGroup::motionDistanceNB()}\index{CMobotGroup::motionDistanceNB()}}\\
%\phantomsection
\addcontentsline{toc}{section}{motionDistance()}
\addcontentsline{toc}{section}{motionDistanceNB()}

\noindent
{\bf Synopsis}
\vspace{-8pt}
\begin{verbatim}
#include <mobot.h>
int CMobotGroup::motionDistance(double radius, double distance);
int CMobotGroup::motionDistanceNB(double radius, double distance);
\end{verbatim}

\noindent
{\bf Purpose}\\
Use the faceplates as wheels to roll mobots forward a certain distance.\\

\noindent
{\bf Return Value}\\
The function returns 0 on success and non-zero otherwise.\\

\noindent
{\bf Parameters}\\
\vspace{-0.1in}
\begin{description}
\item               
\begin{tabular}{p{15 mm}p{145 mm}}
\texttt{radius} & The radius of the wheels.\\
\texttt{distance} & The distance to roll the Mobot.\\
\end{tabular}
\end{description}

\noindent
{\bf Description}\\
\vspace{-12pt}
\begin{quote}
{\bf CMobot::motionDistance()}\\
This function is used to roll a group of wheeled Mobots forward a certain distance. 
Note that the unit of measurement for the radius of the wheels much match the units 
used for distance.

This function has both a blocking and non-blocking version.
The blocking version, \texttt{motionDistance()}, will block until the
mobot motion has completed. The non-blocking version, \texttt{motionDistanceNB()},
will return immediately, and the motion will be performed asynchronously.\\
\end{quote}

\noindent
{\bf See Also}\\

%\CPlot::\DataThreeD(), \CPlot::\DataFile(), \CPlot::\Plotting(), \plotxy().\\
