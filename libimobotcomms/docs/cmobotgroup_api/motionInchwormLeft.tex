\noindent
\vspace{5pt}
\rule{4.5in}{0.015in}\\
\noindent
{\LARGE \texttt{CMobotGroup::motionInchwormLeft()}\index{CMobotGroup::motionInchwormLeft()}}\\
{\LARGE \texttt{CMobotGroup::motionInchwormLeftNB()}\index{CMobotGroup::motionInchwormLeftNB()}}\\
%\phantomsection
\addcontentsline{toc}{section}{motionInchwormLeft()}
\addcontentsline{toc}{section}{motionInchwormLeftNB()}

\noindent
{\bf Synopsis}
\vspace{-8pt}
\begin{verbatim}
#include <mobot.h>
int CMobotGroup::motionInchwormLeft();
int CMobotGroup::motionInchwormLeftNB();
\end{verbatim}

\noindent
{\bf Purpose}\\
Make all robots in the group perform the inch-worm gait to the left.\\

\noindent
{\bf Return Value}\\
The function returns 0 on success and non-zero otherwise.\\

\noindent
{\bf Parameters}\\
None.\\

\noindent
{\bf Description}\\
This function causes the robots to perform a single cycle of the inchworm gait
to the left. This function comes in two flavors; a blocking and a non blocking
version. The function \texttt{motionInchwormLeft()} is blocking, and the function
will hang until the motion has finished. The alternative function, \texttt{motionInchwormLeftNB()} 
will return immediately, and the motion will execute asynchronously. \\

\noindent
{\bf See Also}\\
\texttt{motionInchwormRight()}

%\CPlot::\DataThreeD(), \CPlot::\DataFile(), \CPlot::\Plotting(), \plotxy().\\
