\noindent
\vspace{5pt}
\rule{4.5in}{0.015in}\\
\noindent
{\LARGE \texttt{CMobotGroup::motionTumbleLeft()}\index{CMobotGroup::motionTumbleLeft()}}\\
{\LARGE \texttt{CMobotGroup::motionTumbleLeftNB()}\index{CMobotGroup::motionTumbleLeftNB()}}\\
%\phantomsection
\addcontentsline{toc}{section}{motionTumbleLeft()}
\addcontentsline{toc}{section}{motionTumbleLeftNB()}

\noindent
{\bf Synopsis}
\vspace{-8pt}
\begin{verbatim}
#include <mobot.h>
int CMobotGroup::motionTumbleLeft(int num);
int CMobotGroup::motionTumbleLeftNB(int num);
\end{verbatim}

\noindent
{\bf Purpose}\\
Make the robots in the group tumble end over end.\\

\noindent
{\bf Return Value}\\
The function returns 0 on success and non-zero otherwise.\\

\noindent
{\bf Parameters}\\
\vspace{-0.1in}
\begin{description}
\item               
\begin{tabular}{p{10 mm}p{145 mm}}
\texttt{num} & The number of times to tumble. \\
\end{tabular}
\end{description}

\noindent
{\bf Description}\\
\vspace{-12pt}
\begin{quote}
{\bf CMobot::motionTumbleLeft()}\\
This causes the robot to tumble end over end. The argument, \texttt{num},
indicates the number of times to tumble.

{\bf CMobot::motionTumbleLeftNB()}\\
This causes the robot to tumble end over end. The argument, \texttt{num},
indicates the number of times to tumble.

This function has both a blocking and non-blocking version.
The blocking version, \texttt{motionTumbleLeft()}, will block until the
robot motion has completed. The non-blocking version, \texttt{motionTumbleLeftNB()},
will return immediately, and the motion will be performed asynchronously.\\
\end{quote}

\noindent
{\bf See Also}\\

%\CPlot::\DataThreeD(), \CPlot::\DataFile(), \CPlot::\Plotting(), \plotxy().\\
