\noindent
\vspace{5pt}
\rule{4.5in}{0.015in}\\
\noindent
{\LARGE \texttt{CMobotGroup::isMoving()}\index{isMoving()}}\\
%\phantomsection
\addcontentsline{toc}{section}{isMoving()}

\noindent
{\bf Synopsis}\\
\begin{verbatim}
#include <mobot.h>
int CMobotGroup::isMoving();
\end{verbatim}

\noindent
{\bf Purpose}\\
Check to see if any robot in the group is currently moving any of its joints.\\

\noindent
{\bf Return Value}\\
This function returns 0 if none of the joints are being driven, 1 if any joint
is being driven, and -1 if there was an error communicating with the MoBot.\\

\noindent
{\bf Parameters}\\
None.\\

\noindent
{\bf Description}\\
This function is used to determine if any robot in the robot group is currently moving any of
its joints. \\

\noindent
{\bf Example}\\
\noindent

\noindent
{\bf See Also}\\

%\CPlot::\DataThreeD(), \CPlot::\DataFile(), \CPlot::\Plotting(), \plotxy().\\
