\noindent
\vspace{5pt}
\rule{4.5in}{0.015in}\\
\noindent
{\LARGE \texttt{CMobotGroup::moveJointTo()}\index{CMobotGroup::moveJointTo()}}\\
{\LARGE \texttt{CMobotGroup::moveJointToNB()}\index{CMobotGroup::moveJointToNB()}}\\
%\phantomsection
\addcontentsline{toc}{section}{moveJointTo()}
\addcontentsline{toc}{section}{moveJointToNB()}

\noindent
{\bf Synopsis}
\vspace{-8pt}
\begin{verbatim}
#include <mobot.h>
int CMobotGroup::moveJointTo(mobotJointId_t id, double angle);
int CMobotGroup::moveJointToNB(mobotJointId_t id, double angle);
\end{verbatim}

\noindent
{\bf Purpose}\\
Move a joint on mobots to an absolute position.\\

\noindent
{\bf Return Value}\\
The function returns 0 on success and non-zero otherwise.\\

\noindent
{\bf Parameters}\\
\vspace{-0.1in}
\begin{description}
\item               
\begin{tabular}{p{10 mm}p{145 mm}}
\texttt{id} & The joint number to wait for. \\
\texttt{angle} & The absolute angle in degrees to move the motor to.  \\
\end{tabular}
\end{description}

\noindent
{\bf Description}\\
\vspace{-12pt}
\begin{quote}
{\bf CMobot::moveJointTo()}\\
This function commands the motor on mobots in a group to move to a position specified in degrees at
the current motor's speed. The current motor speed may be set with the
\texttt{setJointSpeed()} member function. Please note that if the motor speed
is set to zero, the motor will not move after calling the
\texttt{moveJointTo()} function. 

{\bf CMobot::moveJointToNB()}\\
This function commands the motor on mobots in a group to move to a position specified in degrees at
the current motor's speed. The current motor speed may be set with the
\texttt{setJointSpeed()} member function. Please note that if the motor speed
is set to zero, the motor will not move after calling the
\texttt{moveJointToNB()} function. 

The function \texttt{moveJointTo()} is a blocking function, which means that 
the function will not return until the commanded motion is 
completed. The function \texttt{moveJointToNB()} is the non-blocking version of
the \texttt{moveJointTo()} function, which means that the function will return
immediately and the physical mobot motion will occur asynchronously. For
more details on blocking and non-blocking functions, please refer to 
Section \ref{sec:blocking} on page \pageref{sec:blocking}.\\
\end{quote}
\noindent
{\bf Example}\\
Please see the example in Section \ref{sec:democode} on page \pageref{sec:democode}.\\
\noindent

\noindent
{\bf See Also}\\

%\CPlot::\DataThreeD(), \CPlot::\DataFile(), \CPlot::\Plotting(), \plotxy().\\
