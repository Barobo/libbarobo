\noindent
\vspace{5pt}
\rule{4.5in}{0.015in}\\
\noindent
{\LARGE \texttt{CMobotGroup::motionSkinny()}\index{CMobotGroup::motionSkinny()}}\\
{\LARGE \texttt{CMobotGroup::motionSkinnyNB()}\index{CMobotGroup::motionSkinnyNB()}}\\
%\phantomsection
\addcontentsline{toc}{section}{motionSkinny()}
\addcontentsline{toc}{section}{motionSkinnyNB()}

\noindent
{\bf Synopsis}
\vspace{-8pt}
\begin{verbatim}
#include <mobot.h>
int CMobotGroup::motionSkinny(double angle);
int CMobotGroup::motionSkinnyNB(double angle);
\end{verbatim}

\noindent
{\bf Purpose}\\
Move the mobots in the group into a skinny profile.\\

\noindent
{\bf Return Value}\\
The function returns 0 on success and non-zero otherwise.\\

\noindent
{\bf Parameters}\\
\vspace{-0.1in}
\begin{description}
\item               
\begin{tabular}{p{10 mm}p{145 mm}}
\texttt{angle} & The angle in degrees to move the joints. A value of zero means a
completely flat profile, while a value of 90 degrees means a fully skinny
profile.  \\
\end{tabular}
\end{description}


\noindent
{\bf Description}\\
\vspace{-12pt}
\begin{quote}
{\bf CMobot::motionSkinny()}\\
This function makes the mobots assume a skinny rolling profile.

{\bf CMobot::motionSkinnyNB()}\\
This function makes the mobots assume a skinny rolling profile.

This function has both a blocking and non-blocking version.
The blocking version, \texttt{motionSkinny()}, will block until the
mobot motion has completed. The non-blocking version, \texttt{motionSkinnyNB()},
will return immediately, and the motion will be performed asynchronously.\\
\end{quote}

\noindent
{\bf See Also}\\

%\CPlot::\DataThreeD(), \CPlot::\DataFile(), \CPlot::\Plotting(), \plotxy().\\
