\noindent
\vspace{5pt}
\rule{4.5in}{0.015in}\\
\noindent
{\LARGE \texttt{CMobotGroup::setJointSafetyAngle()}\index{CMobotGroup::setJointSafetyAngle()}}\\
%\phantomsection
\addcontentsline{toc}{section}{setJointSafetyAngle()}

\noindent
{\bf Synopsis}
\vspace{-8pt}
\begin{verbatim}
#include <mobot.h>
int CMobotGroup::setJointSafetyAngle(double degrees);
\end{verbatim}

\noindent
{\bf Purpose}\\
Set the current angle safety limit of the Mobot.\\

\noindent
{\bf Return Value}\\
The function returns 0 on success and -1 on failure.\\

\noindent
{\bf Parameters}\\
The requested joint safety angle limit for the Mobot.\\

\noindent
{\bf Description}\\
The Mobot is equipped with a safety feature to protect itself and its surrounding
environment. When a motor deviates by a certain amount from its expected value, 
the Mobot will shut off all power to the motor, in case it has hit an obstacle,
or for any other reason. The amount of deviation required to trigger the safety
protocol is the joint safety angle which can be set using this function.
The default setting is 10 degrees. Higher values indicate ``less safe'' behavior
of the Mobot because the Mobot will not engage safety protocols until the joint has
deviated by a greater amount. Values greater than 90 degrees effectively disengage
the Mobot's safety protocols altogether, and this function should be used with care.

\noindent
{\bf Example}\\
\noindent

\noindent
{\bf See Also}\\
\texttt{setJointSafetyAngleTimeout()}\\

%\CPlot::\DataThreeD(), \CPlot::\DataFile(), \CPlot::\Plotting(), \plotxy().\\

