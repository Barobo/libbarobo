\begin{tabular}{p{1.75in}p{4.5in}}
Compound Motions & These are convenience functions of commonly used compound motions. \\
\hline
\texttt{motionArch()}  & Move each robot in the group into an arched configuration. \\
\texttt{motionArchNB()}  & Identical to \texttt{motionArch()} but non-blocking. \\
\texttt{motionInchwormLeft()}  & Inchworm motion towards the left. \\
\texttt{motionInchwormLeftNB()}  & Identical to \texttt{motionInchwormLeft()} but non-blocking. \\
\texttt{motionInchwormRight()}  & Inchworm motion towards the right. \\
\texttt{motionInchwormRightNB()}  & Identical to \texttt{motionInchwormRight()} but non-blocking. \\
\texttt{motionRollBackward()}  & Roll on the faceplates toward the backward direction. \\
\texttt{motionRollBackwardNB()}  & Identical to \texttt{motionRollBackward()} but non-blocking. \\
\texttt{motionRollForward()}  & Roll on the faceplates forwards. \\
\texttt{motionRollForwardNB()}  & Identical to \texttt{motionRollForward()} but non-blocking. \\
\texttt{motionSkinny()}  & Move the robots into a skinny configuration. \\
\texttt{motionSkinnyNB()}  & Identical to \texttt{motionSkinnyNB()} but non-blocking. \\
\texttt{motionStand()}  & Stand the robots up on its end. \\
\texttt{motionStandNB()}  & Identical to \texttt{motionStandNB()} but non-blocking. \\
\texttt{motionTumble()}  & Tumble the robots end over end. \\
\texttt{motionTumbleNB()}  & Identical to \texttt{motionTumbleNB()} but non-blocking. \\
\texttt{motionTurnLeft()}  & Rotate the robots counterclockwise. \\
\texttt{motionTurnLeftNB()}  & Identical to \texttt{motionTurnLeft()} but non-blocking. \\
\texttt{motionTurnRight()}  & Rotate the robots clockwise. \\
\texttt{motionTurnRightNB()}  & Identical to \texttt{motionTurnRight()} but non-blocking. \\
\texttt{motionUnstand()}  & Move each robot in the group currently standing on its end down into zero position. \\
\texttt{motionUnstandNB()}  & Identical to \texttt{motionUnstand()} but non-blocking. \\
\texttt{motionWait()}  & Wait for preprogrammed robotic motions to complete. \\
\hline
\end{tabular}
