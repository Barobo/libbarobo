\noindent
\vspace{5pt}
\rule{4.5in}{0.015in}\\
\noindent
{\LARGE \texttt{CMobot::moveTo()}\index{CMobot::moveTo()}}\\
{\LARGE \texttt{CMobot::moveToNB()}\index{CMobot::moveToNB()}}\\
%\phantomsection
\addcontentsline{toc}{section}{moveTo()}
\addcontentsline{toc}{section}{moveToNB()}

\noindent
{\bf Synopsis}
\vspace{-8pt}
\begin{verbatim}
#include <mobot.h>
int CMobot::moveTo(double angle1, double angle2, double angle3, double angle4);
int CMobot::moveToNB(double angle1, double angle2, double angle3, double angle4);
\end{verbatim}

\noindent
{\bf Purpose}\\
Move all of the joints of a robot to the specified positions.\\

\noindent
{\bf Return Value}\\
The function returns 0 on success and non-zero otherwise.\\

\noindent
{\bf Parameters}\\
None.\\

\noindent
{\bf Description}\\
This function moves all of the joints of a robot to the specified absolute positions. 

The function \texttt{moveTo()} is a blocking function, which means that 
the function will not return until the commanded motion is 
completed. The function \texttt{moveToNB()} is the non-blocking version of
the \texttt{moveTo()} function, which means that the function will return
immediately and the physical robot motion will occur asynchronously. For
more details on blocking and non-blocking functions, please refer to 
Section \ref{sec:blocking} on page \pageref{sec:blocking}.\\

\noindent
{\bf Example}\\
Please see the demo at Section \ref{sec:democode} on page \pageref{sec:democode}.\\
\noindent

\noindent
{\bf See Also}\\

%\CPlot::\DataThreeD(), \CPlot::\DataFile(), \CPlot::\Plotting(), \plotxy().\\
