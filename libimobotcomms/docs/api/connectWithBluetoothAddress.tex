\noindent
\vspace{5pt}
\rule{4.5in}{0.015in} \\
\noindent
{\LARGE \texttt{CMobot::connectWithBluetoothAddress()}\index{CMobot::connectWithBluetoothAddress()}}\\
%\phantomsection
\addcontentsline{toc}{section}{connectWithBluetoothAddress()}

\noindent
{\bf Synopsis}
\vspace{-8pt}
\begin{verbatim}
#include <mobot.h>
int CMobot::connectWithBluetoothAddress(char address[], int channel = 1);
\end{verbatim}

\noindent
{\bf Purpose}\\
Connect to a remote mobot via Bluetooth by specifying the specific Bluetooth
address of the device.\\

\noindent
{\bf Return Value}\\
The function returns 0 on success and non-zero otherwise.\\

\noindent
{\bf Parameters}
\vspace{-0.1in}
\begin{description}
\item               
\begin{tabular}{p{10 mm}p{145 mm}}
\texttt{address} & The Bluetooth address of the mobot. \\
\texttt{channel} & (optional) The Bluetooth channel that the listening program is
listening on. The default channel is channel 1. \\
\end{tabular}
\end{description}

\noindent
{\bf Description}\\
This function is used to connect to a mobot. 

\noindent
{\bf Example}\\
\begin{verbatim}
mobot.connectWithBluetoothAddress("00:06:66:45:DA:02", 1);
mobot.connectWithBluetoothAddress("00:06:66:45:DA:F3");
\end{verbatim}
\noindent

\noindent
{\bf See Also}\\
\texttt{connect(), disconnect()}

%\CPlot::\DataThreeD(), \CPlot::\DataFile(), \CPlot::\Plotting(), \plotxy().\\
