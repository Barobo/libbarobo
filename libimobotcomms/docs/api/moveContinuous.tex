\noindent
\vspace{5pt}
\rule{4.5in}{0.015in}\\
\noindent
{\LARGE \texttt{CMobot::moveContinuous()}\index{moveContinuous()}}\\
%\phantomsection
\addcontentsline{toc}{section}{moveContinuous()}

\noindent
{\bf Synopsis}\\
\begin{verbatim}
#include <mobot.h>
int CMobot::moveContinuous(int dir1, int dir2, int dir3, int dir4);
\end{verbatim}

\noindent
{\bf Purpose}\\
Move the joints of an iMobot continuously in the specified directions.\\

\noindent
{\bf Return Value}\\
The function returns 0 on success and non-zero otherwise.\\

\noindent
{\bf Parameters}\\
Each integer parameter specifies the direction the joint should move. The types
are enumerated in \texttt{mobot.h} and have the following values:
\begin{itemize}
\item \texttt{MOBOT\_NEUTRAL} : The joint should not move.
\item \texttt{MOBOT\_FORWARD} : The joint will begin moving in the positive direction.
\item \texttt{MOBOT\_BACKWARD}: The joint will begin moving in the negative direction.
\end{itemize}

\noindent
{\bf Description}\\
This function causes joints of an iMobot to begin moving at the previously set
speed. The joints will continue moving until the joint hits a joint limit, or
the joint is stopped by setting the speed to zero.\\

\noindent
{\bf Example}\\
\noindent

\noindent
{\bf See Also}\\

%\CPlot::\DataThreeD(), \CPlot::\DataFile(), \CPlot::\Plotting(), \plotxy().\\
