\noindent
\vspace{5pt}
\rule{4.5in}{0.015in}\\
\noindent
{\LARGE \texttt{CMobot::setJointSpeeds()}\index{CMobot::setJointSpeeds()}}\\
%\phantomsection
\addcontentsline{toc}{section}{setJointSpeeds()}

\noindent
{\bf Synopsis}
\vspace{-8pt}
\begin{verbatim}
#include <mobot.h>
int CMobot::setJointSpeeds(double speed1, double speed2, double speed3, double speed4);
\end{verbatim}

\noindent
{\bf Purpose}\\
Set the speed settings of all joints on the mobot.\\

\noindent
{\bf Return Value}\\
The function returns 0 on success and non-zero otherwise.\\

\noindent
{\bf Parameters}
\vspace{-0.1in}
\begin{description}
\item               
\begin{tabular}{p{10 mm}p{145 mm}}
\texttt{speed1} & The speed for the first joint, in degrees per second. \\
\texttt{speed2} & The speed for the second joint, in degrees per second. \\
\texttt{speed3} & The speed for the third joint, in degrees per second. \\
\texttt{speed4} & The speed for the fourth joint, in degrees per second. \\
\end{tabular}
\end{description}

\noindent
{\bf Description}\\
This function is used to simultaneously set the angular speed settings of
all four joints of a mobot. \\

\noindent
{\bf Example}\\
\noindent

\noindent
{\bf See Also}\\
\texttt{getJointSpeeds(), setJointSpeed(), getJointSpeed()}

%\CPlot::\DataThreeD(), \CPlot::\DataFile(), \CPlot::\Plotting(), \plotxy().\\
