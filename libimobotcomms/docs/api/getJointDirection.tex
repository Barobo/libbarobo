\noindent
\vspace{5pt}
\rule{4.5in}{0.015in}\\
\noindent
{\LARGE \texttt{CMobot::getJointDirection()}\index{getJointDirection()}}\\
%\phantomsection
\addcontentsline{toc}{section}{getJointDirection()}

\noindent
{\bf Synopsis}\\
\begin{verbatim}
#include <mobot.h>
int CMobot::getJointDirection(mobotJointId_t id, int &direction);
\end{verbatim}

\noindent
{\bf Purpose}\\
Get the speed of a joint on the MoBot.\\

\noindent
{\bf Return Value}\\
The function returns 0 on success and non-zero otherwise.\\

\noindent
{\bf Parameters}
\vspace{-0.1in}
\begin{description}
\item               
\begin{tabular}{p{10 mm}p{145 mm}}
\texttt{id} & The joint number to pose. This is an enumerated type 
discussed in Section \ref{sec:mobotJointId_t} on page
\pageref{sec:mobotJointId_t}.\\
\texttt{direction} & An integer variable. This variable will be overwritten
with the current speed of the joint.
\end{tabular}
\end{description}

\noindent
{\bf Description}\\
This function is used to retrieve the motor's direction status. The valid
status directions are
\begin{itemize}
\item 0: Automatic direction
\item 1: Forward direction
\item 2: Backward direction
\end{itemize}

\noindent
{\bf Example}\\
\noindent

\noindent
{\bf See Also}\\
\texttt{setJointDirection()}

%\CPlot::\DataThreeD(), \CPlot::\DataFile(), \CPlot::\Plotting(), \plotxy().\\
