\noindent
\vspace{5pt}
\rule{4.5in}{0.015in}\\
\noindent
{\LARGE \texttt{CMobot::moveToDirect()}\index{CMobot::moveToDirect()}}\\
{\LARGE \texttt{CMobot::moveToDirectNB()}\index{CMobot::moveToDirectNB()}}\\
%\phantomsection
\addcontentsline{toc}{section}{moveToDirect()}
\addcontentsline{toc}{section}{moveToDirectNB()}

\noindent
{\bf Synopsis}
\vspace{-8pt}
\begin{verbatim}
#include <mobot.h>
int CMobot::moveToDirect(double angle1, double angle2, double angle3, double angle4);
int CMobot::moveToDirectNB(double angle1, double angle2, double angle3, double angle4);
\end{verbatim}

\noindent
{\bf Purpose}\\
Move all of the joints of a mobot to the specified positions.\\

\noindent
{\bf Return Value}\\
The function returns 0 on success and non-zero otherwise.\\

\noindent
{\bf Parameters}\\
\vspace{-0.1in}
\begin{description}
\item               
\begin{tabular}{p{15 mm}p{105 mm}}
\texttt{angle1} & The absolute position to move joint 1, expressed in degrees. \\
\texttt{angle2} & The absolute position to move joint 2, expressed in degrees. \\
\texttt{angle3} & The absolute position to move joint 3, expressed in degrees. \\
\texttt{angle4} & The absolute position to move joint 4, expressed in degrees. \\
\end{tabular}
\end{description}
\noindent

{\bf Description}\\
\vspace{-12pt}
\begin{quote}
{\bf CMobot::moveToDirect()}\\
This function moves all of the joints of a mobot to the specified absolute positions. 

{\bf CMobot::moveToDirectNB()}\\
This function moves all of the joints of a mobot to the specified absolute positions. 

The function \texttt{moveToDirectNB()} is the non-blocking version of
the \texttt{moveToDirect()} function, which means that the function will return
immediately and the physical mobot motion will occur asynchronously. For
more details on blocking and non-blocking functions, please refer to 
Section \ref{sec:blocking} on page \pageref{sec:blocking}.\\
\end{quote}

Note that the main difference between this function and its sister function,
\texttt{moveJointTo()}, is that this function explicitly moves the motor in
whichever direction is the shortest to achieve its goal.

\noindent
{\bf Example}\\
The \texttt{moveToDirect()} function moves the joint to the requested position choosing
the shortest distance. For instance, consider the following function calls:
\begin{verbatim}
mobot.moveTo(350, 0, 0, 0);
\end{verbatim}
When the preceding line of code is executed, joint 1 on the mobot will rotate 350
degrees in the forward direction. However, in the following line,
\begin{verbatim}
mobot.moveToDirect(350, 0, 0, 0);
\end{verbatim}
the first joint of the mobot will actually rotate -10 degrees in the negative direction to achieve the
desired final position. Instead of rotating the entire 350 degrees, the mobot calculates
that a movement of -10 degrees will require less traveled distance to achieve the same final
position.\\
\noindent

\noindent
{\bf See Also}\\

%\CPlot::\DataThreeD(), \CPlot::\DataFile(), \CPlot::\Plotting(), \plotxy().\\
