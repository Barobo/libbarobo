\noindent
\vspace{5pt}
\rule{4.5in}{0.015in}\\
\noindent
{\LARGE \texttt{CMobot::motionWait()}\index{CMobot::motionWait()}}\\
%\phantomsection
\addcontentsline{toc}{section}{motionWait()}

\noindent
{\bf Synopsis}
\begin{verbatim}
#include <mobot.h>
int CMobot::motionWait();
\end{verbatim}

\noindent
{\bf Purpose}\\
Wait for a motion to complete execution.\\

\noindent
{\bf Return Value}\\
The function returns 0 on success and non-zero otherwise.\\

\noindent
{\bf Description}\\
This function is used to wait for a motion function to fully complete its cycle.
The \texttt{CMobot} motion functions are those member functions which begin
with ``motion'' as part of their name, such as \texttt{motionInchwormLeft()}.


\noindent
{\bf Example}\\
\noindent

\noindent
{\bf See Also}\\

%\CPlot::\DataThreeD(), \CPlot::\DataFile(), \CPlot::\Plotting(), \plotxy().\\
