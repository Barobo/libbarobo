\noindent
\vspace{5pt}
\rule{4.5in}{0.015in}\\
\noindent
{\LARGE \texttt{CMobot::recordAngles()}\index{CMobot::recordAngles()}}\\
%\phantomsection
\addcontentsline{toc}{section}{recordAngles()}

\noindent
{\bf Synopsis}
\vspace{-8pt}
\begin{verbatim}
#include <mobot.h>
int CMobot::recordAngles(robotJointId_t id, double time[:], double angle1[:], double angle2[:], double angle3[:], double angle4[:], int num, int msecs);
\end{verbatim}

\noindent
{\bf Purpose}\\
Record joint angle data for all joint for a set amount of time at a specified time interval.\\

\noindent
{\bf Return Value}\\
The function returns 0 on success and non-zero otherwise.\\

\noindent
{\bf Parameters}\\
\vspace{-0.1in}
\begin{description}
\item               
\begin{tabular}{p{15 mm}p{145 mm}}
\texttt{id} & The joint number. This is an enumerated type 
discussed in Section \ref{sec:robotJointId_t} on page
\pageref{sec:robotJointId_t}.\\
\texttt{time} & An array which will store time values for each of the angle readings. \\
\texttt{angle1} & An array which will store the angle values for joint 1. \\
\texttt{angle2} & An array which will store the angle values for joint 2. \\
\texttt{angle3} & An array which will store the angle values for joint 3. \\
\texttt{angle4} & An array which will store the angle values for joint 4. \\
\texttt{num} & The size of the arrays. \\
\texttt{msecs} & The number of milliseconds between angle readings.
\end{tabular}
\end{description}

\noindent
{\bf Description}\\
This function is used to accurately record the motion of a Mobot joint at a relatively fast
rate. The function will fill the \texttt{time} and \texttt{angle} arrays with data
at the rate specified by \texttt{msecs}. A typical value for \texttt{msecs} is 100, or
polling 10 times a second. If the communication speed cannot maintain 
the requested rate, (if \texttt{msecs} is too low, in other words), the function will
collect data as fast as possible. 

The length of time to collect the data can be calculated by the formula \\
\begin{equation*}
\text{seconds} = (\text{num} \times \text{msecs}) / 1000
\end{equation*}

This function is a non-blocking function. After calling this function, a call to
\texttt{recordWait()} should be performed to ensure that the data has been fully collected.

\noindent
{\bf Example}\\
\noindent

\noindent
{\bf See Also}\\
\texttt{recordAngle(), recordWait()}
%\CPlot::\DataThreeD(), \CPlot::\DataFile(), \CPlot::\Plotting(), \plotxy().\\
