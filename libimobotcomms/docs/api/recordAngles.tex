\noindent
\vspace{5pt}
\rule{4.5in}{0.015in}\\
\noindent
{\LARGE \texttt{CMobot::recordAngles()}\index{CMobot::recordAngles()}}\\
%\phantomsection
\addcontentsline{toc}{section}{recordAngles()}

\noindent
{\bf Synopsis}
\vspace{-8pt}
\begin{verbatim}
#include <mobot.h>
int CMobot::recordAngles(double time[], double angle1[], double angle2[], 
                         double angle3[], double angle4[], int num, 
                         double seconds, int shiftData = 1);
\end{verbatim}

\noindent
{\bf Purpose}\\
Record joint angle data for all joint for a set amount of time at a specified time interval.\\

\noindent
{\bf Return Value}\\
The function returns 0 on success and non-zero otherwise.\\

\noindent
{\bf Parameters}\\
\vspace{-0.1in}
\begin{description}
\item               
\begin{tabular}{p{15 mm}p{145 mm}}
\texttt{time} & An array which will store time values for each of the angle readings. \\
\texttt{angle1} & An array which will store the angle values for joint 1. \\
\texttt{angle2} & An array which will store the angle values for joint 2. \\
\texttt{angle3} & An array which will store the angle values for joint 3. \\
\texttt{angle4} & An array which will store the angle values for joint 4. \\
\texttt{num} & The number of elements of the arrays. \\
\texttt{seconds} & The number of seconds between angle readings. \\
\texttt{threshold} & (optional) This argument indicates whether or not to align the first
detected motion to the y-axis. 
\end{tabular}
\end{description}

\noindent
{\bf Description}\\
This function is used to accurately record the motion of a Mobot joint at a relatively fast
rate. The function will fill the \texttt{time}, \texttt{angle1},
\texttt{angle2}, \texttt{angle3}, and \texttt{angle4} arrays with data
at the rate specified by \texttt{seconds}. A typical value for \texttt{seconds} is 0.1, or
polling 10 times a second. If the communication speed cannot maintain 
the requested rate, (if \texttt{msecs} is too low, in other words), the function will
collect data as fast as possible. The lowest allowable rate is 0.05, however there is
no guarantee that data will actually be collected at that rate, due to communication noise,
distance to the module, etc.

The length of time to collect the data can be calculated by the formula \\
\begin{equation*}
\text{Total Time} = (\text{num} \times \text{seconds}) 
\end{equation*}

This function is a non-blocking function. After calling this function, a call to
\texttt{recordWait()} should be performed to ensure that the data has been fully collected.

\noindent
{\bf Example}\\
\noindent

\noindent
{\bf See Also}\\
\texttt{recordAngle(), recordWait()}\\
%\CPlot::\DataThreeD(), \CPlot::\DataFile(), \CPlot::\Plotting(), \plotxy().\\
