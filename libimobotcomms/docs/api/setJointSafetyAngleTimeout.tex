\noindent
\vspace{5pt}
\rule{4.5in}{0.015in}\\
\noindent
{\LARGE \texttt{CMobot::setJointSafetyAngleTimeout()}\index{CMobot::setJointSafetyAngleTimeout()}}\\
%\phantomsection
\addcontentsline{toc}{section}{setJointSafetyAngleTimeout()}

\noindent
{\bf Synopsis}
\vspace{-8pt}
\begin{verbatim}
#include <mobot.h>
int CMobot::setJointSafetyAngleTimeout(double seconds);
\end{verbatim}

\noindent
{\bf Purpose}\\
Set the current angle safety limit timeout of the Mobot.\\

\noindent
{\bf Return Value}\\
The function returns 0 on success and -1 on failure.\\

\noindent
{\bf Parameters}\\
A variable which will be overwritten with the safety angle limit timeout in seconds.\\

\noindent
{\bf Description}\\
The Mobot is equipped with a safety feature to protect itself and its surrounding
environment. When a motor deviates by a certain amount from its expected value, 
the Mobot will shut off all power to the motor after a certain period of time,
in case it has hit an obstacle, or for any other reason. The period of time that the
mobot waits before shutting the motor off is the joint safety angle timeout, which
can be set with this function. The default value for the timeout is 500 milliseconds,
or 0.5 seconds.
 
\noindent
{\bf Example}\\
\noindent

\noindent
{\bf See Also}\\
\texttt{getJointSafetyAngle(), setJointSafetyAngle(), getJointSafetyAngleTimeout()}\\

%\CPlot::\DataThreeD(), \CPlot::\DataFile(), \CPlot::\Plotting(), \plotxy().\\


