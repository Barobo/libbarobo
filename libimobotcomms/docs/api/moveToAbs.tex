\noindent
\vspace{5pt}
\rule{4.5in}{0.015in}\\
\noindent
{\LARGE \texttt{CMobot::moveToAbs()}\index{CMobot::moveToAbs()}}\\
{\LARGE \texttt{CMobot::moveToAbsNB()}\index{CMobot::moveToAbsNB()}}\\
%\phantomsection
\addcontentsline{toc}{section}{moveToAbs()}
\addcontentsline{toc}{section}{moveToAbsNB()}

\noindent
{\bf Synopsis}
\vspace{-8pt}
\begin{verbatim}
#include <mobot.h>
int CMobot::moveToAbs(double angle1, double angle2, double angle3, double angle4);
int CMobot::moveToAbsNB(double angle1, double angle2, double angle3, double angle4);
\end{verbatim}

\noindent
{\bf Purpose}\\
Move all of the joints of a mobot to the specified positions.\\

\noindent
{\bf Return Value}\\
The function returns 0 on success and non-zero otherwise.\\

\noindent
{\bf Parameters}\\
\vspace{-0.1in}
\begin{description}
\item               
\begin{tabular}{p{15 mm}p{105 mm}}
\texttt{angle1} & The absolute position to move joint 1, expressed in degrees. \\
\texttt{angle2} & The absolute position to move joint 2, expressed in degrees. \\
\texttt{angle3} & The absolute position to move joint 3, expressed in degrees. \\
\texttt{angle4} & The absolute position to move joint 4, expressed in degrees. \\
\end{tabular}
\end{description}
\noindent

{\bf Description}\\
\vspace{-12pt}
\begin{quote}
{\bf CMobot::moveToAbs()}\\
This function moves all of the joints of a mobot to the specified absolute positions. 

{\bf CMobot::moveToAbsNB()}\\
This function moves all of the joints of a mobot to the specified absolute positions. 

The function \texttt{moveToAbsNB()} is the non-blocking version of
the \texttt{moveToAbs()} function, which means that the function will return
immediately and the physical mobot motion will occur asynchronously. For
more details on blocking and non-blocking functions, please refer to 
Section \ref{sec:blocking} on page \pageref{sec:blocking}.\\
\end{quote}

The difference between this function and its sister function, \texttt{moveTo()},
is that this function takes into consideration any full rotations a joint has 
experienced. For instance, consider the following code segment:
\begin{verbatim}
mobot.moveTo(375, 0, 0, 0);
mobot.moveTo(0, 0, 0, 0);
\end{verbatim}
The first line in the above code segment will cause joint 1 to rotate one full
rotation in the positive direction, plus an additional 15 degrees. The second
line will cause the same joint to rotate backwards 15 degrees to reach its "0"
goal.

This following code segment, however, differs.
\begin{verbatim}
mobot.moveToAbs(375, 0, 0, 0);
mobot.moveToAbs(0, 0, 0, 0);
\end{verbatim}
The first line will have the same effect as the first code segment, rotating
joint 1 375 degrees, assuming it is starting at 0. The second line, however,
will cause joint 1 to rotate the full 375 degrees backwards back to its original 
location.

\noindent
{\bf Example}\\
Please see the demo at Section \ref{sec:democode} on page \pageref{sec:democode}.\\
\noindent

\noindent
{\bf See Also}\\

%\CPlot::\DataThreeD(), \CPlot::\DataFile(), \CPlot::\Plotting(), \plotxy().\\
