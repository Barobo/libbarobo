\noindent
\vspace{5pt}
\rule{4.5in}{0.015in}\\
\noindent
{\LARGE \texttt{CMobot::recordAngleBegin()}\index{CMobot::recordAngleBegin()}}\\
%\phantomsection
\addcontentsline{toc}{section}{recordAngleBegin()}

\noindent
{\bf Synopsis}
\vspace{-8pt}
\begin{verbatim}
#include <mobot.h>
int CMobot::recordAngleBegin(
    mobotJointId_t id, 
    mobotRecordData_t &time, 
    mobotRecordData_t &angle, 
    double seconds, 
    int shiftData = 1);
\end{verbatim}

\noindent
{\bf Purpose}\\
Record joint angle data for a joint for a set amount of time at a specified time interval.\\

\noindent
{\bf Return Value}\\
The function returns 0 on success and non-zero otherwise.\\

\noindent
{\bf Parameters}\\
\vspace{-0.1in}
\begin{description}
\item               
\begin{tabular}{p{15 mm}p{145 mm}}
\texttt{id} & The joint number. This is an enumerated type 
discussed in Section \ref{sec:mobotJointId_t} on page
\pageref{sec:mobotJointId_t}.\\
\texttt{time} & A variable which will store time values for each of the angle readings. \\
\texttt{angle} & A variable which will store the angle values for each time. \\
\texttt{seconds} & The number of seconds between angle readings. The minimum value allowed for
this variable is 0.05. \\
\texttt{shiftData} & (optional) This argument indicates whether or not to align the first
detected motion to the y-axis. 
\end{tabular}
\end{description}

\noindent
{\bf Description}\\
This function is used in conjunction with the \texttt{recordAngleEnd()}
function to accurately record the motion of a Mobot joint at a relatively fast
rate. 

The function will allocate memory and fill the \texttt{time} and \texttt{angle} variables with data
at the rate specified by \texttt{seconds}. 

If the communication speed cannot maintain 
the requested rate, (if \texttt{seconds} is too low, in other words), the function will
collect data as fast as possible. The minimum value for \texttt{seconds} is 0.05, but
the actual minimum time will depend on other factors, such as communication noise and
distance to the mobot.

When the \texttt{recordAngleBegin()} function is called, the Mobot library will 
immediately begin recording angle data. The \texttt{recordAngleEnd()} function
must be called to halt the recording process. 

\noindent
{\bf Example}\\
\noindent

\noindent
{\bf See Also}\\
\texttt{recordAngleEnd()} \\
%\CPlot::\DataThreeD(), \CPlot::\DataFile(), \CPlot::\Plotting(), \plotxy().\\
