\noindent
\vspace{5pt}
\rule{4.5in}{0.015in}\\
\noindent
{\LARGE \texttt{CMobot::recordAngle()}\index{CMobot::recordAngle()}}\\
%\phantomsection
\addcontentsline{toc}{section}{recordAngle()}

\noindent
{\bf Synopsis}
\vspace{-8pt}
\begin{verbatim}
#include <mobot.h>
int CMobot::recordAngle(
    mobotJointId_t id, 
    double time[], 
    double angle[], 
    int num, 
    double seconds, 
    double threshold = 0.0);
\end{verbatim}

\noindent
{\bf Purpose}\\
Record joint angle data for a joint for a set amount of time at a specified time interval.\\

\noindent
{\bf Return Value}\\
The function returns 0 on success and non-zero otherwise.\\

\noindent
{\bf Parameters}\\
\vspace{-0.1in}
\begin{description}
\item               
\begin{tabular}{p{15 mm}p{145 mm}}
\texttt{id} & The joint number. This is an enumerated type 
discussed in Section \ref{sec:mobotJointId_t} on page
\pageref{sec:mobotJointId_t}.\\
\texttt{time} & An array which will store time values for each of the angle readings. \\
\texttt{angle} & An array which will store the angle values for each time. \\
\texttt{num} & The size of the arrays. \\
\texttt{seconds} & The number of seconds between angle readings. The minimum value allowed for
this variable is 0.05. \\
\texttt{threshold} & (optional) This argument is used to align the first
detected motion to the y-axis. When the angle of any joint moves by
\texttt{threshold} degrees, that point is aligned with the y-axis.
\end{tabular}
\end{description}

\noindent
{\bf Description}\\
This function is used to accurately record the motion of a Mobot joint at a relatively fast
rate. The function will fill the \texttt{time} and \texttt{angle} arrays with data
at the rate specified by \texttt{seconds}. If the communication speed cannot maintain 
the requested rate, (if \texttt{msecs} is too low, in other words), the function will
collect data as fast as possible. The minimum value for \texttt{seconds} is 0.05, but
the actual minimum time will depend on other factors, such as communication noise and
distance to the mobot.

The length of time to collect the data can be calculated by the formula \\
\begin{equation*}
\text{Total Time} = (\text{num} \times \text{seconds}) 
\end{equation*}

This function is a non-blocking function. After calling this function, a call to
\texttt{recordWait()} should be performed to ensure that the data has been fully collected.

\noindent
{\bf Example}\\
\noindent

\noindent
{\bf See Also}\\
\texttt{recordAngles(), recordWait()} \\
%\CPlot::\DataThreeD(), \CPlot::\DataFile(), \CPlot::\Plotting(), \plotxy().\\
