\noindent
\vspace{5pt}
\rule{4.5in}{0.015in}\\
\noindent
{\LARGE \texttt{CMobot::setMovementStateTime()}\index{CMobot::setMovementStateTime()}}\\
{\LARGE \texttt{CMobot::setMovementStateTimeNB()}\index{CMobot::setMovementStateTimeNB()}}\\
%\phantomsection
\addcontentsline{toc}{section}{setMovementStateTime()}
\addcontentsline{toc}{section}{setMovementStateTimeNB()}

\noindent
{\bf Synopsis}
\vspace{-8pt}
\begin{verbatim}
#include <mobot.h>
int CMobot::setMovementStateTime( mobotJointState_t dir1, 
                                mobotJointState_t dir2, 
                                mobotJointState_t dir3, 
                                mobotJointState_t dir4, 
                                double seconds);
int CMobot::setMovementStateTimeNB( mobotJointState_t dir1, 
                                mobotJointState_t dir2, 
                                mobotJointState_t dir3, 
                                mobotJointState_t dir4, 
                                double seconds);
\end{verbatim}

\noindent
{\bf Purpose}\\
Move the joints of a mobot continuously in the specified directions for some amount of time.\\

\noindent
{\bf Return Value}\\
The function returns 0 on success and non-zero otherwise.\\

\noindent
{\bf Parameters}\\
Each direction parameter specifies the direction the joint should move. The types
are enumerated in \texttt{mobot.h} and have the following values:
\\
\noindent
\begin{tabular}{p{1.75in}p{4.5in}} \hline 
Value & Description \\
\hline \\
\texttt{MOBOT\_NEUTRAL}& This value indicates that the joint is not moving and is not actuated. The joint is freely backdrivable. \\
\texttt{MOBOT\_FORWARD}& This value indicates that the joint is currently moving forward. \\
\texttt{MOBOT\_BACKWARD}& This value indicates that the joint is currently moving backward. \\
\texttt{MOBOT\_HOLD}& This value indicates that the joint is currently not moving and is holding its current position. The joint is not currently backdrivable. \\
\hline
\end{tabular}\\


The \texttt{seconds} parameter is the time to perform the movement, in seconds.
\\

\noindent
{\bf Description}\\
This function causes joints of a mobot to begin moving. The joints will continue moving
until the joint hits a joint limit, or the time specified in the \texttt{seconds} parameter
is reached. The \texttt{setMovementStateTime()} function will block until the motion is completed, 
whereas the \texttt{setMovementStateTimeNB()} function will not block. 

\noindent
{\bf Example}\\
\noindent

\noindent
{\bf See Also}\\

%\CPlot::\DataThreeD(), \CPlot::\DataFile(), \CPlot::\Plotting(), \plotxy().\\
