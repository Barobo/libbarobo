\noindent
\vspace{5pt}
\rule{4.5in}{0.015in}\\
\noindent
{\LARGE \texttt{CMobot::recordAngleEnd()}\index{CMobot::recordAngleEnd()}}\\
%\phantomsection
\addcontentsline{toc}{section}{recordAngleEnd()}

\noindent
{\bf Synopsis}
\vspace{-8pt}
\begin{verbatim}
#include <mobot.h>
int CMobot::recordAngleEnd(mobotJointId_t id);
\end{verbatim}

\noindent
{\bf Purpose}\\
End a joint recording process that is currently running.\\

\noindent
{\bf Return Value}\\
This function returns the number of datapoints captured during the recording procedure.\\

\noindent
{\bf Parameters}\\
The joint to end the recording on. \\

\noindent
{\bf Description}\\
This function is used in conjunction with the \texttt{recordAngleBegin()} function. 
This function stops the recording process and returns the number of valid data points
allocated for the arrays.\\

\noindent
{\bf Example}\\
\noindent

\noindent
{\bf See Also}\\
\texttt{recordAngleBegin()} \\
%\CPlot::\DataThreeD(), \CPlot::\DataFile(), \CPlot::\Plotting(), \plotxy().\\
