\noindent
\vspace{5pt}
\rule{4.5in}{0.015in}\\
\noindent
{\LARGE \texttt{CMobot::stopAllJoints()}\index{CMobot::stopAllJoints()}}\\
{\LARGE \texttt{CMobot::stopOneJoint()}\index{CMobot::stopOneJoint()}}\\
{\LARGE \texttt{CMobot::stopTwoJoints()}\index{CMobot::stopTwoJoints()}}\\
{\LARGE \texttt{CMobot::stopThreeJoints()}\index{CMobot::stopThreeJoints()}}\\
%\phantomsection
\addcontentsline{toc}{section}{stopAllJoints()}
\addcontentsline{toc}{section}{stopOneJoint()}
\addcontentsline{toc}{section}{stopTwoJoints()}
\addcontentsline{toc}{section}{stopThreeJoints()}

\noindent
{\bf Synopsis}
\vspace{-8pt}
\begin{verbatim}
#include <mobot.h>
int CMobot::stopAllJoints();
int CMobot::stopOneJoint(mobotJointId_t id);
int CMobot::stopTwoJoints(mobotJointId_t id1, mobotJointId_t id2);
int CMobot::stopThreeJoints(
    mobotJointId_t id1,
    mobotJointId_t id2,
    mobotJointId_t id3);
\end{verbatim}

\noindent
{\bf Purpose}\\
These functions stop joints on a mobot.\\

\noindent
{\bf Return Value}\\
The function returns 0 on success and non-zero otherwise.\\

\noindent
{\bf Description}\\
These functions are used to stop joints on a mobot. A stopped joint
will immediately cease any and all actuation and go limp.

\noindent
{\bf Example}\\
\noindent

\noindent
{\bf See Also}\\
\texttt{setJointSpeed(), setJointSpeeds()}

%\CPlot::\DataThreeD(), \CPlot::\DataFile(), \CPlot::\Plotting(), \plotxy().\\
