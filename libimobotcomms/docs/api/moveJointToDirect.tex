\noindent
\vspace{5pt}
\rule{4.5in}{0.015in}\\
\noindent
{\LARGE \texttt{CMobot::moveJointToDirect()}\index{CMobot::moveJointToDirect()}}\\
{\LARGE \texttt{CMobot::moveJointToDirectNB()}\index{CMobot::moveJointToDirectNB()}}\\
%\phantomsection
\addcontentsline{toc}{section}{moveJointToDirect()}
\addcontentsline{toc}{section}{moveJointToDirectNB()}

\noindent
{\bf Synopsis}
\vspace{-8pt}
\begin{verbatim}
#include <mobot.h>
int CMobot::moveJointToDirect(mobotJointId_t id, double angle);
int CMobot::moveJointToDirectNB(mobotJointId_t id, double angle);
\end{verbatim}

\noindent
{\bf Purpose}\\
Move a joint on the mobot to an absolute position taking the shortest direction possible.\\

\noindent
{\bf Return Value}\\
The function returns 0 on success and non-zero otherwise.\\

\noindent
{\bf Parameters}\\
\vspace{-0.1in}
\begin{description}
\item               
\begin{tabular}{p{10 mm}p{145 mm}}
\texttt{id} & The joint number to wait for. \\
\texttt{angle} & The absolute angle in degrees to move the motor to.  \\
\end{tabular}
\end{description}

\noindent
{\bf Description}\\
\vspace{-12pt}

\begin{quote}
{\bf CMobot::moveJointToDirect()}\\
This function commands the motor to move to a position specified in degrees at
the current motor's speed. The current motor speed may be set with the
\texttt{setJointSpeed()} member function. Please note that if the motor speed
is set to zero, the motor will not move after calling the
\texttt{moveJointToDirect()} function. 

{\bf CMobot::moveJointToDirectNB()}\\
This function commands the motor to move to a position specified in degrees at
the current motor's speed. The current motor speed may be set with the
\texttt{setJointSpeed()} member function. Please note that if the motor speed
is set to zero, the motor will not move after calling the
\texttt{moveJointToDirectNB()} function. 

The function \texttt{moveJointToDirectNB()} is the non-blocking version of
the \texttt{moveJointToDirect()} function, which means that the function will return
immediately and the physical mobot motion will occur asynchronously. For
more details on blocking and non-blocking functions, please refer to 
Section \ref{sec:blocking} on page \pageref{sec:blocking}.\\
\end{quote}

Note that the main difference between this function and its sister function,
\texttt{moveJointTo()}, is that this function explicitely moves the motor in
whichever direction is the shortest to achieve its goal.

\noindent
{\bf Example}\\
Please see the example in Section \ref{sec:democode} on page \pageref{sec:democode}.\\
\noindent

\noindent
{\bf See Also}\\
\texttt{connectWithBluetoothAddress()}

%\CPlot::\DataThreeD(), \CPlot::\DataFile(), \CPlot::\Plotting(), \plotxy().\\
