\noindent
\vspace{5pt}
\rule{4.5in}{0.015in}\\
\noindent
{\LARGE \texttt{CMobot::getJointAngle()}\index{getJointAngle()}}\\
%\phantomsection
\addcontentsline{toc}{section}{getJointAngle()}

\noindent
{\bf Synopsis}\\
\begin{verbatim}
#include <mobot.h>
int CMobot::getJointAngle(robotJointId_t id, double &position);
\end{verbatim}

\noindent
{\bf Purpose}\\
Retrieve a robot joint's current angle.\\

\noindent
{\bf Return Value}\\
The function returns 0 on success and non-zero otherwise.\\

\noindent
{\bf Parameters}\\
\vspace{-0.1in}
\begin{description}
\item               
\begin{tabular}{p{15 mm}p{105 mm}}
\texttt{id} & The joint number. This is an enumerated type 
discussed in Section \ref{sec:robotJointId_t} on page
\pageref{sec:robotJointId_t}.\\
\texttt{position} & A variable to store the current position of the robot
motor. The contents of this variable will be overwritten with a value that
represents the motor's angle in radians.  \\
\end{tabular}
\end{description}

\noindent
{\bf Description}\\
This function gets the current motor position of a robot's motor. The
position returned is in units of degrees and is accurate to roughly $\pm0.1$
degrees. \\

\noindent
{\bf Example}\\
\noindent

\noindent
{\bf See Also}\\

%\CPlot::\DataThreeD(), \CPlot::\DataFile(), \CPlot::\Plotting(), \plotxy().\\
