\noindent
\vspace{5pt}
\rule{4.5in}{0.015in}\\
\noindent
{\LARGE \texttt{CMobot::getJointState()}\index{CMobot::getJointState()}}\\
%\phantomsection
\addcontentsline{toc}{section}{getJointState()}

\noindent
{\bf Synopsis}
\vspace{-8pt}
\begin{verbatim}
#include <mobot.h>
int CMobot::getJointState(mobotJointId_t id, mobotJointState_t &state);
\end{verbatim}

\noindent
{\bf Purpose}\\
Determine whether a motor is moving or not.\\

\noindent
{\bf Return Value}\\
The function returns 0 on success and non-zero otherwise.\\

\noindent
{\bf Parameters}
\vspace{-0.1in}
\begin{description}
\item               
\begin{tabular}{p{10 mm}p{145 mm}}
\texttt{id} & The joint number. This is an enumerated type 
discussed in Section \ref{sec:mobotJointId_t} on page
\pageref{sec:mobotJointId_t}.\\
\texttt{state} & An integer variable which will be overwritten with the current state of the motor. 
This is an enumerated type 
discussed in Section \ref{sec:mobotJointState_t} on page
\pageref{sec:mobotJointState_t}.
\end{tabular}
\end{description}

\noindent
{\bf Description}\\
This function is used to determine the current state of a motor. Valid states are listed below.
\\
\noindent
\begin{tabular}{p{1.75in}p{4.5in}} \hline 
Value & Description \\
\hline \\
\texttt{MOBOT\_NEUTRAL}& This value indicates that the joint is not moving and is not actuated. The joint is freely backdrivable. \\
\texttt{MOBOT\_FORWARD}& This value indicates that the joint is currently moving forward. \\
\texttt{MOBOT\_BACKWARD}& This value indicates that the joint is currently moving backward. \\
\texttt{MOBOT\_HOLD}& This value indicates that the joint is currently not moving and is holding its current position. The joint is not currently backdrivable. \\
\hline
\end{tabular}\\



\noindent
{\bf Example}\\
\noindent

\noindent
{\bf See Also}\\
\texttt{isMoving()}\\
%\CPlot::\DataThreeD(), \CPlot::\DataFile(), \CPlot::\Plotting(), \plotxy().\\
