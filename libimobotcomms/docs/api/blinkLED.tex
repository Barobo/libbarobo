\noindent
\vspace{5pt}
\rule{4.5in}{0.015in}\\
\noindent
{\LARGE \texttt{CMobot::blinkLED()}\index{CMobot::blinkLED()}}\\
%\phantomsection
\addcontentsline{toc}{section}{blinkLED()}

\noindent
{\bf Synopsis}
\vspace{-8pt}
\begin{verbatim}
#include <mobot.h>
int CMobot::blinkLED(double delay, int numBlinks);
\end{verbatim}

\noindent
{\bf Purpose}\\
Blink the on-board LED on a Mobot module.\\

\noindent
{\bf Return Value}\\
The function returns 0 on success and non-zero otherwise.\\

\noindent
{\bf Parameters}\\
\vspace{-0.1in}
\begin{description}
\item               
\begin{tabular}{p{15 mm}p{125 mm}}
\texttt{delay} & The amount of time between blinks. \\
\texttt{numBlinks} & The number of times to blink the LED. \\
\end{tabular}
\end{description}
\noindent
{\bf Description}\\
This function is used to blink or flash the LED on a Mobot module. The first
argument, \texttt{delay}, is used to control the speed of the blinking
LED, and the second argument, \texttt{numBlinks}, controls the number of times
to blink. For instance, the line
\begin{verbatim}
mobot.blinkLED(0.1, 10);
\end{verbatim}
would cause a mobot to blink 10 times fast, and the line
\begin{verbatim}
mobot.blinkLED(1, 2);
\end{verbatim}
would cause the mobot to do two slow blinks.

\noindent
{\bf Example}\\
\noindent

\noindent
{\bf See Also}\\

%\CPlot::\DataThreeD(), \CPlot::\DataFile(), \CPlot::\Plotting(), \plotxy().\\
