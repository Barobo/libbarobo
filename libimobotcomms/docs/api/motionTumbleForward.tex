\noindent
\vspace{5pt}
\rule{4.5in}{0.015in}\\
\noindent
{\LARGE \texttt{CMobot::motionTumbleForward()}\index{CMobot::motionTumbleForward()}}\\
{\LARGE \texttt{CMobot::motionTumbleForwardNB()}\index{CMobot::motionTumbleForwardNB()}}\\
%\phantomsection
\addcontentsline{toc}{section}{motionTumbleForward()}
\addcontentsline{toc}{section}{motionTumbleForwardNB()}

\noindent
{\bf Synopsis}
\vspace{-8pt}
\begin{verbatim}
#include <mobot.h>
int CMobot::motionTumbleForward(int num);
int CMobot::motionTumbleForwardNB(int num);
\end{verbatim}

\noindent
{\bf Purpose}\\
Make the robot tumble end over end.\\

\noindent
{\bf Return Value}\\
The function returns 0 on success and non-zero otherwise.\\

\noindent
{\bf Parameters}\\
\vspace{-0.1in}
\begin{description}
\item               
\begin{tabular}{p{10 mm}p{145 mm}}
\texttt{num} & The number of times to tumble. \\
\end{tabular}
\end{description}

\noindent
{\bf Description}\\
\vspace{-12pt}
\begin{quote}
{\bf CMobot::motionTumbleForward()}\\
This causes the robot to tumble end over end. The argument, \texttt{num},
indicates the number of times to tumble.

{\bf CMobot::motionTumbleForwardNB()}\\
This causes the robot to tumble end over end. The argument, \texttt{num},
indicates the number of times to tumble.

This function has both a blocking and non-blocking version.
The blocking version, \texttt{motionTumbleForward()}, will block until the
robot motion has completed. The non-blocking version, \texttt{motionTumbleForwardNB()},
will return immediately, and the motion will be performed asynchronously.\\
\end{quote}

\noindent
{\bf See Also}\\

%\CPlot::\DataThreeD(), \CPlot::\DataFile(), \CPlot::\Plotting(), \plotxy().\\
