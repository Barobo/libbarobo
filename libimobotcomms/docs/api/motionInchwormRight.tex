\noindent
\vspace{5pt}
\rule{4.5in}{0.015in}\\
\noindent
{\LARGE \texttt{CMobot::motionInchwormRight()}\index{CMobot::motionInchwormRight()}}\\
{\LARGE \texttt{CMobot::motionInchwormRightNB()}\index{CMobot::motionInchwormRightNB()}}\\
%\phantomsection
\addcontentsline{toc}{section}{motionInchwormRight()}
\addcontentsline{toc}{section}{motionInchwormRightNB()}

\noindent
{\bf Synopsis}
\begin{verbatim}
#include <mobot.h>
int CMobot::motionInchwormRight();
int CMobot::motionInchwormRightNB();
\end{verbatim}

\noindent
{\bf Purpose}\\
Perform the inch-worm gait to the right.\\

\noindent
{\bf Return Value}\\
The function returns 0 on success and non-zero otherwise.\\

\noindent
{\bf Parameters}\\
None.\\

\noindent
{\bf Description}\\
This function causes the robot to perform a single cycle of the inchworm gait
to the right. 

This function has both a blocking and non-blocking version.
The blocking version, \texttt{motionInchwormRight()}, will block until the
robot motion has completed. The non-blocking version, \texttt{motionInchwormRightNB()},
will return immediately, and the motion will be performed asynchronously.\\

\noindent
{\bf See Also}\\
\texttt{motionInchwormLeft()}

%\CPlot::\DataThreeD(), \CPlot::\DataFile(), \CPlot::\Plotting(), \plotxy().\\
