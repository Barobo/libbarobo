\noindent
\vspace{5pt}
\rule{4.5in}{0.015in}\\
\noindent
{\LARGE \texttt{CMobot::moveJointToAbs()}\index{CMobot::moveJointToAbs()}}\\
{\LARGE \texttt{CMobot::moveJointToAbsNB()}\index{CMobot::moveJointToAbsNB()}}\\
%\phantomsection
\addcontentsline{toc}{section}{moveJointToAbs()}
\addcontentsline{toc}{section}{moveJointToAbsNB()}

\noindent
{\bf Synopsis}
\vspace{-8pt}
\begin{verbatim}
#include <mobot.h>
int CMobot::moveJointToAbs(mobotJointId_t id, double angle);
int CMobot::moveJointToAbsNB(mobotJointId_t id, double angle);
\end{verbatim}

\noindent
{\bf Purpose}\\
Move a joint on the robot to an absolute position.\\

\noindent
{\bf Return Value}\\
The function returns 0 on success and non-zero otherwise.\\

\noindent
{\bf Parameters}\\
\vspace{-0.1in}
\begin{description}
\item               
\begin{tabular}{p{10 mm}p{145 mm}}
\texttt{id} & The joint number to wait for. \\
\texttt{angle} & The absolute angle in degrees to move the motor to.  \\
\end{tabular}
\end{description}

\noindent
{\bf Description}\\
\vspace{-12pt}
\begin{quote}
{\bf CMobot::moveJointToAbs()}\\
This function commands the motor to move to a position specified in degrees at
the current motor's speed. The current motor speed may be set with the
\texttt{setJointSpeed()} member function. Please note that if the motor speed
is set to zero, the motor will not move after calling the
\texttt{moveJointToAbs()} function. 

{\bf CMobot::moveJointToAbsNB()}\\
This function commands the motor to move to a position specified in degrees at
the current motor's speed. The current motor speed may be set with the
\texttt{setJointSpeed()} member function. Please note that if the motor speed
is set to zero, the motor will not move after calling the
\texttt{moveJointToAbsNB()} function. 

The function \texttt{moveJointToAbsNB()} is the non-blocking version of
the \texttt{moveJointToAbs()} function, which means that the function will return
immediately and the physical robot motion will occur asynchronously. For
more details on blocking and non-blocking functions, please refer to 
Section \ref{sec:blocking} on page \pageref{sec:blocking}.\\
\end{quote}

The difference between this function and its sister function, \texttt{moveJointTo()},
is that this function takes into consideration any full rotations a joint has 
experienced. For instance, consider the following code segment:
\begin{verbatim}
robot.moveJointTo(MOBOT_JOINT1, 375);
robot.moveJointTo(MOBOT_JOINT1, 0);
\end{verbatim}
The first line in the above code segment will cause joint 1 to rotate one full
rotation in the positive direction, plus an additional 15 degrees. The second
line will cause the same joint to rotate backwards 15 degrees to reach its "0"
goal.

This following code segment, however, differs.
\begin{verbatim}
robot.moveJointToAbs(MOBOT_JOINT1, 375);
robot.moveJointToAbs(MOBOT_JOINT1, 0);
\end{verbatim}
The first line will have the same effect as the first code segment, rotating
joint 1 375 degrees, assuming it is starting at 0. The second line, however,
will cause joint 1 to rotate the full 375 degrees backwards back to its original 
location.

\noindent
{\bf Example}\\
Please see the example in Section \ref{sec:democode} on page \pageref{sec:democode}.\\
\noindent

\noindent
{\bf See Also}\\
\texttt{connectWithAddress()}

%\CPlot::\DataThreeD(), \CPlot::\DataFile(), \CPlot::\Plotting(), \plotxy().\\
