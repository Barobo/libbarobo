\noindent
\vspace{5pt}
\rule{4.5in}{0.015in} \\
\noindent
{\LARGE \texttt{CMobot::connectWithIPAddress()}\index{CMobot::connectWithIPAddress()}}\\
%\phantomsection
\addcontentsline{toc}{section}{connectWithIPAddress()}

\noindent
{\bf Synopsis}
\vspace{-8pt}
\begin{verbatim}
#include <mobot.h>
int CMobot::connectWithIPAddress(char address[], char port[] = "5768");
\end{verbatim}

\noindent
{\bf Purpose}\\
Connect to a remote mobot connected to a remote computer over the internet.\\

\noindent
{\bf Return Value}\\
The function returns 0 on success and non-zero otherwise.\\

\noindent
{\bf Parameters}
\vspace{-0.1in}
\begin{description}
\item               
\begin{tabular}{p{10 mm}p{145 mm}}
\texttt{address} & The IP address of the remote computer. \\
\texttt{port} & (optional) The port to connect to. \\
\end{tabular}
\end{description}

\noindent
{\bf Description}\\
This function is used to connect to a mobot. 

\noindent
{\bf Example}\\
\begin{verbatim}
mobot.connectWithIPAddress("192.168.0.132", "8734");
mobot.connectWithIPAddress("mobots.example.com", "5768");
mobot.connectWithIPAddress("192.168.100.221");
\end{verbatim}
\noindent

\noindent
{\bf See Also}\\
\texttt{connect(), disconnect()}

%\CPlot::\DataThreeD(), \CPlot::\DataFile(), \CPlot::\Plotting(), \plotxy().\\
