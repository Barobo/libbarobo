\noindent
\vspace{5pt}
\rule{4.5in}{0.015in}\\
\noindent
{\LARGE \texttt{CMobot::motionRollBackward()}\index{motionRollBackward()}}\\
{\LARGE \texttt{CMobot::motionRollBackwardNB()}\index{motionRollBackwardNB()}}\\
%\phantomsection
\addcontentsline{toc}{section}{motionRollBackward()}
\addcontentsline{toc}{section}{motionRollBackwardNB()}

\noindent
{\bf Synopsis}\\
\begin{verbatim}
#include <mobot.h>
int CMobot::motionRollBackward();
int CMobot::motionRollBackwardNB();
\end{verbatim}

\noindent
{\bf Purpose}\\
Use the faceplates as wheels to roll backward.\\

\noindent
{\bf Return Value}\\
The function returns 0 on success and non-zero otherwise.\\

\noindent
{\bf Parameters}\\
None.\\

\noindent
{\bf Description}\\
This function causes each of the faceplates to rotate 90 degrees to roll the
robot backward.

This function has both a blocking and non-blocking version.
The blocking version, \texttt{motionRollBackward()}, will block until the
MoBot motion has completed. The non-blocking version, \texttt{motionRollBackwardNB()},
will return immediately, and the motion will be performed asynchronously.\\
\\

\noindent
{\bf See Also}\\
\texttt{motionRollForward()}

%\CPlot::\DataThreeD(), \CPlot::\DataFile(), \CPlot::\Plotting(), \plotxy().\\
