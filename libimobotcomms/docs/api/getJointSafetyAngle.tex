\noindent
\vspace{5pt}
\rule{4.5in}{0.015in}\\
\noindent
{\LARGE \texttt{CMobot::getJointSafetyAngle()}\index{CMobot::getJointSafetyAngle()}}\\
%\phantomsection
\addcontentsline{toc}{section}{getJointSafetyAngle()}

\noindent
{\bf Synopsis}
\vspace{-8pt}
\begin{verbatim}
#include <mobot.h>
int CMobot::getJointSafetyAngle(double &degrees);
\end{verbatim}

\noindent
{\bf Purpose}\\
Get the current angle safety limit of the Mobot.\\

\noindent
{\bf Return Value}\\
The function returns 0 on success and -1 on failure.\\

\noindent
{\bf Parameters}\\
A variable which will be overwritten with the safety angle limit in degrees.\\

\noindent
{\bf Description}\\
The Mobot is equipped with a safety feature to protect itself and its surrounding
environment. When a motor deviates by a certain amount from its expected value, 
the Mobot will shut off all power to the motor, in case it has hit an obstacle,
or for any other reason. The amount of deviation required to trigger the safety
protocol is the joint safety angle which can be retrieved using this function.
The default setting is 10 degrees.

\noindent
{\bf Example}\\
\noindent

\noindent
{\bf See Also}\\
\texttt{getJointSafetyAngleTimeout(), setJointSafetyAngle(), setJointSafetyAngleTimeout()}\\

%\CPlot::\DataThreeD(), \CPlot::\DataFile(), \CPlot::\Plotting(), \plotxy().\\
