\noindent
\vspace{5pt}
\rule{4.5in}{0.015in}\\
\noindent
{\LARGE \texttt{CiMobotComms::setMotorPosition()}\index{setMotorPosition()}}\\
%\phantomsection
\addcontentsline{toc}{section}{setMotorPosition()}

\noindent
{\bf Synopsis}\\
\begin{verbatim}
#include <imobot.h>
int CiMobotComms::setMotorPosition(int id, double position);
\end{verbatim}

\noindent
{\bf Purpose}\\
Connect to a remote iMobot via Bluetooth.\\

\noindent
{\bf Return Value}\\
The function returns 0 on success and non-zero otherwise.\\

\noindent
{\bf Parameters}\\
\vspace{-0.1in}
\begin{description}
\item               
\begin{tabular}{p{10 mm}p{145 mm}}
\texttt{id} & The joint number to wait for. \\
\texttt{position} & The absolute angle to move the motor to.  \\
\end{tabular}
\end{description}

\noindent
{\bf Description}\\
This function commands the motor to move to a position specified in degrees at
the current motor's speed. The current motor speed may be set with the
\texttt{setMotorSpeed()} member function. Please note that if the motor speed
is set to zero, the motor will not move after calling the
\texttt{setMotorPosition()} function. \\

\noindent
{\bf Example}\\
Please see the example in Section \ref{sec:democode} on page \pageref{sec:democode}.\\
\noindent

\noindent
{\bf See Also}\\
\texttt{connectAddress()}

%\CPlot::\DataThreeD(), \CPlot::\DataFile(), \CPlot::\Plotting(), \plotxy().\\
