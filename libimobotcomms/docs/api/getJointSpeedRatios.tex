\noindent
\vspace{5pt}
\rule{4.5in}{0.015in}\\
\noindent
{\LARGE \texttt{CMobot::getJointSpeedRatios()}\index{CMobot::getJointSpeedRatios()}}\\
%\phantomsection
\addcontentsline{toc}{section}{getJointSpeedRatios()}

\noindent
{\bf Synopsis}
\vspace{-8pt}
\begin{verbatim}
#include <mobot.h>
int CMobot::getJointSpeedRatios(double &ratio1, double &ratio2, double &ratio3, double &ratio4);
\end{verbatim}

\noindent
{\bf Purpose}\\
Get the speed ratio settings of all joints on the robot.\\

\noindent
{\bf Return Value}\\
The function returns 0 on success and non-zero otherwise.\\

\noindent
{\bf Parameters}
\vspace{-0.1in}
\begin{description}
\item               
\begin{tabular}{p{10 mm}p{145 mm}}
\texttt{ratio1} & A variable to store the speed ratio of joint 1.\\
\texttt{ratio2} & A variable to store the speed ratio of joint 2.\\
\texttt{ratio3} & A variable to store the speed ratio of joint 3.\\
\texttt{ratio4} & A variable to store the speed ratio of joint 4.\\
\end{tabular}
\end{description}

\noindent
{\bf Description}\\
This function is used to retrieve all four joint speed ratio settings of a robot
simultaneously. The speed ratios are as a value from 0 to 1. \\

\noindent
{\bf Example}\\
\noindent

\noindent
{\bf See Also}\\
\texttt{setJointSpeeds(), getJointSpeedRatios(), getJointSpeed()}

%\CPlot::\DataThreeD(), \CPlot::\DataFile(), \CPlot::\Plotting(), \plotxy().\\
