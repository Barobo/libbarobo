\noindent
\vspace{5pt}
\rule{4.5in}{0.015in}\\
\noindent
{\LARGE \texttt{CMobot::setJointSpeed()}\index{CMobot::setJointSpeed()}}\\
%\phantomsection
\addcontentsline{toc}{section}{setJointSpeed()}

\noindent
{\bf Synopsis}
\vspace{-8pt}
\begin{verbatim}
#include <mobot.h>
int CMobot::setJointSpeed(robotJointId_t id, double speed);
\end{verbatim}

\noindent
{\bf Purpose}\\
Set the speed of a joint on the robot.\\

\noindent
{\bf Return Value}\\
The function returns 0 on success and non-zero otherwise.\\

\noindent
{\bf Parameters}
\vspace{-0.1in}
\begin{description}
\item               
\begin{tabular}{p{10 mm}p{145 mm}}
\texttt{id} & The joint number to pose. \\
\texttt{speed} & An variable of type \texttt{double} for the requested average
angular speed in radians per second.
\end{tabular}
\end{description}

\noindent
{\bf Description}\\
This function is used to set the angular speed of a joint of a robot. The
maximum possible angular speed for a particular joint may be obtained
by using the function \texttt{getJointMaxSpeed()}.
\noindent\\
{\bf Example}\\
\noindent

\noindent
{\bf See Also}\\

%\CPlot::\DataThreeD(), \CPlot::\DataFile(), \CPlot::\Plotting(), \plotxy().\\
