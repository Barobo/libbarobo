\noindent
\vspace{5pt}
\rule{4.5in}{0.015in}\\
\noindent
{\LARGE \texttt{CMobot::setTwoWheelRobotSpeed()}\index{CMobot::setTwoWheelRobotSpeed()}}\\
%\phantomsection
\addcontentsline{toc}{section}{setTwoWheelRobotSpeed()}

\noindent
{\bf Synopsis}\\
\begin{verbatim}
#include <mobot.h>
int CMobot::setTwoWheelRobotSpeed(double speed, double radius, char unit[]);
\end{verbatim}

\noindent
{\bf Purpose}\\
Roll the robot at a certain speed in a straight line.\\

\noindent
{\bf Return Value}\\
The function returns 0 on success and non-zero otherwise.\\

\noindent
{\bf Parameters}
\vspace{-0.1in}
\begin{description}
\item               
\begin{tabular}{p{10 mm}p{145 mm}}
\texttt{speed} & The speed at which to roll the robot. The units used will be the units
specified in the \texttt{unit} parameter. \\
\texttt{radius} & The radius of the wheels attached to the robot. The units of the parameter
should match the units provided in the \texttt{unit} parameter. \\
\texttt{unit} & Specify the units used for the \texttt{speed} and \texttt{radius} parameters.

\begin{tabular}{lll}
speed & radius & unit \\
\hline \\
cm/s & cm & \texttt{"cm"} \\
m/s & m & \texttt{"m"} \\
inch/s & inch & \texttt{"inch"} \\
foot/s & foot & \texttt{"foot"}
\end{tabular}
\end{tabular}
\end{description}

\noindent
{\bf Description}\\
This function is used to make a two wheeled robot roll at a certain speed. The desired 
speed and radius of the wheels is provided and the function will rotate the wheels at the
appropriate rate in order to achieve the desired speed.
\noindent\\
{\bf Example}\\
\noindent

\noindent
{\bf See Also}\\

%\CPlot::\DataThreeD(), \CPlot::\DataFile(), \CPlot::\Plotting(), \plotxy().\\
