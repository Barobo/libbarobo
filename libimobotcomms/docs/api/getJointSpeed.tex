\noindent
\vspace{5pt}
\rule{4.5in}{0.015in}\\
\noindent
{\LARGE \texttt{CMobot::getJointSpeed()}\index{getJointSpeed()}}\\
%\phantomsection
\addcontentsline{toc}{section}{getJointSpeed()}

\noindent
{\bf Synopsis}\\
\begin{verbatim}
#include <imobot.h>
int CMobot::getJointSpeed(int id, double &speed);
\end{verbatim}

\noindent
{\bf Purpose}\\
Get the speed of a joint on the iMobot.\\

\noindent
{\bf Return Value}\\
The function returns 0 on success and non-zero otherwise.\\

\noindent
{\bf Parameters}
\vspace{-0.1in}
\begin{description}
\item               
\begin{tabular}{p{10 mm}p{145 mm}}
\texttt{id} & The joint number to pose. \\
\texttt{speed} & A variable of type \texttt{double}. The value of this variable
will be overwritten with the current speed setting of the joint, which is a
value between 0 and 1.
\end{tabular}
\end{description}

\noindent
{\bf Description}\\
This function is used to find the speed of a joint.  This is the speed at which the joint will move when given motion commands. The values should be between 0 and 1. \\

\noindent
{\bf Example}\\
\noindent

\noindent
{\bf See Also}\\

%\CPlot::\DataThreeD(), \CPlot::\DataFile(), \CPlot::\Plotting(), \plotxy().\\
