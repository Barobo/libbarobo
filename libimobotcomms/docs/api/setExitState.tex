\noindent
\vspace{5pt}
\rule{4.5in}{0.015in}\\
\noindent
{\LARGE \texttt{CMobot::setExitState()}\index{CMobot::setExitState()}}\\
%\phantomsection
\addcontentsline{toc}{section}{setExitState()}

\noindent
{\bf Synopsis}
\vspace{-8pt}
\begin{verbatim}
#include <mobot.h>
int CMobot::setExitState(mobotJointState_t state);
\end{verbatim}

\noindent
{\bf Purpose}\\
Sets the behaviour of joints when the program exits.\\

\noindent
{\bf Return Value}\\
The function returns 0 on success and non-zero otherwise.\\

\noindent
{\bf Parameters}
\vspace{-0.1in}
\begin{description}
\item               
\begin{tabular}{p{10 mm}p{145 mm}}
\texttt{state} & The desired behaviour of the robot joints on program exit. By
default, the behavior is set to \texttt{MOBOT\_NEUTRAL}, which relaxes all of
the joints. Alternatively, the state may be set to \texttt{MOBOT\_HOLD} to hold
the joints on program exit. \\
\end{tabular}
\end{description}

\noindent
{\bf Description}\\
This function is used to change the behaviour of the joints when the program
exits. By default, when a Mobot program exits, the all joints of the connected
Mobots go to the \texttt{MOBOT\_NEUTRAL} state. Using this function, the exit
state may be changed to \texttt{MOBOT\_HOLD} or \texttt{MOBOT\_NEUTRAL}. All
other joint states are invalid when used with this function.
\noindent\\
{\bf Example}\\
\noindent

\noindent
{\bf See Also}\\

%\CPlot::\DataThreeD(), \CPlot::\DataFile(), \CPlot::\Plotting(), \plotxy().\\
