\noindent
\vspace{5pt}
\rule{4.5in}{0.015in}\\
\noindent
{\LARGE \texttt{CMobot::motionRollForward()}\index{CMobot::motionRollForward()}}\\
{\LARGE \texttt{CMobot::motionRollForwardNB()}\index{CMobot::motionRollForwardNB()}}\\
%\phantomsection
\addcontentsline{toc}{section}{motionRollForward()}
\addcontentsline{toc}{section}{motionRollForwardNB()}

\noindent
{\bf Synopsis}
\begin{verbatim}
#include <mobot.h>
int CMobot::motionRollForward();
int CMobot::motionRollForwardNB();
\end{verbatim}

\noindent
{\bf Purpose}\\
Use the faceplates as wheels to roll forward.\\

\noindent
{\bf Return Value}\\
The function returns 0 on success and non-zero otherwise.\\

\noindent
{\bf Parameters}\\
None.\\

\noindent
{\bf Description}\\
This function causes each of the faceplates to rotate 90 degrees to roll the
robot forward.

This function has both a blocking and non-blocking version.
The blocking version, \texttt{motionRollForward()}, will block until the
robot motion has completed. The non-blocking version, \texttt{motionRollForwardNB()},
will return immediately, and the motion will be performed asynchronously.\\

\noindent
{\bf See Also}\\
\texttt{motionRollBackward()}

%\CPlot::\DataThreeD(), \CPlot::\DataFile(), \CPlot::\Plotting(), \plotxy().\\
