\noindent
\vspace{5pt}
\rule{4.5in}{0.015in}\\
\noindent
{\LARGE \texttt{CMobot::getJointAngles()}\index{CMobot::getJointAngles()}}\\
{\LARGE \texttt{CMobot::getJointAnglesAbs()}\index{CMobot::getJointAnglesAbs()}}\\
%\phantomsection
\addcontentsline{toc}{section}{getJointAngles()}

\noindent
{\bf Synopsis}
\vspace{-8pt}
\begin{verbatim}
#include <mobot.h>
int CMobot::getJointAngles(
    double &angle1,
    double &angle2,
    double &angle3,
    double &angle4);
int CMobot::getJointAnglesAbs(
    double &angle1,
    double &angle2,
    double &angle3,
    double &angle4);
\end{verbatim}

\noindent
{\bf Purpose}\\
Retrieve a Mobot's current joint angles.\\

\noindent
{\bf Return Value}\\
The function returns 0 on success and non-zero otherwise.\\

\noindent
{\bf Parameters}\\
\vspace{-0.1in}
\begin{description}
\item               
\begin{tabular}{p{15 mm}p{145 mm}}
\texttt{angle1} & A variable to store the current angle of the mobot
motor. The contents of this variable will be overwritten with a value that
represents the motor's angle in degrees.  \\
\texttt{angle2} & ...  \\
\texttt{angle3} & ...  \\
\texttt{angle4} & ...  \\
\end{tabular}
\end{description}

\noindent
{\bf Description}\\
This function gets the current motor angles of a Mobot's motors. The
angle returned is in units of degrees and is accurate to roughly $\pm0.17$
degrees. 

The function \texttt{getJointAngles()} always returns an angle between -180 and
+180 degrees. The \texttt{getJointAnglesAbs()} function, however, gets the total
angle the joint has turned since the mobot has been powered on. For instance, 
if the faceplate joint 1 has been rotated two full rotations after initial power up,
  the function \texttt{getJointAngles()} will report that the joint is at angle 0,
  whereas the function \texttt{getJointAnglesAbs()} will report that the joint
  angle is 720 degrees.
\\

\noindent
{\bf Example}\\
\noindent

\noindent
{\bf See Also}\\

%\CPlot::\DataThreeD(), \CPlot::\DataFile(), \CPlot::\Plotting(), \plotxy().\\
