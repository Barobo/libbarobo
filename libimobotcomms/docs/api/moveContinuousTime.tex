\noindent
\vspace{5pt}
\rule{4.5in}{0.015in}\\
\noindent
{\LARGE \texttt{CMobot::moveContinuousTime()}\index{moveContinuousTime()}}\\
%\phantomsection
\addcontentsline{toc}{section}{moveContinuousTime()}

\noindent
{\bf Synopsis}\\
\begin{verbatim}
#include <mobot.h>
int CMobot::moveContinuousTime(int dir1, int dir2, int dir3, int dir4, int msecs);
\end{verbatim}

\noindent
{\bf Purpose}\\
Move the joints of a MoBot continuously in the specified directions.\\

\noindent
{\bf Return Value}\\
The function returns 0 on success and non-zero otherwise.\\

\noindent
{\bf Parameters}\\
Each integer direction parameter specifies the direction the joint should move. The types
are enumerated in \texttt{mobot.h} and have the following values:
\begin{itemize}
\item \texttt{MOBOT\_NEUTRAL} : The joint should not move.
\item \texttt{MOBOT\_FORWARD} : The joint will begin moving in the positive direction.
\item \texttt{MOBOT\_BACKWARD}: The joint will begin moving in the negative direction.
\end{itemize}
The \texttt{msecs} parameter is the time to perform the movement, in milliseconds.
\\

\noindent
{\bf Description}\\
This function causes joints of a MoBot to begin moving. The joints will continue moving
until the joint hits a joint limit, or the time specified in the \texttt{msecs} parameter
is reached. This function will block until the motion is completed.\\

\noindent
{\bf Example}\\
\noindent

\noindent
{\bf See Also}\\

%\CPlot::\DataThreeD(), \CPlot::\DataFile(), \CPlot::\Plotting(), \plotxy().\\
