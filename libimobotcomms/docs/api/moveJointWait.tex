\noindent
\vspace{5pt}
\rule{4.5in}{0.015in}\\
\noindent
{\LARGE \texttt{CMobot::moveJointWait()}\index{CMobot::moveJointWait()}}\\
%\phantomsection
\addcontentsline{toc}{section}{moveJointWait()}

\noindent
{\bf Synopsis}
\vspace{-8pt}
\begin{verbatim}
#include <mobot.h>
int CMobot::moveJointWait(robotJointId_t id);
\end{verbatim}

\noindent
{\bf Purpose}\\
Wait for a joint to stop moving.\\

\noindent
{\bf Return Value}\\
The function returns 0 on success and non-zero otherwise.\\

\noindent
{\bf Parameters}
\vspace{-0.1in}
\begin{description}
\item               
\begin{tabular}{p{10 mm}p{145 mm}}
\texttt{id} & The joint number to wait for. \\
\end{tabular}
\end{description}

\noindent
{\bf Description}\\
This function is used to wait for a joint motion to finish. Functions such as
\texttt{moveNB()} and \texttt{moveJointNB()} do not wait for a joint to finish
moving before continuing to allow multiple joints to move at the same time. The
\texttt{moveJointWait()} function is used to wait for a 
robotic joint motion to complete.

Please note that if this function is called after a motor has been commanded to
turn indefinitely, this function may never return and your program may hang.\\

\noindent
{\bf Example}\\
Please see the example in Section \ref{sec:democode} on page \pageref{sec:democode}.\\
\noindent

\noindent
{\bf See Also}\\
\texttt{moveWait()}

%\CPlot::\DataThreeD(), \CPlot::\DataFile(), \CPlot::\Plotting(), \plotxy().\\
