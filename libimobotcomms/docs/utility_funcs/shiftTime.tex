\noindent
\vspace{5pt}
\rule{4.5in}{0.015in}\\
\noindent
{\LARGE \texttt{shiftTime()}\index{shiftTime()}}\\
%\phantomsection
\addcontentsline{toc}{section}{shiftTime()}

\noindent
{\bf Synopsis}
\vspace{-8pt}
\begin{verbatim}
#include <mobot.h>
int shiftTime(double tolerance, int numDataPoints, double time[], double data1[], ...);
\end{verbatim}

\noindent
{\bf Syntax}
\vspace{-8pt}
\begin{verbatim}
#include <mobot.h>
shiftTime(tolerance, numDataPoints, time, angles1);
shiftTime(tolerance, numDataPoints, time, angles1, angles2);
shiftTime(tolerance, numDataPoints, time, angles1, angles2, angles3);
shiftTime(tolerance, numDataPoints, time, angles1, angles2, angles3, angles4);
etc...
\end{verbatim}

\noindent
{\bf Purpose}\\
This function is used to shift the data in one or more plots to the left. It is commonly used to
line up the beginning of robot motions with the y-axis on plots.\\

\noindent
{\bf Return Value}\\
The return value is the number of elements which have been shifted of the plots. \\

\noindent
{\bf Parameters}
\vspace{-0.1in}
\begin{description}
\item               
\begin{tabular}{p{25 mm}p{145 mm}}
\texttt{tolerance} & The angle tolerance to detect the beginning of the motion. A lower tolerance
is more sensitive to small motions, but also more sensitive to noise. A higher tolerance will
reject noise, but may yield an inaccurate shift in time such that the motion does not appear to
begin at time 0. \\
\texttt{numDataPoints} & The number of elements in the arrays. \\
\texttt{time} & The array holding time or "x-axis" values. \\
\texttt{data1} & An array holding data. \\
\texttt{...} & Additional arrays holding data. 
\end{tabular}
\end{description}

\noindent
{\bf Description}\\
This function is used to shift data to the left to align motion start times with the
y-axis. This is done by detecting a change in value in any of the data arrays provided
to the function. If there is a change greater than the value provided as the
tolerance, that time is labeled as the beginning of the motion. All data points
prior to the beginning of the motion are deleted, and the beginning of the
motion is aligned with time 0.

\noindent
{\bf Example}\\
\noindent

\noindent
{\bf See Also}\\

%\CPlot::\DataThreeD(), \CPlot::\DataFile(), \CPlot::\Plotting(), \plotxy().\\
