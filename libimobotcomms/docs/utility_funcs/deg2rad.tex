\noindent
\vspace{5pt}
\rule{4.5in}{0.015in}\\
\noindent
{\LARGE \texttt{deg2rad()}\index{deg2rad()}}\\
%\phantomsection
\addcontentsline{toc}{section}{deg2rad()}

\noindent
{\bf Synopsis}
\vspace{-8pt}
\begin{verbatim}
#include <mobot.h>
double deg2rad(double degrees);
array double deg2rad(double degrees[:])[:];
\end{verbatim}

\noindent
{\bf Purpose}\\
Convert degrees to radians.

\noindent
{\bf Return Value}\\
The angle parameter converted to radians.

\noindent
{\bf Parameters}
\vspace{-0.1in}
\begin{description}
\item               
\begin{tabular}{p{10 mm}p{145 mm}}
\texttt{degrees} & The angle to convert, in degrees. \\
\end{tabular}
\end{description}

\noindent
{\bf Description}\\
This function converts an angle expressed in degrees into radians. Degrees and
radians are two popular ways to express an angle, though they are not interchangable.
The following equation is used to convert degrees to radians:
\begin{equation*}
\theta = \delta * \frac{\pi}{180}
\end{equation*}
where $\theta$ is the angle in radians and $\delta$ is the angle in degrees.

\noindent
{\bf Example}\\
\noindent

\noindent
{\bf See Also}\\
\texttt{rad2deg()}

%\CPlot::\DataThreeD(), \CPlot::\DataFile(), \CPlot::\Plotting(), \plotxy().\\
