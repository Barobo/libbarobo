%\lhead{libimobotcomms API Documentation}
\noindent
There are several utility functions which are useful when programming for
the Mobot. 

\begin{table}[!h]
%\capstart
\begin{center}
\caption{Mobot Utility Functions.}
\begin{tabular}{p{38 mm}p{110 mm}}
%\begin{tabular}{ll}
\hline
Function & Description \\
\hline
%\texttt{pose()} \dotfill & Pose multiple joints of the mobot. \\
\texttt{angle2distance()} & Calculates the angle a wheel has turned from the
radius and distance traveled.\\
\texttt{angles2distances()} & Calculates the angle a wheel has turned from the
radius and distance traveled. Use this function when the data is stored in normal C arrays rather than computational arrays.\\
\texttt{deg2rad()} & Converts degrees to radians. \\
\texttt{delay()} & Puts a pause into a program. \\
\texttt{distance2angle()} & Calculates the distance traveled by a wheel from the wheel's radius and angle turned.\\
\texttt{rad2deg()} & Converts radians to degrees.\\
\texttt{shiftTime()} & Shift the data of a plot.\\
\hline
\end{tabular}
\end{center}
\label{mobilec_api_cbinary}
\end{table}

\clearpage
\newpage
\noindent
\vspace{5pt}
\rule{4.5in}{0.015in}\\
\noindent
{\LARGE \texttt{angle2distance()}\index{angle2distance()}}\\
%\phantomsection
\addcontentsline{toc}{section}{angle2distance()}

\noindent
{\bf Synopsis}
\vspace{-8pt}
\begin{verbatim}
#include <mobot.h>
double angle2distance(double radius, double angle);
array double angle2distance(double radius, array double angle[:])[:];
\end{verbatim}

\noindent
{\bf Purpose}\\
Calculate the distance a wheel has traveled from the radius of the wheel and
the angle the wheel has turned.

\noindent
{\bf Return Value}\\
The value returned is the distance traveled by the wheel. If the angle argument is an
array of angles, then the value returned is an array of distances. Each element
of the distance array returned is the distance calculated from the respective
element in the angle array.\\

\noindent
{\bf Parameters}
\vspace{-0.1in}
\begin{description}
\item               
\begin{tabular}{p{10 mm}p{145 mm}}
\texttt{radius} & The radius of the wheel. \\
\texttt{angle} & This value is the angle the wheel has turned. This parameter may be of \texttt{double} type, or a Ch computational array. \\
\end{tabular}
\end{description}

\noindent
{\bf Description}\\
This function calculates the angle a wheel has turned given the wheel 
radius and distance traveled. The equation used is
\begin{equation*}
d = r \theta
\end{equation*}
where $d$ is the distance traveled, $r$ is the radius of the wheel, and $\theta$ is
the angle the wheel has turned in radians.
\\

\noindent
{\bf Example}\\
\noindent

\noindent
{\bf See Also}\\
\texttt{distance2angle()}

%\CPlot::\DataThreeD(), \CPlot::\DataFile(), \CPlot::\Plotting(), \plotxy().\\

\noindent
\vspace{5pt}
\rule{4.5in}{0.015in}\\
\noindent
{\LARGE \texttt{angles2distances()}\index{angles2distances()}}\\
%\phantomsection
\addcontentsline{toc}{section}{angles2distances()}

\noindent
{\bf Synopsis}
\vspace{-8pt}
\begin{verbatim}
#include <mobot.h>
void angles2distances(double radius, double *angles, double *distances, int num);
\end{verbatim}

\noindent
{\bf Purpose}\\
Calculate the distance a wheel has traveled from the radius of the wheel and
the angle the wheel has turned.\\

\noindent
{\bf Return Value}\\
The value returned is the distance traveled by the wheel. If the angle argument is an
array of angles, then the value returned is an array of distances. Each element
of the distance array returned is the distance calculated from the respective
element in the angle array.\\

\noindent
{\bf Parameters}
\vspace{-0.1in}
\begin{description}
\item               
\begin{tabular}{p{15 mm}p{145 mm}}
\texttt{radius} & The radius of the wheel. \\
\texttt{angles} & (In) An array of angle values.\\
\texttt{distances} & (Out) An array that will be filled with distance values.\\
\texttt{num} & The number of elements in the \texttt{angles} array.\\
\end{tabular}
\end{description}

\noindent
{\bf Description}\\
This function calculates the angle a wheel has turned given the wheel 
radius and distance traveled. The equation used is
\begin{equation*}
d = r \theta
\end{equation*}
where $d$ is the distance traveled, $r$ is the radius of the wheel, and $\theta$ is
the angle the wheel has turned in radians.
\\

\noindent
{\bf Example}\\
\noindent

\noindent
{\bf See Also}\\
\texttt{distance2angle()}

%\CPlot::\DataThreeD(), \CPlot::\DataFile(), \CPlot::\Plotting(), \plotxy().\\

\noindent
\vspace{5pt}
\rule{4.5in}{0.015in}\\
\noindent
{\LARGE \texttt{deg2rad()}\index{deg2rad()}}\\
%\phantomsection
\addcontentsline{toc}{section}{deg2rad()}

\noindent
{\bf Synopsis}
\vspace{-8pt}
\begin{verbatim}
#include <mobot.h>
double deg2rad(double degrees);
array double deg2rad(double degrees[:])[:];
\end{verbatim}

\noindent
{\bf Purpose}\\
Convert degrees to radians.

\noindent
{\bf Return Value}\\
The angle parameter converted to radians.

\noindent
{\bf Parameters}
\vspace{-0.1in}
\begin{description}
\item               
\begin{tabular}{p{10 mm}p{145 mm}}
\texttt{degrees} & The angle to convert, in degrees. \\
\end{tabular}
\end{description}

\noindent
{\bf Description}\\
This function converts an angle expressed in degrees into radians. Degrees and
radians are two popular ways to express an angle, though they are not interchangable.
The following equation is used to convert degrees to radians:
\begin{equation*}
\theta = \delta * \frac{\pi}{180}
\end{equation*}
where $\theta$ is the angle in radians and $\delta$ is the angle in degrees.

\noindent
{\bf Example}\\
\noindent

\noindent
{\bf See Also}\\
\texttt{rad2deg()}

%\CPlot::\DataThreeD(), \CPlot::\DataFile(), \CPlot::\Plotting(), \plotxy().\\

\noindent
\vspace{5pt}
\rule{4.5in}{0.015in}\\
\noindent
{\LARGE \texttt{delay()}\index{delay()}}\\
%\phantomsection
\addcontentsline{toc}{section}{delay()}

\noindent
{\bf Synopsis}
\vspace{-8pt}
\begin{verbatim}
#include <mobot.h>
void delay(double seconds);
\end{verbatim}

\noindent
{\bf Purpose}\\
Pause a program for a set amount of time.\\

\noindent
{\bf Return Value}\\
None.\\

\noindent
{\bf Parameters}
\vspace{-0.1in}
\begin{description}
\item               
\begin{tabular}{p{15 mm}p{145 mm}}
\texttt{seconds} & The number of seconds to delay. \\
\end{tabular}
\end{description}

\noindent
{\bf Description}\\
This function delays or pauses a program for a number of seconds. For instance, 
the code 
\begin{verbatim}
delay(0.5);
printf("Hello.\n");
delay(2);
printf("Goodbye.\n");
\end{verbatim}
will pause for half a second, print the text \texttt{Hello.}, delay for 2 seconds,
and then print the text \texttt{Goodbye.}. 
\noindent
{\bf Example}\\
\noindent

\noindent
{\bf See Also}\\

%\CPlot::\DataThreeD(), \CPlot::\DataFile(), \CPlot::\Plotting(), \plotxy().\\

\noindent
\vspace{5pt}
\rule{4.5in}{0.015in}\\
\noindent
{\LARGE \texttt{distance2angle()}\index{distance2angle()}}\\
%\phantomsection
\addcontentsline{toc}{section}{distance2angle()}

\noindent
{\bf Synopsis}
\vspace{-8pt}
\begin{verbatim}
#include <mobot.h>
double distance2angle(double radius, double distance);
array double distance2angle(double radius, array double distance[:])[:];
\end{verbatim}

\noindent
{\bf Purpose}\\
Calculate the angle a wheel has turned from the radius of the wheel and
the distance the wheel has traveled.

\noindent
{\bf Return Value}\\
The value returned is the angle turned by the wheel in degrees. If the distance argument is an
array of distances, then the value returned is an array of angles. Each element
of the angle array returned is the angle calculated from the respective
element in the distance array.\\

\noindent
{\bf Parameters}
\vspace{-0.1in}
\begin{description}
\item               
\begin{tabular}{p{10 mm}p{145 mm}}
\texttt{radius} & The radius of the wheel. \\
\texttt{distance} & This value is the distance the wheel has traveled. This parameter may be of \texttt{double} type, or a Ch computational array. \\
\end{tabular}
\end{description}

\noindent
{\bf Description}\\
This function calculates the distance a wheel has turned given the wheel 
radius and angle turned. The equation used is
\begin{equation*}
\theta = \frac{d}{r}
\end{equation*}
where $d$ is the distance traveled, $r$ is the radius of the wheel, and $\theta$ is
the angle the wheel has turned in radians. A further conversion is done in the code to
convert the angle from radians into degrees before returning the value.
\\

\noindent
{\bf Example}\\
\noindent

\noindent
{\bf See Also}\\
\texttt{angle2distance()}

%\CPlot::\DataThreeD(), \CPlot::\DataFile(), \CPlot::\Plotting(), \plotxy().\\

\noindent
\vspace{5pt}
\rule{4.5in}{0.015in}\\
\noindent
{\LARGE \texttt{rad2deg()}\index{rad2deg()}}\\
%\phantomsection
\addcontentsline{toc}{section}{rad2deg()}

\noindent
{\bf Synopsis}
\vspace{-8pt}
\begin{verbatim}
#include <mobot.h>
double rad2deg(double radians);
array double rad2deg(double radians[:])[:];
\end{verbatim}

\noindent
{\bf Purpose}\\
Convert radians to degrees.

\noindent
{\bf Return Value}\\
The angle parameter converted to degrees.

\noindent
{\bf Parameters}
\vspace{-0.1in}
\begin{description}
\item               
\begin{tabular}{p{10 mm}p{145 mm}}
\texttt{radians} & The angle to convert, in radians. \\
\end{tabular}
\end{description}

\noindent
{\bf Description}\\
This function converts an angle expressed in radians into degrees. Degrees and
radians are two popular ways to express an angle, though they are not interchangable.
The following equation is used to convert radians to degrees:
\begin{equation*}
\delta = \theta * \frac{180}{\pi}
\end{equation*}
where $\theta$ is the angle in radians and $\delta$ is the angle in degrees.

\noindent
{\bf Example}\\
\noindent

\noindent
{\bf See Also}\\
\texttt{deg2rad()}

%\CPlot::\DataThreeD(), \CPlot::\DataFile(), \CPlot::\Plotting(), \plotxy().\\

\noindent
\vspace{5pt}
\rule{4.5in}{0.015in}\\
\noindent
{\LARGE \texttt{shiftTime()}\index{shiftTime()}}\\
%\phantomsection
\addcontentsline{toc}{section}{shiftTime()}

\noindent
{\bf Synopsis}
\vspace{-8pt}
\begin{verbatim}
#include <mobot.h>
int shiftTime(double tolerance, int numDataPoints, double array time[:], double array data1[:], ...);
\end{verbatim}

\noindent
{\bf Purpose}\\
This function is used to shift the data in one or more plots to the left. It is commonly used to
line up the beginning of robot motions with the y-axis on plots.

\noindent
{\bf Return Value}\\
The return value is the number of elements which have been shifted of the plots. \\

\noindent
{\bf Parameters}
\vspace{-0.1in}
\begin{description}
\item               
\begin{tabular}{p{10 mm}p{145 mm}}
\texttt{tolerance} & The angle tolerance to detect the beginning of the motion. A lower tolerance
is more sensitive to small motions, but also more sensitive to noise. A higher tolerance will
reject noise, but may yield an inaccurate shift in time such that the motion does not appear to
begin at time 0. \\
\texttt{numDataPoints} & The number of elements in the arrays. \\
\texttt{time} & The array holding time or "x-axis" values. \\
\texttt{data1} & An array holding data. \\
\texttt{...} & Additional arrays holding data. 
\end{tabular}
\end{description}

\noindent
{\bf Description}\\
This function is used to shift data to the left to align motion start times with the
y-axis. This is done by detecting a change in value in any of the data arrays provided
to the function. If there is a change greater than the value provided as the
tolerance, that time is labeled as the beginning of the motion. All data points
prior to the beginning of the motion are deleted, and the beginning of the
motion is aligned with time 0.

\noindent
{\bf Example}\\
\noindent

\noindent
{\bf See Also}\\

%\CPlot::\DataThreeD(), \CPlot::\DataFile(), \CPlot::\Plotting(), \plotxy().\\

