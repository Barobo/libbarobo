\noindent
\vspace{5pt}
\rule{4.5in}{0.015in}\\
\noindent
{\LARGE \texttt{rad2deg()}\index{rad2deg()}}\\
%\phantomsection
\addcontentsline{toc}{section}{rad2deg()}

\noindent
{\bf Synopsis}
\vspace{-8pt}
\begin{verbatim}
#include <mobot.h>
double rad2deg(double radians);
array double rad2deg(double radians[:])[:];
\end{verbatim}

\noindent
{\bf Purpose}\\
Convert radians to degrees.

\noindent
{\bf Return Value}\\
The angle parameter converted to degrees.

\noindent
{\bf Parameters}
\vspace{-0.1in}
\begin{description}
\item               
\begin{tabular}{p{10 mm}p{145 mm}}
\texttt{radians} & The angle to convert, in radians. \\
\end{tabular}
\end{description}

\noindent
{\bf Description}\\
This function converts an angle expressed in radians into degrees. Degrees and
radians are two popular ways to express an angle, though they are not interchangable.
The following equation is used to convert radians to degrees:
\begin{equation*}
\delta = \theta * \frac{180}{\pi}
\end{equation*}
where $\theta$ is the angle in radians and $\delta$ is the angle in degrees.

\noindent
{\bf Example}\\
\noindent

\noindent
{\bf See Also}\\
\texttt{deg2rad()}

%\CPlot::\DataThreeD(), \CPlot::\DataFile(), \CPlot::\Plotting(), \plotxy().\\
