\begin{tabular}{p{1.75in}p{4.5in}}
%\begin{tabular}{ll}
\hline
Function & Description \\
\hline
%\texttt{pose()} \dotfill & Pose multiple joints of the robot. \\
\texttt{CMobotGroup()} & The CMobotGroup constructor function. This function
is called automatically and should not be called explicitly. \\
\texttt{\textasciitilde CMobotGroup()} & The CMobotGroup destructor function. This function
is called automatically and should not be called explicitly. \\
& \\
\texttt{addRobot()} & Add a robot to be a member of the robot group. \\
\texttt{move()} & Move all four joints of the robots by specified angles. \\
\texttt{moveNB()} & Identical to \texttt{move()} but non-blocking. \\
\texttt{moveContinuousNB()} & Move joints continuously. Joints will move untill stopped.\\
\texttt{moveContinuousTime()} & Move joints continuously for a certain amount of time.\\
\texttt{moveJointContinuousNB()} & Move a single joint on all robots continuously. \\
\texttt{moveJointContinuousTime()} & Move a single joint on all robots continuously for a specific amount of time. \\
\texttt{moveTo()} & Move all four joints of the robots to specified absolute angles. \\
\texttt{moveToNB()} & Identical to \texttt{moveTo()} but non-blocking. \\
\texttt{moveJoint()} & Move a motor from its current position by an angle. \\
\texttt{moveJointNB()} & Identical to \texttt{moveJoint()} but non-blocking. \\
\texttt{moveJointTo()} & Set the desired motor position for a joint. \\
\texttt{moveJointToNB()} & Identical to \texttt{moveJointTo()} but non-blocking. \\
\texttt{moveJointWait()} & Wait until the specified motor has stopped moving. \\
\texttt{moveWait()} & Wait until all motors have stopped moving. \\
\texttt{moveToZero()} & Instructs all motors to go to their zero positions. \\
\texttt{moveToZeroNB()} & Identical to \texttt{moveToZero()} but non-blocking. \\
%\texttt{setJointDirection()} & Set the motor direction of a motor. Set
%to "0" for automatic direction, "1" for forward, and "2" for reverse. \\
\texttt{setJointSpeed()} & Set a motor's speed setting in radians per second. \\
\texttt{setJointSpeeds()} & Set all motor speeds in radians per second. \\
\texttt{setJointSpeedRatio()} & Set a joints speed setting to a fraction of its maximum speed, a value between 0 and 1. \\
\texttt{setJointSpeedRatios()} & Set all joint speed settings to a fraction of its
maximum speed, expressed as a value from 0 to 1. \\
\texttt{setTwoWheelRobotSpeed()} & Move the robot at a constant forward velocity. \\
\texttt{stop()} & Stop all currently executing motions of the robot. \\
\hline
\end{tabular}

\begin{comment}
\begin{tabular}{p{1.75in}p{4.5in}} \hline 
Function Name & Description \\
\hline
\texttt{addRobot()} & Add a robot to the group. \\
\texttt{move()} & Move joints on robots from their current positions for all robots in the group. \\
\texttt{moveContinuousNB()} & Move joints continuously for all robots in the group. \\
\texttt{moveContinunousTime()} & Move joints continuously for a specific amount of time for all robots in the group.\\
\texttt{moveJointContinuousNB()} & Move a single joint continuously for all robots in the group. \\
\texttt{moveJointContinuousTime()} & Move a single joint continuously for a specific amount of time for all robots in the group.\\
\texttt{moveJointTo()} & Move a joint to an absolute angle for all robots in the group.\\
\texttt{moveJointWait()} & Wait for a joint to finish moving for all robots in the group. \\
\texttt{moveTo()} & Move joints to a set of specific angles for all robots in the group. \\
\texttt{moveWait()} & Wait for all joints to finish moving. \\
\texttt{moveToZero()} & Move all joints to their zero positions. \\
\texttt{setJointSpeed()} & Set the speed of a joint, in radians per second. \\
\texttt{setJointSpeedRatio()} & Set the speed of a joint as a ratio of the maximum speed. \\
\texttt{setTwoWheelRobotSpeed()} & Move the robot at a linear speed, given the wheel size and desired speed.\\
\texttt{stop()} & Stop all the joints of all robots in the group. \\
\texttt{motionArch()} & Cause all robots in the group to arch for higher rolling clearance. \\
\texttt{motionArchNB()} & Cause all robots in the group to arch for higher rolling clearance. \\
\texttt{motionInchwormLeft()} & Cause all robots in the group to inchworm. \\
\texttt{motionInchwormLeftNB()} & Cause all robots in the group to inchworm. \\
\texttt{motionInchwormRight()} & Cause all robots in the group to inchworm. \\
\texttt{motionInchwormRightNB()} & Cause all robots in the group to inchworm. \\
\texttt{motionRollBackward()} & Makes all robots in the group roll backward. \\
\texttt{motionRollForwardNB()} & Makes all robots in the group roll forward. \\
\texttt{motionSkinny()} & Make all robots in the group assume a skinny rolling profile. \\
\texttt{motionSkinnyNB()} & Make all robots in the group assume a skinny rolling profile. \\
\texttt{motionStand()} & Make all robots in the group stand. \\
\texttt{motionStandNB()} & Make all robots in the group stand. \\
\texttt{motionTumble()} & Make all the robots in the group perform the tumbling motion. \\
\texttt{motionTumbleNB()} & Make all the robots in the group perform the tumbling motion. \\
\texttt{motionTurnLeft()} & Make all robots in the group turn left. \\
\texttt{motionTurnLeftNB()} & Make all robots in the group turn left. \\
\texttt{motionTurnRight()} & Make all robots in the group turn right. \\
\texttt{motionTurnRightNB()} & Make all robots in the group turn right.\\
\texttt{motionUnstand()} & Make all robots in the group drop down from a standing position. \\
\texttt{motionUnstandNB()} & Make all robots in the group drop down from a standing position. \\
\texttt{motionWait()} & Wait for the robots to complete all currently executing compound motions.\\
\hline
\end{tabular}
\end{comment}
