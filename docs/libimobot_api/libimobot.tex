\lhead{libimobot API Documentation}
\noindent
The header file {\bf libimobot.h} defines all the data types, macros 
and function prototypes for the iMobot API library. The header file
declares a class called \texttt{CiMobot} which contains member functions which
may be used to control the robot.

\begin{table}[!hp]
\capstart
\begin{center}
\caption{CiMobot Member Functions.}
\begin{tabular}{p{58 mm}p{97 mm}}
%\begin{tabular}{ll}
\hline
Function & Description \\
\hline
%\texttt{pose()} \dotfill & Pose multiple joints of the iMobot. \\
\texttt{CiMobot()} \dotfill & The CiMobot constructor function. This function
is called automatically and should not be called explicitly. \\
\texttt{\textasciitilde CiMobot()} \dotfill & The CiMobot destructor function. This function
is called automatically and should not be called explicitly. \\
& \\
\texttt{getMotorSpeed()} \dotfill & Gets a motor's speed. \\
\texttt{initListenerBluetooth()} \dotfill & Initialize the Bluetooth module to
listen for incoming commands. \\
\texttt{isBusy()} \dotfill & See if the robot is currently moving. \\
\texttt{listenerMainLoop()} \dotfill & Main execution loop for the Bluetooth listener. \\
\texttt{moveWait()} \dotfill & Wait until all motors have stopped moving. \\
\texttt{poseJoint()} \dotfill & Pose a single joint of the iMobot. \\
\texttt{setMotorDirection()} \dotfill & Set the motor direction of a motor. Set
to "0" for automatic direction, "1" for forward, and "2" for reverse. \\
\texttt{setMotorSpeed()} \dotfill & Sets a motor's speed. \\
\texttt{stop()} \dotfill & Stop all currently executing motions of the iMobot. \\
\texttt{terminate()} \dotfill & Terminate the robotic control. \\
\texttt{waitMotor()} \dotfill & Wait until the specified motor has stopped moving. \\
\hline
\end{tabular}
\end{center}
\label{mobilec_api_cbinary}
\end{table}

\section{Constants and Enumerations}
There are various macros and enumerations declared in the header file which 
represent descriptive names for otherwise meaningless values. They are 
presented in the following table and used throughout the API.

\begin{tabular}{lp{3.7in}}
Macro & Description \\
\texttt{iMobot\_motors\_e} & \\\hline
\texttt{IMOBOT\_MOTOR1} & This macro represents the first motor of an iMobot. \\
\texttt{IMOBOT\_MOTOR2} & This macro represents the second motor of an iMobot. \\
\texttt{IMOBOT\_MOTOR3} & This macro represents the third motor of an iMobot. \\
\texttt{IMOBOT\_MOTOR4} & This macro represents the fourth motor of an iMobot. \\
\texttt{IMOBOT\_NUM\_MOTORS} & This macro contains the number of motors on an iMobot. \\
 & \\
\texttt{iMobot\_motor\_direction\_e} & \\\hline
\texttt{IMOBOT\_MOTOR\_DIR\_AUTO} & Set the iMobot direction to automatic. The iMobot will choose the best direction to turn the motor in order to get to the requested position. \\
\texttt{IMOBOT\_MOTOR\_DIR\_FORWARD} & Force the motor to turn in the ``forward'' direction. \\
\texttt{IMOBOT\_MOTOR\_DIR\_BACKWARD} & Force the motor to turn in the ``backward'' direction.
\end{tabular}

\newpage
\noindent
\vspace{5pt}
\rule{4.5in}{0.015in}\\
\noindent
{\LARGE \texttt{CiMobotComms::getJointDirection()}\index{getJointDirection()}}\\
%\phantomsection
\addcontentsline{toc}{section}{getJointDirection()}

\noindent
{\bf Synopsis}\\
\begin{verbatim}
#include <imobot.h>
int CiMobotComms::getJointDirection(int id, int &direction);
\end{verbatim}

\noindent
{\bf Purpose}\\
Get the speed of a joint on the iMobot.\\

\noindent
{\bf Return Value}\\
The function returns 0 on success and non-zero otherwise.\\

\noindent
{\bf Parameters}
\vspace{-0.1in}
\begin{description}
\item               
\begin{tabular}{p{10 mm}p{145 mm}}
\texttt{id} & The joint number to pose. \\
\texttt{direction} & An integer variable. This variable will be overwritten
with the current speed of the joint.
\end{tabular}
\end{description}

\noindent
{\bf Description}\\
This function is used to retrieve the motor's direction status. The valid
status directions are
\begin{itemize}
\item 0: Automatic direction
\item 1: Forward direction
\item 2: Backward direction
\end{itemize}

\noindent
{\bf Example}\\
\noindent

\noindent
{\bf See Also}\\
\texttt{setJointDirection()}

%\CPlot::\DataThreeD(), \CPlot::\DataFile(), \CPlot::\Plotting(), \plotxy().\\

\pagebreak
\noindent
\vspace{5pt}
\rule{4.5in}{0.015in}\\
\noindent
{\LARGE \texttt{CiMobotComms::getMotorPosition()}\index{getMotorPosition()}}\\
%\phantomsection
\addcontentsline{toc}{section}{getMotorPosition()}

\noindent
{\bf Synopsis}\\
\begin{verbatim}
#include <imobot.h>
int CiMobotComms::getMotorPosition(int id, double &position);
\end{verbatim}

\noindent
{\bf Purpose}\\
Connect to a remote iMobot via Bluetooth.\\

\noindent
{\bf Return Value}\\
The function returns 0 on success and non-zero otherwise.\\

\noindent
{\bf Parameters}\\
\vspace{-0.1in}
\begin{description}
\item               
\begin{tabular}{p{10 mm}p{145 mm}}
\texttt{id} & The joint number to wait for. \\
\texttt{position} & A variable to store the current position of the iMobot
motor. The contents of this variable will be overwritten with a value that
represents the motor's angle in degrees.  \\
\end{tabular}
\end{description}

\noindent
{\bf Description}\\
This function gets the current motor position of an iMobot's motor. The
position returned is in units of degrees and is accurate to roughly $\pm0.1$
degrees. \\

\noindent
{\bf Example}\\
\noindent

\noindent
{\bf See Also}\\
\texttt{connectAddress()}

%\CPlot::\DataThreeD(), \CPlot::\DataFile(), \CPlot::\Plotting(), \plotxy().\\

\pagebreak
\noindent
\vspace{5pt}
\rule{4.5in}{0.015in}\\
\noindent
{\LARGE \texttt{CiMobotComms::getMotorSpeed()}\index{getMotorSpeed()}}\\
%\phantomsection
\addcontentsline{toc}{section}{getMotorSpeed()}

\noindent
{\bf Synopsis}\\
\begin{verbatim}
#include <imobot.h>
int CiMobotComms::getMotorSpeed(int id, int &speed);
\end{verbatim}

\noindent
{\bf Purpose}\\
Get the speed of a joint on the iMobot.\\

\noindent
{\bf Return Value}\\
The function returns 0 on success and non-zero otherwise.\\

\noindent
{\bf Parameters}
\vspace{-0.1in}
\begin{description}
\item               
\begin{tabular}{p{10 mm}p{145 mm}}
\texttt{id} & The joint number to pose. \\
\texttt{speed} & The address of an unsigned short variable. This variable will be overwritten
with the current speed of the joint.
\end{tabular}
\end{description}

\noindent
{\bf Description}\\
This function is used to find the speed of a joint.  This is the speed at which the joint will move when given motion commands. The values should be between 0 and 100. \\

\noindent
{\bf Example}\\
\noindent

\noindent
{\bf See Also}\\

%\CPlot::\DataThreeD(), \CPlot::\DataFile(), \CPlot::\Plotting(), \plotxy().\\

\pagebreak
\noindent
\vspace{5pt}
\rule{4.5in}{0.015in}\\
\noindent
{\LARGE \texttt{CiMobotComms::getMotorState()}\index{getMotorState()}}\\
%\phantomsection
\addcontentsline{toc}{section}{getMotorState()}

\noindent
{\bf Synopsis}\\
\begin{verbatim}
#include <imobotcomms.h>
int CiMobotComms::getMotorState(int id, int &state);
\end{verbatim}

\noindent
{\bf Purpose}\\
Determine whether a motor is moving or not.\\

\noindent
{\bf Return Value}\\
The function returns 0 on success and non-zero otherwise.\\

\noindent
{\bf Parameters}
\vspace{-0.1in}
\begin{description}
\item               
\begin{tabular}{p{10 mm}p{145 mm}}
\texttt{id} & The joint number to pose. \\
\texttt{state} & An integer variable which will be overwritten with the current state of the motor. 
\end{tabular}
\end{description}

\noindent
{\bf Description}\\
This function is used to determine the current state of a motor. Valid states are:
\begin{itemize}
\item 0: The motor is idle.
\item 1: The motor is moving.
\item 2: The motor is heading towards a specified position.
\end{itemize}

\noindent
{\bf Example}\\
\noindent

\noindent
{\bf See Also}\\

%\CPlot::\DataThreeD(), \CPlot::\DataFile(), \CPlot::\Plotting(), \plotxy().\\

\pagebreak
\noindent
\vspace{5pt}
\rule{6.5in}{0.015in}
\noindent
{\LARGE \texttt{CiMobot::initListenerBluetooth()}\index{initListenerBluetooth()}}\\
\phantomsection
\addcontentsline{toc}{section}{initListenerBluetooth()}

\noindent
{\bf Synopsis}\\
\begin{verbatim}
#include <imobot.h>
int CiMobot::initListenerBluetooth(int channel);
\end{verbatim}

\noindent
{\bf Purpose}\\
Initialize the bluetooth listening service to process incoming bluetooth commands.\\

\noindent
{\bf Return Value}\\
The function returns 0 on success and non-zero otherwise.\\

\noindent
{\bf Parameters}
\vspace{-0.1in}
\begin{description}
\item               
\begin{tabular}{p{20 mm}p{145 mm}}
\texttt{channel} & The bluetooth channel to listen on. The default value is "20". \\
\end{tabular}
\end{description}

\noindent
{\bf Description}\\
This function initialized the iMobot's bluetooth listening service so that it may process incoming Bluetooth commands. If this function is not called, the iMobot will not listen to any bluetooth commands.

\noindent
{\bf Example}\\
See the sample program in Section \ref{subsec:simple.cpp} on page \pageref{subsec:simple.cpp}.
\noindent

\noindent
{\bf See Also}\\
\texttt{moveWait(), waitMotor()}

%\CPlot::\DataThreeD(), \CPlot::\DataFile(), \CPlot::\Plotting(), \plotxy().\\

\pagebreak
\noindent
\vspace{5pt}
\rule{6.5in}{0.015in}
\noindent
{\LARGE \texttt{CiMobot::isBusy()}\index{isBusy()}}\\
\phantomsection
\addcontentsline{toc}{section}{isBusy()}

\noindent
{\bf Synopsis}\\
\begin{verbatim}
#include <imobot.h>
int CiMobot::isBusy();
\end{verbatim}

The usage of this function is identical to the
\texttt{CMobot::isBusy()} function for the MoBot. Please refer
to the MoBot documentation for \texttt{CMobot::isBusy} for
detailed usage documentation.


\pagebreak
\noindent
\vspace{5pt}
\rule{6.5in}{0.015in}
\noindent
{\LARGE \texttt{listenerMainLoop()}\index{listenerMainLoop()}}\\
\phantomsection
\addcontentsline{toc}{section}{listenerMainLoop()}

\noindent
{\bf Synopsis}\\
\begin{verbatim}
#include "imobot.h"
int CiMobot::listenerMainLoop();
\end{verbatim}

\noindent
{\bf Purpose}\\
Put the iMobot into Bluetooth listening mode.\\

\noindent
{\bf Return Value}\\
The function returns 0 on success and non-zero otherwise.\\

\noindent
{\bf Description}\\
This function puts the iMobot into Bluetooth listening mode. This function will
not return until it receives a Bluetooth "quit" command.

\noindent
{\bf Example}\\
See the sample program in Section \ref{subsec:simple.cpp} on page \pageref{subsec:simple.cpp}.
\noindent

\noindent
{\bf See Also}\\

%\CPlot::\DataThreeD(), \CPlot::\DataFile(), \CPlot::\Plotting(), \plotxy().\\

\pagebreak
\noindent
\vspace{5pt}
\rule{6.5in}{0.015in}
\noindent
{\LARGE \texttt{CiMobot::moveWait()}\index{moveWait()}}\\
\phantomsection
\addcontentsline{toc}{section}{moveWait()}

\noindent
{\bf Synopsis}\\
\begin{verbatim}
#include <imobot.h>
int CiMobot::moveWait();
\end{verbatim}

\noindent
{\bf Purpose}\\
Wait for all joints to stop moving.\\

\noindent
{\bf Return Value}\\
The function returns 0 on success and non-zero otherwise.\\

\noindent
{\bf Description}\\
This function is used to wait for all joint motions to finish. Functions such as
\texttt{poseJoint()} and \texttt{moveJoint()} do not wait for a joint to finish
moving before continuing to allow multiple joints to move at the same time. The
\texttt{moveWait()} or \texttt{waitMotor()} functions are used to wait for
robotic motions to complete.

Please note that if this function is called after a motor has been commanded to
turn indefinitely, this function may never return and your program may hang.\\

\noindent
{\bf Example}\\
See the sample program in Section \ref{subsec:simple.cpp} on page \pageref{subsec:simple.cpp}.
\noindent

\noindent
{\bf See Also}\\
\texttt{moveWait(), waitMotor()}

%\CPlot::\DataThreeD(), \CPlot::\DataFile(), \CPlot::\Plotting(), \plotxy().\\

\pagebreak
\noindent
\vspace{5pt}
\rule{4.5in}{0.015in}\\
\noindent
{\LARGE \texttt{CMobotGroup::motionWait()}\index{CMobotGroup::motionWait()}}\\
%\phantomsection
\addcontentsline{toc}{section}{motionWait()}

\noindent
{\bf Synopsis}
\vspace{-8pt}
\begin{verbatim}
#include <mobot.h>
int CMobotGroup::motionWait();
\end{verbatim}

\noindent
{\bf Purpose}\\
Wait for a preprogrammed mobotic motion to finish.\\

\noindent
{\bf Return Value}\\
The function returns 0 on success and non-zero otherwise.\\

\noindent
{\bf Description}\\
This function is used to wait for a preprogrammed motion to finish. Functions such as
\texttt{motionInchwormLeftNB()} and \texttt{moveForwardNB()} do not wait for the motion to finish
moving before continuing. The
\texttt{motionWait()} function is used to wait for
preprogrammed motions to complete. See Section \ref{sec:preprogrammed_motions} for 
a list of all preprogrammed mobotic motions.\\

\noindent
{\bf Example}\\
See the sample program in Section \ref{sec:democode} on page \pageref{sec:democode}.
\noindent

\noindent
{\bf See Also}\\
\texttt{motionWait(), moveJointWait()}

%\CPlot::\DataThreeD(), \CPlot::\DataFile(), \CPlot::\Plotting(), \plotxy().\\

\pagebreak
\noindent
\vspace{5pt}
\rule{4.5in}{0.015in}\\
\noindent
{\LARGE \texttt{CiMobotComms::poseZero()}\index{poseZero()}}\\
%\phantomsection
\addcontentsline{toc}{section}{poseZero()}

\noindent
{\bf Synopsis}\\
\begin{verbatim}
#include <imobotcomms.h>
int CiMobotComms::poseZero();
\end{verbatim}

\noindent
{\bf Purpose}\\
Move all of the joints of an iMobot to their zero position.\\

\noindent
{\bf Return Value}\\
The function returns 0 on success and non-zero otherwise.\\

\noindent
{\bf Parameters}\\
None.\\

\noindent
{\bf Description}\\
This function moves all of the joints of an iMobot to their zero position.
Please note that this function is non-blocking and will return immediately. Use
this function in conjunction with the \texttt{moveWait()} function to block
until the movement completes.\\

\noindent
{\bf Example}\\
Please see the demo at Section \ref{sec:democode} on page \pageref{sec:democode}.\\
\noindent

\noindent
{\bf See Also}\\

%\CPlot::\DataThreeD(), \CPlot::\DataFile(), \CPlot::\Plotting(), \plotxy().\\

\pagebreak
\noindent
\vspace{5pt}
\rule{4.5in}{0.015in}\\
\noindent
{\LARGE \texttt{CiMobotComms::setMotorDirection()}\index{setMotorDirection()}}\\
%\phantomsection
\addcontentsline{toc}{section}{setMotorDirection()}

\noindent
{\bf Synopsis}\\
\begin{verbatim}
#include <imobot.h>
int CiMobotComms::setMotorDirection(int id, int direction);
\end{verbatim}

\noindent
{\bf Purpose}\\
Set's a motor's direction. In conjunction with \texttt{setMotorSpeed()}, this
function may be used to cause a motor to turn indefinitely.\\

\noindent
{\bf Return Value}\\
The function returns 0 on success and non-zero otherwise.\\

\noindent
{\bf Parameters}
\vspace{-0.1in}
\begin{description}
\item               
\begin{tabular}{p{20 mm}p{145 mm}}
\texttt{id} & The joint number to move. \\
\texttt{direction} & A value indicating the desired direction.
\end{tabular}
\end{description}

\noindent
{\bf Description}\\
This function is used to set a motor's turn direction. Possible values for the
direction are:
\begin{itemize}
\item 0: Automatic direction. This is the default setting. 
\item 1: Forward. If this value is used with a non-zero speed set using the
\texttt{setMotorSpeed()} function, the motor will turn forward indefinitely.
\time 2: Backward. Similar to "1", except the motor will spin backward.
\end{itemize}

\noindent
{\bf Example}\\
\noindent

\noindent
{\bf See Also}\\

%\CPlot::\DataThreeD(), \CPlot::\DataFile(), \CPlot::\Plotting(), \plotxy().\\

\pagebreak
\noindent
\vspace{5pt}
\rule{4.5in}{0.015in}\\
\noindent
{\LARGE \texttt{CiMobot::setMotorPosition()}\index{setMotorPosition()}}\\
%\phantomsection
\addcontentsline{toc}{section}{setMotorPosition()}

\noindent
{\bf Synopsis}\\
\begin{verbatim}
#include <imobot.h>
int CiMobot::setMotorPosition(int id, double position);
\end{verbatim}

\noindent
{\bf Purpose}\\
Connect to a remote iMobot via Bluetooth.\\

\noindent
{\bf Return Value}\\
The function returns 0 on success and non-zero otherwise.\\

\noindent
{\bf Parameters}\\
\vspace{-0.1in}
\begin{description}
\item               
\begin{tabular}{p{20 mm}p{145 mm}}
\texttt{id} & The joint number to wait for. \\
\texttt{position} & The absolute angle to move the motor to.  \\
\end{tabular}
\end{description}

\noindent
{\bf Description}\\
This function commands the motor to move to a position specified in degrees at
the current motor's speed. The current motor speed may be set with the
\texttt{setMotorSpeed()} member function. Please note that if the motor speed
is set to zero, the motor will not move after calling the
\texttt{setMotorPosition()} function. \\

\noindent
{\bf Example}\\
Please see the example in Section \ref{sec:democode} on page \pageref{sec:democode}.\\
\noindent

\noindent
{\bf See Also}\\
\texttt{connectAddress()}

%\CPlot::\DataThreeD(), \CPlot::\DataFile(), \CPlot::\Plotting(), \plotxy().\\

\pagebreak
\noindent
\vspace{5pt}
\rule{4.5in}{0.015in}\\
\noindent
{\LARGE \texttt{CiMobotComms::setMotorSpeed()}\index{setMotorSpeed()}}\\
%\phantomsection
\addcontentsline{toc}{section}{setMotorSpeed()}

\noindent
{\bf Synopsis}\\
\begin{verbatim}
#include <imobotcomms.h>
int CiMobotComms::setMotorSpeed(int id, int speed);
\end{verbatim}

\noindent
{\bf Purpose}\\
Get the speed of a joint on the iMobot.\\

\noindent
{\bf Return Value}\\
The function returns 0 on success and non-zero otherwise.\\

\noindent
{\bf Parameters}
\vspace{-0.1in}
\begin{description}
\item               
\begin{tabular}{p{10 mm}p{145 mm}}
\texttt{id} & The joint number to pose. \\
\texttt{speed} & An unsigned short variable indicating the requested speed.
\end{tabular}
\end{description}

\noindent
{\bf Description}\\
This function is used to set the speed of a joint of an iMobot. Valid speed
values range from 0 to 100.
\noindent
{\bf Example}\\
\noindent

\noindent
{\bf See Also}\\

%\CPlot::\DataThreeD(), \CPlot::\DataFile(), \CPlot::\Plotting(), \plotxy().\\

\pagebreak
\noindent
\vspace{5pt}
\rule{4.5in}{0.015in}\\
\noindent
{\LARGE \texttt{CMobot::stopAllJoints()}\index{CMobot::stopAllJoints()}}\\
{\LARGE \texttt{CMobot::stopOneJoint()}\index{CMobot::stopOneJoint()}}\\
{\LARGE \texttt{CMobot::stopTwoJoints()}\index{CMobot::stopTwoJoints()}}\\
{\LARGE \texttt{CMobot::stopThreeJoints()}\index{CMobot::stopThreeJoints()}}\\
%\phantomsection
\addcontentsline{toc}{section}{stopAllJoints()}
\addcontentsline{toc}{section}{stopOneJoint()}
\addcontentsline{toc}{section}{stopTwoJoints()}
\addcontentsline{toc}{section}{stopThreeJoints()}

\noindent
{\bf Synopsis}
\vspace{-8pt}
\begin{verbatim}
#include <mobot.h>
int CMobot::stopAllJoints();
int CMobot::stopOneJoint(mobotJointId_t id);
int CMobot::stopTwoJoints(mobotJointId_t id1, mobotJointId_t id2);
int CMobot::stopThreeJoints(
    mobotJointId_t id1,
    mobotJointId_t id2,
    mobotJointId_t id3);
\end{verbatim}

\noindent
{\bf Purpose}\\
These functions stop joints on a mobot.\\

\noindent
{\bf Return Value}\\
The function returns 0 on success and non-zero otherwise.\\

\noindent
{\bf Description}\\
These functions are used to stop joints on a mobot. A stopped joint
will immediately cease any and all actuation and go limp.

\noindent
{\bf Example}\\
\noindent

\noindent
{\bf See Also}\\
\texttt{setJointSpeed(), setJointSpeeds()}

%\CPlot::\DataThreeD(), \CPlot::\DataFile(), \CPlot::\Plotting(), \plotxy().\\

\pagebreak
\noindent
\vspace{5pt}
\rule{6.5in}{0.015in}
\noindent
{\LARGE \texttt{CiMobot::terminate()}\index{terminate()}}\\
\phantomsection
\addcontentsline{toc}{section}{terminate()}

\noindent
{\bf Synopsis}\\
\begin{verbatim}
#include <imobot.h>
int CiMobot::terminate();
\end{verbatim}

\noindent
{\bf Purpose}\\
Terminate control of an iMobot.\\

\noindent
{\bf Return Value}\\
The function returns 0 on success and non-zero otherwise.\\

\noindent
{\bf Description}\\
This function releases control of the I2C bus, thus terminating communications
with the iMobot. Only a single program may control the iMobot at the same time. If multiple programs are running, the program that currently has control of the iMobot needs to call the \texttt{terminate()} function before other programs can control the iMobot.

\noindent
{\bf Example}\\
See the sample program in Section \ref{subsec:simple.cpp} on page \pageref{subsec:simple.cpp}.
\noindent

\noindent
{\bf See Also}\\

%\CPlot::\DataThreeD(), \CPlot::\DataFile(), \CPlot::\Plotting(), \plotxy().\\

\pagebreak
\noindent
\vspace{5pt}
\rule{6.5in}{0.015in}
\noindent
{\LARGE \texttt{waitMotor()}\index{waitMotor()}}\\
\phantomsection
\addcontentsline{toc}{section}{waitMotor()}

\noindent
{\bf Synopsis}\\
\begin{verbatim}
#include "imobot.h"
int CiMobot::waitMotor(unsigned short id);
\end{verbatim}

\noindent
{\bf Purpose}\\
Wait for a joint to stop moving.\\

\noindent
{\bf Return Value}\\
The function returns 0 on success and non-zero otherwise.\\

\noindent
{\bf Parameters}
\vspace{-0.1in}
\begin{description}
\item               
\begin{tabular}{p{10 mm}p{145 mm}}
\texttt{id} & The joint number to wait for. \\
\end{tabular}
\end{description}

\noindent
{\bf Description}\\
This function is used to wait for a joint motion to finish. Functions such as
\texttt{poseJoint()} and \texttt{moveJoint()} do not wait for a joint to finish
moving before continuing to allow multiple joints to move at the same time. The
\texttt{waitMotor()} or \texttt{waitMotor()} functions are used to wait for
robotic motions to complete.

Please note that if this function is called after a motor has been commanded to
turn indefinitely, this function may never return and your program may hang.\\

\noindent
{\bf Example}\\
See the sample program in Section \ref{subsec:simple.cpp} on page \pageref{subsec:simple.cpp}.
\noindent

\noindent
{\bf See Also}\\
\texttt{moveWait()}

%\CPlot::\DataThreeD(), \CPlot::\DataFile(), \CPlot::\Plotting(), \plotxy().\\

