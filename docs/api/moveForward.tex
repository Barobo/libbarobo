\noindent
\vspace{5pt}
\rule{4.5in}{0.015in}\\
\noindent
{\LARGE \texttt{CMobot::moveForward()}\index{CMobot::moveForward()}}\\
{\LARGE \texttt{CMobot::moveForwardNB()}\index{CMobot::moveForwardNB()}}\\
%\phantomsection
\addcontentsline{toc}{section}{moveForward()}
\addcontentsline{toc}{section}{moveForwardNB()}

\noindent
{\bf Synopsis}
\vspace{-8pt}
\begin{verbatim}
#include <mobot.h>
int CMobot::moveForward(double angle);
int CMobot::moveForwardNB(double angle);
\end{verbatim}

\noindent
{\bf Purpose}\\
Use the faceplates as wheels to roll forward.\\

\noindent
{\bf Return Value}\\
The function returns 0 on success and non-zero otherwise.\\

\noindent
{\bf Parameters}\\
\vspace{-0.1in}
\begin{description}
\item               
\begin{tabular}{p{15 mm}p{145 mm}}
\texttt{angle} & The angle to turn the wheels, specified in degrees.\\
\end{tabular}
\end{description}

\noindent
{\bf Description}\\
\vspace{-12pt}
\begin{quote}
{\bf CMobot::moveForward()}\\
This function causes each of the faceplates to rotate to roll the
mobot forward. The amount to roll the wheels is specified by the argument,
\texttt{angle}.

{\bf CMobot::moveForwardNB()}\\
This function causes each of the faceplates to rotate to roll the
mobot forward. The amount to roll the wheels is specified by the argument,
\texttt{angle}.

This function has both a blocking and non-blocking version.
The blocking version, \texttt{moveForward()}, will block until the
mobot motion has completed. The non-blocking version, \texttt{moveForwardNB()},
will return immediately, and the motion will be performed asynchronously.\\
\end{quote}

\noindent
{\bf See Also}\\
\texttt{moveBackward()}

%\CPlot::\DataThreeD(), \CPlot::\DataFile(), \CPlot::\Plotting(), \plotxy().\\
