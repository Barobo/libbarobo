\noindent
\vspace{5pt}
\rule{4.5in}{0.015in}\\
\noindent
{\LARGE \texttt{CMobot::recordAngleEnd()}\index{CMobot::recordAngleEnd()}}\\
%\phantomsection
\addcontentsline{toc}{section}{recordAngleEnd()}

\noindent
{\bf Synopsis}
\vspace{-8pt}
\begin{verbatim}
#include <mobot.h>
int CMobot::recordAngleEnd(mobotJointId_t id, int &num);
\end{verbatim}

\noindent
{\bf Purpose}\\
End a joint recording process that is currently running.\\

\noindent
{\bf Return Value}\\
This function returns zero on success and -1 on failure.\\

\noindent
{\bf Parameters}\\
\vspace{-0.1in}
\begin{description}
\item               
\begin{tabular}{p{15 mm}p{145 mm}}
\texttt{id} & The joint number. This is an enumerated type 
discussed in Section \ref{sec:mobotJointId_t} on page
\pageref{sec:mobotJointId_t}.\\
\texttt{num} & An integer variable that will be set to the number of recorded elements. \\
\end{tabular}
\end{description}

\noindent
{\bf Description}\\
This function is used in conjunction with the \texttt{recordAngleBegin()} function. 
This function stops the recording process and returns the number of valid data points
allocated for the arrays.\\

\noindent
{\bf Example}\\
\noindent

\noindent
{\bf See Also}\\
\texttt{recordAngleBegin()} \\
%\CPlot::\DataThreeD(), \CPlot::\DataFile(), \CPlot::\Plotting(), \plotxy().\\
