\noindent
\vspace{5pt}
\rule{4.5in}{0.015in}\\
\noindent
{\LARGE \texttt{CMobot::recordWait()}\index{CMobot::recordWait()}}\\
%\phantomsection
\addcontentsline{toc}{section}{recordWait()}

\noindent
{\bf Synopsis}
\vspace{-8pt}
\begin{verbatim}
#include <mobot.h>
int CMobot::recordWait();
\end{verbatim}

\noindent
{\bf Purpose}\\
Wait for a joint recording operation to finish.\\

\noindent
{\bf Return Value}\\
The function returns 0 on success and non-zero otherwise.\\

\noindent
{\bf Parameters}\\
None\\

\noindent
{\bf Description}\\
This function is used in conjunction with the \texttt{recordAngle()} function and/or
the \texttt{recordAngles()} function. The \texttt{recordAngle()} and \texttt{recordAngles()} 
functions both initiate a recording operation that runs in the background for a certain 
amount of time. The \texttt{recordWait()} function is used to pause the main program
until the recording operation has finished. \\

\noindent
{\bf Example}\\
\noindent

\noindent
{\bf See Also}\\
\texttt{recordAngle(), recordAngles()} \\
%\CPlot::\DataThreeD(), \CPlot::\DataFile(), \CPlot::\Plotting(), \plotxy().\\
