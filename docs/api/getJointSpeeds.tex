\noindent
\vspace{5pt}
\rule{4.5in}{0.015in}\\
\noindent
{\LARGE \texttt{CMobot::getJointSpeeds()}\index{CMobot::getJointSpeeds()}}\\
%\phantomsection
\addcontentsline{toc}{section}{getJointSpeeds()}

\noindent
{\bf Synopsis}
\vspace{-8pt}
\begin{verbatim}
#include <mobot.h>
int CMobot::getJointSpeeds(double &speed1, double &speed2, double &speed3, double &speed4);
\end{verbatim}

\noindent
{\bf Purpose}\\
Get the speed settings of all joints on the mobot.\\

\noindent
{\bf Return Value}\\
The function returns 0 on success and non-zero otherwise.\\

\noindent
{\bf Parameters}
\vspace{-0.1in}
\begin{description}
\item               
\begin{tabular}{p{10 mm}p{145 mm}}
\texttt{speed1} & The joint speed setting for joint 1.\\
\texttt{speed2} & The joint speed setting for joint 2.\\
\texttt{speed3} & The joint speed setting for joint 3.\\
\texttt{speed4} & The joint speed setting for joint 4.\\
\end{tabular}
\end{description}

\noindent
{\bf Description}\\
This function is used to retrieve all four joint speed settings of a mobot
simultaneously. The speeds are in degrees per second. \\

\noindent
{\bf Example}\\
\noindent

\noindent
{\bf See Also}\\
\texttt{setJointSpeeds(), getJointSpeedRatios(), getJointSpeed()}

%\CPlot::\DataThreeD(), \CPlot::\DataFile(), \CPlot::\Plotting(), \plotxy().\\
