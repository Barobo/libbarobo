\noindent
\vspace{5pt}
\rule{4.5in}{0.015in}\\
\noindent
{\LARGE \texttt{CMobot::moveJointContinuousNB()}\index{CMobot::moveJointContinuousNB()}}\\
%\phantomsection
\addcontentsline{toc}{section}{moveJointContinuousNB()}

\noindent
{\bf Synopsis}
\vspace{-8pt}
\begin{verbatim}
#include <mobot.h>
int CMobot::moveJointContinuousNB(mobotJointId_t id, mobotJointState_t dir1);
\end{verbatim}

\noindent
{\bf Purpose}\\
Move a joint of a mobot continuously in the specified directions.\\

\noindent
{\bf Return Value}\\
The function returns 0 on success and non-zero otherwise.\\

\noindent
{\bf Parameters}\\
\vspace{-0.1in}
\begin{description}
\item               
\begin{tabular}{p{10 mm}p{145 mm}}
\texttt{id} & The joint number to move. \\
\texttt{dir} &  
This parameter specifies the direction the joint should move. \\
\end{tabular}
\end{description}

\noindent
{\bf Description}\\
The \texttt{dir} parameter specifies the direction the joint should move.
The types
are enumerated in \texttt{mobot.h} and have the following values:
\\
\noindent
\begin{tabular}{p{1.75in}p{4.5in}} \hline 
Value & Description \\
\hline \\
\texttt{MOBOT\_NEUTRAL}& This value indicates that the joint is not moving and is not actuated. The joint is freely backdrivable. \\
\texttt{MOBOT\_FORWARD}& This value indicates that the joint is currently moving forward. \\
\texttt{MOBOT\_BACKWARD}& This value indicates that the joint is currently moving backward. \\
\texttt{MOBOT\_HOLD}& This value indicates that the joint is currently not moving and is holding its current position. The joint is not currently backdrivable. \\
\hline
\end{tabular}\\


More documentation about these types may be found at Section
\ref{sec:mobotJointState_t} on page
\pageref{sec:mobotJointState_t}.  

This function causes joints of a mobot to begin moving at the previously set
speed. The joints will continue moving until the joint hits a joint limit, or
the joint is stopped by setting the speed to zero. This function is a non-blocking
function.\\

\noindent
{\bf Example}\\
\noindent

\noindent
{\bf See Also}\\

%\CPlot::\DataThreeD(), \CPlot::\DataFile(), \CPlot::\Plotting(), \plotxy().\\
