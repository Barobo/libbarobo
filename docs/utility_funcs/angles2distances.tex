\noindent
\vspace{5pt}
\rule{4.5in}{0.015in}\\
\noindent
{\LARGE \texttt{angles2distances()}\index{angles2distances()}}\\
%\phantomsection
\addcontentsline{toc}{section}{angles2distances()}

\noindent
{\bf Synopsis}
\vspace{-8pt}
\begin{verbatim}
#include <mobot.h>
void angles2distances(double radius, double *angles, double *distances, int num);
\end{verbatim}

\noindent
{\bf Purpose}\\
Calculate the distance a wheel has traveled from the radius of the wheel and
the angle the wheel has turned.\\

\noindent
{\bf Return Value}\\
The value returned is the distance traveled by the wheel. If the angle argument is an
array of angles, then the value returned is an array of distances. Each element
of the distance array returned is the distance calculated from the respective
element in the angle array.\\

\noindent
{\bf Parameters}
\vspace{-0.1in}
\begin{description}
\item               
\begin{tabular}{p{15 mm}p{145 mm}}
\texttt{radius} & The radius of the wheel. \\
\texttt{angles} & (In) An array of angle values.\\
\texttt{distances} & (Out) An array that will be filled with distance values.\\
\texttt{num} & The number of elements in the \texttt{angles} array.\\
\end{tabular}
\end{description}

\noindent
{\bf Description}\\
This function calculates the angle a wheel has turned given the wheel 
radius and distance traveled. The equation used is
\begin{equation*}
d = r \theta
\end{equation*}
where $d$ is the distance traveled, $r$ is the radius of the wheel, and $\theta$ is
the angle the wheel has turned in radians.
\\

\noindent
{\bf Example}\\
\noindent

\noindent
{\bf See Also}\\
\texttt{distance2angle()}

%\CPlot::\DataThreeD(), \CPlot::\DataFile(), \CPlot::\Plotting(), \plotxy().\\
