\noindent
\vspace{5pt}
\rule{4.5in}{0.015in}\\
\noindent
{\LARGE \texttt{processRecordedAngles()}\index{processRecordedAngles()}}\\
%\phantomsection
\addcontentsline{toc}{section}{processRecordedAngles()}

\noindent
{\bf Synopsis}
\vspace{-8pt}
\begin{verbatim}
#include <mobot.h>
int processRecordedAngles(double time[], double anglen[], int numElements);
\end{verbatim}

\noindent
{\bf Syntax}
\vspace{-8pt}
\begin{verbatim}
#include <mobot.h>
int processRecordedAngles(double time[], double anglen[], int numElements);
\end{verbatim}

\noindent
{\bf Purpose}\\
This function is used to condition the data obtained by the \texttt{recordAngle()} family 
of functions. \\

\noindent
{\bf Return Value}\\
The return value is the number of elements which have been shifted of the plots. \\

\noindent
{\bf Parameters}
\vspace{-0.1in}
\begin{description}
\item               
\begin{tabular}{p{25 mm}p{145 mm}}
\texttt{time} & The array holding time or "x-axis" values. \\
\texttt{data1} & An array holding data. \\
\texttt{numElements} & The number of elements in the arrays. \\
\end{tabular}
\end{description}

\noindent
{\bf Description}\\
This function conditions recorded data by shifting the graphs to the left
such that the first motion occurs at time 0. \\

\noindent
{\bf Example}\\
\noindent

\noindent
{\bf See Also}\\

%\CPlot::\DataThreeD(), \CPlot::\DataFile(), \CPlot::\Plotting(), \plotxy().\\
