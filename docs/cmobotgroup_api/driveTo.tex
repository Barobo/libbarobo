\noindent
\vspace{5pt}
\rule{4.5in}{0.015in}\\
\noindent
{\LARGE \texttt{CMobotGroup::driveTo()}\index{CMobotGroup::driveTo()}}\\
{\LARGE \texttt{CMobotGroup::driveToNB()}\index{CMobotGroup::driveToNB()}}\\
%\phantomsection
\addcontentsline{toc}{section}{driveTo()}
\addcontentsline{toc}{section}{driveToNB()}

\noindent
{\bf Synopsis}
\vspace{-8pt}
\begin{verbatim}
#include <mobot.h>
int CMobotGroup::driveTo(double angle1, double angle2, double angle3, double angle4);
int CMobotGroup::driveToNB(double angle1, double angle2, double angle3, double angle4);
\end{verbatim}

\noindent
{\bf Purpose}\\
Move all of the joints of a mobot to the specified positions.\\

\noindent
{\bf Return Value}\\
The function returns 0 on success and non-zero otherwise.\\

\noindent
{\bf Parameters}\\
\vspace{-0.1in}
\begin{description}
\item               
\begin{tabular}{p{15 mm}p{105 mm}}
\texttt{angle1} & The absolute position to move joint 1, expressed in degrees. \\
\texttt{angle2} & The absolute position to move joint 2, expressed in degrees. \\
\texttt{angle3} & The absolute position to move joint 3, expressed in degrees. \\
\texttt{angle4} & The absolute position to move joint 4, expressed in degrees. \\
\end{tabular}
\end{description}
\noindent

{\bf Description}\\
\vspace{-12pt}
\begin{quote}
Note that this function is similar to the \texttt{moveTo()} family of functions, except
that this variant ignores the speed settings of the robot joints. This function causes
the robot to move its joints towards its goal using a Proportional-Integral-Derivative (PID)
controller.

{\bf CMobotGroup::driveTo()}\\
This function moves all of the joints of a mobot to the specified absolute positions. 

{\bf CMobotGroup::driveToNB()}\\
This function moves all of the joints of a mobot to the specified absolute positions. 

The function \texttt{driveToNB()} is the non-blocking version of
the \texttt{driveTo()} function, which means that the function will return
immediately and the physical mobot motion will occur asynchronously. For
more details on blocking and non-blocking functions, please refer to 
Section \ref{sec:blocking} on page \pageref{sec:blocking}.\\
\end{quote}

\noindent
{\bf Example}\\
\noindent

\noindent
{\bf See Also}\\

%\CPlot::\DataThreeD(), \CPlot::\DataFile(), \CPlot::\Plotting(), \plotxy().\\
