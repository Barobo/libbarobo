\noindent
\vspace{5pt}
\rule{6.5in}{0.015in}
\noindent
{\LARGE \texttt{getJointAngle()}\index{getJointAngle()}}\\
\phantomsection
\addcontentsline{toc}{section}{getJointAngle()}

\noindent
{\bf Synopsis}\\
\begin{verbatim}
#include "imobot.h"
int CiMobot::getJointAngle(unsigned short id, double *angle);
\end{verbatim}

\noindent
{\bf Purpose}\\
Get the angle of a joint on the iMobot.\\

\noindent
{\bf Return Value}\\
The function returns 0 on success and non-zero otherwise.\\

\noindent
{\bf Parameters}
\vspace{-0.1in}
\begin{description}
\item               
\begin{tabular}{p{10 mm}p{145 mm}}
\texttt{id} & The joint number to pose. \\
\texttt{angle} & The address of a double. This variable will be overwritten
with the current angle of the joint.
\end{tabular}
\end{description}

\noindent
{\bf Description}\\
This function is used to find the angle of a joint. The angles are accurate to
within half a degree. \\

\noindent
{\bf Example}\\
\noindent

\noindent
{\bf See Also}\\

%\CPlot::\DataThreeD(), \CPlot::\DataFile(), \CPlot::\Plotting(), \plotxy().\\
