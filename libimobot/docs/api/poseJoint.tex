\noindent
\vspace{5pt}
\rule{6.5in}{0.015in}
\noindent
{\LARGE \texttt{poseJoint()}\index{poseJoint()}}\\
\phantomsection
\addcontentsline{toc}{section}{poseJoint()}

\noindent
{\bf Synopsis}\\
\begin{verbatim}
#include "imobot.h"
int CiMobot::poseJoint(unsigned short id, double angle);
\end{verbatim}

\noindent
{\bf Purpose}\\
Pose a joint on the iMobot.\\

\noindent
{\bf Return Value}\\
The function returns 0 on success and non-zero otherwise.\\

\noindent
{\bf Parameters}
\vspace{-0.1in}
\begin{description}
\item               
\begin{tabular}{p{10 mm}p{145 mm}}
\texttt{id} & The joint number to pose. \\
\texttt{angle} & The desired angle in degrees.
\end{tabular}
\end{description}

\noindent
{\bf Description}\\
This function instructs the iMobot to pose a joint to a certain angle. This
function does not wait for the motion to complete before returning. Thus,
multiple successive calls to this function may be used to pose multiple joints
simultaneously. The \texttt{waitMotor()} or \texttt{moveWait()} functions should
be used if the program should wait for motions to complete before continuing. \\

\noindent
{\bf Example}\\
See the sample program in Section \ref{subsec:simple.cpp} on page \pageref{subsec:simple.cpp}.
\noindent

\noindent
{\bf See Also}\\
\texttt{moveWait(), waitMotor()}

%\CPlot::\DataThreeD(), \CPlot::\DataFile(), \CPlot::\Plotting(), \plotxy().\\
