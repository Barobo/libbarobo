\noindent
\vspace{5pt}
\rule{6.5in}{0.015in}
\noindent
{\LARGE \texttt{CiMobot::waitMotor()}\index{waitMotor()}}\\
\phantomsection
\addcontentsline{toc}{section}{waitMotor()}

\noindent
{\bf Synopsis}\\
\begin{verbatim}
#include <imobot.h>
int CiMobot::waitMotor(int id);
\end{verbatim}

\noindent
{\bf Purpose}\\
Wait for a joint to stop moving.\\

\noindent
{\bf Return Value}\\
The function returns 0 on success and non-zero otherwise.\\

\noindent
{\bf Parameters}
\vspace{-0.1in}
\begin{description}
\item               
\begin{tabular}{p{10 mm}p{145 mm}}
\texttt{id} & The joint number to wait for. \\
\end{tabular}
\end{description}

\noindent
{\bf Description}\\
This function is used to wait for a joint motion to finish. Functions such as
\texttt{poseJoint()} and \texttt{moveJoint()} do not wait for a joint to finish
moving before continuing to allow multiple joints to move at the same time. The
\texttt{waitMotor()} or \texttt{waitMotor()} functions are used to wait for
robotic motions to complete.

Please note that if this function is called after a motor has been commanded to
turn indefinitely, this function may never return and your program may hang.\\

\noindent
{\bf Example}\\
See the sample program in Section \ref{subsec:simple.cpp} on page \pageref{subsec:simple.cpp}.
\noindent

\noindent
{\bf See Also}\\
\texttt{moveWait()}

%\CPlot::\DataThreeD(), \CPlot::\DataFile(), \CPlot::\Plotting(), \plotxy().\\
