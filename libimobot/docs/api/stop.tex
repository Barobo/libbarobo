\noindent
\vspace{5pt}
\rule{6.5in}{0.015in}
\noindent
{\LARGE \texttt{stop()}\index{stop()}}\\
\phantomsection
\addcontentsline{toc}{section}{stop()}

\noindent
{\bf Synopsis}\\
\begin{verbatim}
#include "imobot.h"
int CiMobot::stop();
\end{verbatim}

\noindent
{\bf Purpose}\\
Stop all current motions on the iMobot.\\

\noindent
{\bf Return Value}\\
The function returns 0 on success and non-zero otherwise.\\

\noindent
{\bf Description}\\
This function stops all currently occuring movements on the iMobot. Internally, this function simply sets all motor speeds to zero. If it is only required to stop a single motor, use the 
\texttt{setMotorSpeed()} function to set the motor's speed to zero. \\

\noindent
{\bf Example}\\
\noindent

\noindent
{\bf See Also}\\
\texttt{setMotorSpeed()}

%\CPlot::\DataThreeD(), \CPlot::\DataFile(), \CPlot::\Plotting(), \plotxy().\\
